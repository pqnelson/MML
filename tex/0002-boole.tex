\section{Boolean Properties of Sets}

Originally, this was the article for identifier \verb#BOOLE# which was
refactored into \verb#XBOOLE_0# and \verb#XBOOLE_1#. In MML version
\verb#7.10.01_4.111.1036#, the definition of the empty set $\emptyset$
changed by introducing the empty adjective.

\subsection{Boolean Properties of Sets: Definitions}
\bigskip

\begin{paracol}{2}
        \sloppy
For simplicity
we adopt the following convention: $x$ will denote an object; $X$, $Y$,
$Z$, $V$ will have denote sets.

\switchcolumn

\begin{mizar}
reserve X, Y, Z for set,
  x, y, z for object;
\end{mizar}

\switchcolumn*\ensurevspace{5cm}

The scheme \emph{Separation} concerns a constant $\mathcal{A}$ that has
the type set and a unary predicate $\mathcal{P}$ and states that the
following holds
\begin{scheme+}
  \ex X \st\ \forall x\\
  \holds x\in X\iff x\in\mathcal{A}\land\mathcal{P}[x]
\end{scheme+}
for all values of the parameters.

\switchcolumn

\begin{mizar}
scheme :: XBOOLE_0:sch 1
  Separation { A()-> set, P[object] } :
  ex X being set st
  for x
  holds x in X iff x in A() & P[x];      
\end{mizar}

\switchcolumn*\ensurevspace{5cm}

We say that a set $X$ is \textit{empty} if and only if
\begin{definition}
\neg\ \ex x \st x\in X.
\end{definition}
%\[\neg\ \ex x \st x\in X.\]

\switchcolumn

\begin{mizar}
definition
  let X be set;
  attr X is empty
  means :: XBOOLE_0:def 1
  not ex x st x in X;
end;
\end{mizar}

\switchcolumn*\ensurevspace{5cm}

\noindent Let us observe there is an empty set.

\switchcolumn

\begin{mizar}
registration
  cluster empty for set;
end;
\end{mizar}

\switchcolumn*\ensurevspace{5cm}

We now define several new constructions. The constant $\emptyset$ is a set, and is
defined by
\begin{definition}
\emptyset\is\ \THE \mathrm{empty}\ \mathrm{set}.
\end{definition}
%\[\IT \is\ \THE \mathrm{empty}\ \mathrm{set} \]

\switchcolumn

\begin{mizar}
definition
  func {} -> set
  equals :: XBOOLE_0:def 2
  the empty set;
\end{mizar}

\switchcolumn*\ensurevspace{5cm}\noindent%
Let $X$, $Y$ be sets. The functor
\[ X\cup Y,\]
with values of type set, is defined by
\begin{definition}
x\in X\cup Y \iff x\in X\lor x\in Y. 
\end{definition}
We can verify this functor is commutative and idempotent.
%\[ x\in\IT \iff x\in X\lor x\in Y. \]

\switchcolumn

\begin{mizar}
  let X,Y be set;
  func X \/ Y -> set
  means :: XBOOLE_0:def 3
  for x holds x in it
  iff x in X or x in Y;
  commutativity;
  idempotence;
\end{mizar}

\switchcolumn*\ensurevspace{5cm}\sloppy\noindent%
The functor
\[ X\cap Y,\]
with values of type set, is defined by
\begin{definition}
x\in X\cap Y \iff x\in X\land x\in Y.
\end{definition}
We can verify this functor is commutative and idempotent.
%\[ x\in\IT \iff x\in X\land x\in Y. \]

\switchcolumn

\begin{mizar}
  func X /\ Y -> set
  means :: XBOOLE_0:def 4
  for x holds x in it
  iff x in X & x in Y;
  commutativity;
  idempotence;
\end{mizar}

\switchcolumn*\ensurevspace{5cm}\noindent%
The functor
\[ X\setminus Y,\]
with values of type set, is defined by
\begin{definition}
x\in X\setminus Y\iff x\in X\land\ \neg x\in Y. 
\end{definition}
%\[ x\in\IT \iff x\in X\land\ \neg x\in Y. \]

\switchcolumn

\begin{mizar}
  func X \ Y -> set
  means :: XBOOLE_0:def 5
  for x holds x in it
  iff x in X & not x in Y;
end;
\end{mizar}

\switchcolumn*\ensurevspace{5cm}
Let $X$, $Y$ be sets. The functor
\[ X\dotminus Y\]
yields a set and is defined by
\begin{definition}
X\dotminus Y = (X \setminus Y)\cup(Y \setminus X).
\end{definition}
We can verify this functor is commutative.
%\[ \IT = (X \setminus Y)\cup(Y \setminus X).\]

\switchcolumn
\begin{mizar}
definition
  let X, Y be set;
  func X \+\ Y -> set
  equals :: XBOOLE_0:def 6
  (X \ Y) \/ (Y \ X);
  commutativity;
\end{mizar}

\switchcolumn*\ensurevspace{5cm}
\noindent The predicate
\[ X\misses Y\]
is defined by
\begin{definition}
X\cap Y = \emptyset. %\forall x\holds x\in X\implies\ \neg x\in Y.
\end{definition}
We can verify this predicate is symmetric.
%\[\forall x\holds x\in X\implies\ \neg x\in Y.\]
\switchcolumn
\begin{mizar}
  pred X misses Y
  means :: XBOOLE_0:def 7
  X /\ Y = {};
  symmetry;
\end{mizar}

\switchcolumn*\ensurevspace{5cm}
\noindent The predicate
\[ X \propersubset Y \]
is defined by
\begin{definition}
X\subset Y\land X\neq Y.
\end{definition}
We can verify this predicate is irreflexive and antisymmetric.
% \[ X\subset Y\land X\neq Y.\]
\switchcolumn
\begin{mizar}
  pred X c< Y
  means :: XBOOLE_0:def 8
  X c= Y & X <> Y;
  irreflexivity;
  asymmetry;
\end{mizar}

\switchcolumn*\ensurevspace{5cm}
\noindent The predicate
\[\areComparable{X}{Y}\]
means
\begin{definition}
X\subset Y\lor Y\subset X.
\end{definition}
We can verify this predicate is reflexive and symmetric.
%\[X\subset Y\lor Y\subset X.\]
\switchcolumn
\begin{mizar}
  pred X,Y are_c=-comparable
  means :: XBOOLE_0:def 9
  X c= Y or Y c= X;
  reflexivity;
  symmetry;
\end{mizar}

\switchcolumn*\ensurevspace{5cm}
\noindent Let us note we can characterize the predicate
\[ X = Y \]
by the equivalent condition
\begin{definition}
X\subset Y\land Y\subset X.
\end{definition}
%\[ X\subset Y\land Y\subset X.\]
\switchcolumn
\begin{mizar}
  redefine pred X = Y
  means :: XBOOLE_0:def 10
  X c= Y & Y c= X;
end;
\end{mizar}

\switchcolumn*\ensurevspace{5cm}
Let $X$ and $Y$ be sets. We introduce the notation $X\meets Y$ as an
antonym for $X\misses Y$.
\switchcolumn
\begin{mizar}
notation
  let X, Y be set;
  antonym X meets Y for X misses Y;
end;
\end{mizar}

\switchcolumn*\ensurevspace{5cm}
We now state couple of propositions:
\begin{theorem}
x\in X\dotminus Y \iff\ \neg(x\in X\iff x\in Y).
\end{theorem}
\begin{theorem+}
(\forall x\holds\ \neg x\in X\iff (x\in Y\iff x\in Z))\\
\implies X=Y\dotminus Z
\end{theorem+}
\switchcolumn
\begin{mizar}
theorem :: XBOOLE_0:1
 x in X \+\ Y
 iff not (x in X iff x in Y);
theorem :: XBOOLE_0:2
 (for x holds not x in X
              iff (x in Y iff x in Z))
  implies X = Y \+\ Z;
\end{mizar}

\switchcolumn*\ensurevspace{5cm}
Let us observe $\emptyset$ is empty.
\switchcolumn
\begin{mizar}
registration
  cluster {} -> empty;
end;
\end{mizar}

\switchcolumn*\ensurevspace{5cm}
Let $x$ be an arbitrary object. Let us observe $\{x\}$ is nonempty.
Further, let $y$ be an arbitrary object. We also observe $\{x,y\}$ is
nonempty. 
\switchcolumn
\begin{mizar}
registration
  let x;
  cluster { x } -> non empty;
  let y;
  cluster { x, y } -> non empty;
end;
\end{mizar}

\switchcolumn*\ensurevspace{5cm}
We observe there exists a non-empty set.
\switchcolumn
\begin{mizar}
registration
  cluster non empty for set;
end;
\end{mizar}

\switchcolumn*\ensurevspace{5cm}
Let $D$ be a non-empty set and let $X$ be an arbitrary set.
We observe $D\cup X$ is non-empty, and that $X\cup D$ is non-empty.
\switchcolumn
\begin{mizar}
registration
  let D be non empty set, X be set;
  cluster D \/ X -> non empty;
  cluster X \/ D -> non empty;
end;
\end{mizar}

\switchcolumn*\ensurevspace{5cm}
We will now state several propositions.
\begin{theorem}
X\meets Y\iff\ \ex x\st x\in X\land x\in Y.
\end{theorem}
\begin{theorem}
X\meets Y\iff\ \ex x\st x\in X\cap Y.
\end{theorem}
\begin{theorem+}
X\misses Y\land x\in X\cup Y \implies\\
x\in X\land x\notin Y\;\;\lor\;\; x\in Y\land x\notin X.
\end{theorem+}
\switchcolumn
\begin{mizar}
theorem :: XBOOLE_0:3
  X meets Y iff ex x st x in X & x in Y;

theorem :: XBOOLE_0:4
  X meets Y iff ex x st x in X /\ Y;

theorem :: XBOOLE_0:5
  X misses Y & x in X \/ Y implies
          x in X & not x in Y
          or x in Y & not x in X;
\end{mizar}

\switchcolumn*\ensurevspace{5cm}
The scheme \textit{Extensionality} deals with sets $\mathcal{X}$ and
$\mathcal{Y}$ and a unary predicate $\mathcal{P}$ and states that:
\begin{scheme}
\mathcal{X} = \mathcal{Y}
\end{scheme}
provided
\begin{itemize}
\item $\forall x\holds x\in\mathcal{X}\iff\mathcal{P}[x]$ and
\item $\forall x\holds x\in\mathcal{Y}\iff\mathcal{P}[x]$.
\end{itemize}
\switchcolumn
\begin{mizar}
scheme :: XBOOLE_0:sch 2
  Extensionality { X,Y() -> set,
                   P[object] } :
  X() = Y()
provided
 for x holds x in X() iff P[x] and
 for x holds x in Y() iff P[x];
\end{mizar}

\switchcolumn*\ensurevspace{5cm}
The scheme \textit{SetEq} deals with a unary predicate $\mathcal{P}$ and
states
\begin{scheme+}
\forall X_{1}, X_{2}\being\ \mbox{sets}\\
\st(\forall x\holds x\in X_{1}\iff\mathcal{P}[x])\land\\
   (\forall x\holds x\in X_{2}\iff\mathcal{P}[x])\\
\holds X_{1} = X_{2}.
\end{scheme+}
\switchcolumn
\begin{mizar}
scheme :: XBOOLE_0:sch 3
  SetEq { P[object] } :
  for X1,X2 being set
  st (for x holds x in X1 iff P[x])
   & (for x holds x in X2 iff P[x])
  holds X1 = X2;
\end{mizar}

\switchcolumn*\ensurevspace{5cm}
Now we shall state several more theorems.
\begin{theorem}
X\propersubset Y\implies\ \ex x\st x\in Y\land x\notin X.
\end{theorem}
\begin{theorem}
X\neq\emptyset\implies\ \ex x\st x\in X
\end{theorem}
\begin{theorem}
X\propersubset Y\implies\ \ex x\st x\in Y\land X\subset Y\setminus\{x\}.
\end{theorem}
\switchcolumn
\begin{mizar}
theorem :: XBOOLE_0:6
  X c< Y implies ex x st x in Y
                        & not x in X;
theorem :: XBOOLE_0:7
  X <> {} implies ex x st x in X;
theorem :: XBOOLE_0:8
 X c< Y implies ex x st x in Y 
                      & X c= Y \ {x};
\end{mizar}

\switchcolumn*\ensurevspace{5cm}
Let $x$, $y$ be arbitrary sets.
We adopt the notation $x\nsubseteq y$ as the antonym for $x\subset y$.
\switchcolumn
\begin{mizar}
notation
  let x,y be set;
  antonym x c/= y for x c= y;
end;
\end{mizar}

\switchcolumn*\ensurevspace{5cm}
Let $x$ be an arbitrary object and let $y$ be a set.
We shall adopt the notation $x\notin y$ as the antonym for $x\in y$.
\switchcolumn
\begin{mizar}
notation
  let x be object,y be set;
  antonym x nin y for x in y;
end;
\end{mizar}

%% \switchcolumn*
%% \switchcolumn
%% \begin{mizar}
%% \end{mizar}
\end{paracol}


\subsection{Boolean Properties of Sets: Theorems}
\bigskip

\begin{paracol}{2}
\end{paracol}