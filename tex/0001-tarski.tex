\section{Tarski Grothendieck Set Theory}

\begin{paracol}{2}
%\hbadness5000
For simplicity we adopt the following convention: $x$, $y$,
$z$, $u$ will denote objects of any type; $N$, $M$, $X$, $Y$, $Z$
will denote objects of the type set.
  
Next we will state two axioms:
\begin{theorem}
x \is \textrm{set},
\end{theorem}
\begin{theorem}
(\forall x\holds x\in X\iff x\in Y)\implies X=Y.
\end{theorem}

\switchcolumn

\begin{mizar}
 reserve x,y,z,u for object;
 reserve N,M,X,Y,Z for set;

:: Everything is a set
theorem :: TARSKI:1
  for x being object holds x is set;

:: Extensionality
theorem :: TARSKI:2 
  (for x being object
   holds x in X iff x in Y)
  implies X = Y;
\end{mizar}

\switchcolumn*\ensurevspace{5cm}

We now introduce two functors. Let us consider $y$. The functor
\[ \{\,y\,\} \]
with values of the type set, is defined by
\begin{definition}
x\in\{\,y\,\}\iff x=y.
\end{definition}
%% \[ x\in\IT \iff x=y.\]
Let us consider $z$. The functor
\[ \{\,y,z\,\} \]
with values of the type set, is defined by
\begin{definition}
x\in\{\,y,z\,\}\iff x=y \lor x=z.
\end{definition}
%% \[ x\in\IT \iff x=y \lor x=z.\]
%% The following axioms hold:
%% \begin{theorem}
%% X=\{y\} \iff\ \forall x \holds x\in X \iff x=y,
%% \end{theorem}
%% \begin{theorem+}
%% X=\{y,z\} \iff\ \forall x\holds x\in X\iff\\
%% %\textbf{iff}~
%% x=y\lor x=z.
%% \end{theorem+}

\switchcolumn

\begin{mizar}
definition
  let y be object;
  func { y } -> set means
:: TARSKI:def 1
    for x being object
    holds x in it iff x = y;

  let z be object;
  func { y, z } -> set means
:: TARSKI:def 2
    x in it iff x = y or x = z;
  commutativity;
end;
\end{mizar}

\switchcolumn*\ensurevspace{5cm}

Let us consider $X$, $Y$. The predicate\index{$\subset$}
\[ X\subset Y\]
is defined by
\begin{definition}
x\in X\implies x\in Y.
\end{definition}
Let us note the predicate $X\subset Y$ is reflexive.
\switchcolumn
\begin{mizar}
definition 
  let X,Y;
  pred X c= Y
  means :: TARSKI:def 3
  for x being object 
  holds x in X implies x in Y;
  reflexivity;
end;
\end{mizar}

\switchcolumn*\ensurevspace{5cm}

Let us consider $X$. The functor\index{$\bigcup$}
\[\bigcup X,\]
with values of the type set, is defined by
\begin{definition}
x\in\bigcup X\iff\ \ex Y\st x\in Y\land Y\in X.
\end{definition}
%\[x\in\IT\ \iff\ \ex Y\st x\in Y\land Y\in X.\] 
%% Then we get
%% \begin{theorem+}
%% X=\bigcup Y\iff\ \forall x\holds x\in
%% X\iff\\
%% \ex Z\st x\in Z\land Z\in Y,
%% \end{theorem+}
%% \begin{theorem}
%% X=\bool Y\iff\ \forall Z\holds 
%% Z\in X \iff Z\subset Y.
%% \end{theorem}

\switchcolumn

\begin{mizar}
definition 
  let X;
  func union X -> set means
:: TARSKI:def 4
    x in it iff ex Y st x in Y & Y in X;
end;
\end{mizar}

\switchcolumn*\ensurevspace{5cm}

The regularity axiom states that\index{Regularity Axiom}
\begin{theorem+}
x\in X\implies\ \ex Y\st Y\in X\land\\
\neg\ \ex x\st x\in X\land x\in Y.
\end{theorem+}

\switchcolumn

\begin{mizar}
:: Regularity
theorem :: TARSKI:3
  x in X implies
   ex Y st Y in X &
     not ex x st x in X & x in Y;
\end{mizar}

\switchcolumn*
\noindent%
Let us note that the predicate $x\in X$ is antisymmetric.

\switchcolumn

\begin{mizar}
definition let x, X be set;
  redefine pred x in X;
  asymmetry;
end;
\end{mizar}

\switchcolumn*\ensurevspace{5cm}

The scheme \textit{Fraenkel}\index{Fraenkel scheme} deals with a constant $\mathcal{A}$ that
has the type set and a binary predicate $\mathcal{P}$ and states that
the following holds:
\begin{scheme+}
  \ex X\st\ \forall x\holds \\
  x\in X\iff
\ex y\st y\in\mathcal{A}\land\mathcal{P}[y,x]
\end{scheme+}
provided the parameters satisfy the following extra condition:
\begin{itemize}
\item $\forall$ $x$, $y$, $z$ $\st$ 
  $\mathcal{P}[x,y]\land\mathcal{P}[x,z]$ $\holds$  $y=z$.
\end{itemize}

\switchcolumn

\begin{mizar}
scheme :: TARSKI:sch 1
 Replacement{ A() -> set,
              P[object,object] }:
 ex X
 st for x being object 
    holds x in X iff
          ex y being object 
          st y in A() & P[y,x]
provided
 for x,y,z being object 
 st P[x,y] & P[x,z]
 holds y = z;
\end{mizar}

\switchcolumn*\ensurevspace{5cm}
Let $x$, $y$ be objects. The functor\index{$\langle x,y\rangle$ (ordered pair)}
\[\langle x,y\rangle,\]
is defined by
\begin{definition}
\langle x,y\rangle = \{\,\{x,y\,\},\{\,x\,\}\,\}.
\end{definition}
%% \[\IT = \{\,\{x,y\,\},\{\,x\,\}\,\}.\]
%% According to the definition
%% \begin{theorem}
%% \langle x,y\rangle = \{\,\{x,y\,\},\{\,x\,\}\,\}.
%% \end{theorem}

\switchcolumn
\begin{mizar}
definition
  let x,y be object;
  func [x,y] -> object equals
:: TARSKI:def 5
    { { x,y }, { x } };
end;
\end{mizar}


\switchcolumn*\ensurevspace{5cm}

Let us consider $X$, $Y$. The predicate\index{$\equipotent$ (equipotent)}
\[X\equipotent Y\]
is defined by the condition:
\begin{definition+}
  \ex Z\st\\
  (\forall x\st x\in X\ex y\st y\in Y\land\langle x,y\rangle\in Z)\land\\
(\forall x\st x\in X\ex y\st y\in Y\land\langle x,y\rangle\in Z)\land\\
\forall x,y,z,u\st \langle x,y\rangle\in Z\land\langle z,u\rangle\in Z\\
\holds x=z\iff y=u.
\end{definition+}
%% \begin{multline*}
%% \ex Z\st\!\! (\forall x\st x\in X\ex 
%% y\st y\in Y\land\langle x,y\rangle\in Z)\land\\
%% (\forall x\st x\in X\ex 
%% y\st y\in Y\land\langle x,y\rangle\in Z)\land\\
%% \forall x,y,z,u\st \langle x,y\rangle\in Z\land\langle z,u\rangle\in Z\\
%% \holds x=z\iff y=u.
%% \end{multline*}

\switchcolumn

\begin{mizar}
definition let X,Y;
  pred X,Y are_equipotent means
:: TARSKI:def 6
  ex Z st
  (for x st x in X
   ex y st y in Y & [x,y] in Z) &
  (for y st y in Y
   ex x st x in X & [x,y] in Z) &
  for x,y,z,u st [x,y] in Z & [z,u] in Z
  holds x = z iff y = u;
end;
\end{mizar}

\switchcolumn*\ensurevspace{5cm}

The Tarski's axiom A claims that\index{Tarski's Axiom A}
\begin{theorem+}
\ex  M \st  N\in M\land\\
(\forall X,Y \holds  X\in M\land
Y\subset X \implies Y\in M)\land\\
(\forall X \holds  X\in M
 \implies \bool X\in M)\land\\
(\forall X \holds  X\subset M \implies 
X\equipotent M \lor X\in M).
\end{theorem+}

\switchcolumn\nopagebreak

\begin{mizar}
theorem :: TARSKI_A:1
 ex M st N in M &
   (for X,Y holds X in M & Y c= X
    implies Y in M) &
   (for X st X in M
    ex Z st Z in M &
            for Y st Y c= X
            holds Y in Z) &
   (for X holds X c= M 
    implies X,M are_equipotent
            or X in M);
\end{mizar}

\end{paracol}
