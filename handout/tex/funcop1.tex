\documentclass{article}

\title{Binary Operations Applied to Functions (FUNCOP-1)}
\author{Andrzej Trybulec}
\date{September 4, 1989}
\begin{document}
\maketitle

Let $f$, $g$, $h$ be functions. Let $A$, $B$, $x$, $y$, $z$ be sets.
We can prove the following proposition:
\begin{thm}
\item\label{funcop1:1} $\Delta_{A}=\langle\id_{A},\id_{A}\rangle$.
\end{thm}

\begin{definition}\index{$f^{\smile}$}
Let $f$ be a function.
We define the term $f^{\smile}$ (Mizar: ``\verb#f~#'') to be the function satisfying
\begin{defn}
\item $\dom(f^{\smile})=\dom(f)$ and for all objects $x\in\dom(f)$,
  \begin{enumerate}[label=(\roman*)]
  \item for all objects $y$ and $z$, if $f(x)=(y,z)$, then $f^{\smile}(x)=(z,y)$;
    and
  \item $f(x)=f^{\smile}(x)$ or there exists objects $y$ and $z$ such
    that $f(x)=(y,z)$.
  \end{enumerate}
\end{defn}
Observe this is involutive (i.e., $(f^{\smile})^{\smile}=f$).
\end{definition}

\begin{remark}
Compare to the converse of a relation $\converse{R}$ in \textsc{relat1}~\ref{relat1:def7}.
In future articles, I may write $\converse{f}$ as synonymous with $f^{\smile}$.
\end{remark}

We can prove the following propositions:
\begin{thm}
\item\label{funcop1:2} $\langle f,g\rangle=\langle g,f\rangle^{\smile}$
\item\label{funcop1:3} $(f|_{A})^{\smile}=f^{\smile}|_{A}$.
\item\label{funcop1:4} $(\Delta_{A})^{\smile}=\Delta_{A}$.
\item\label{funcop1:5} $\langle f,g\rangle|_{A}=\langle f|_{A},g\rangle$.
\item\label{funcop1:6} $\langle f,g\rangle|_{A}=\langle f,g|_{A}\rangle$.
\end{thm}

\begin{definition}\index{$A\constantto z$}%
Let $A$ be a set, let $z$ be an object.
We define the term $A\constantto z$ (Mizar: ``\verb#A --> z#'')
to be the set equal to
\begin{defn}
\item $A\constantto z= A\times\{z\}$.
\end{defn}
\end{definition}

Observe that $A\constantto z$ is function-like and relation-like.

Let $x$, $z$ be objects. We can prove the following proposition:
\begin{thm}
\item\label{funcop1:7} If $x\in A$, then $(A\constantto z)(x)=z$.
\end{thm}
Observe when $A$ is a nonempty set, $x$ is any object, and $a$ is an
element of $A$, we can reduce $(A\constantto z)(a)$ to $a$.

Now we can prove the following two propositions:
\begin{thm}
\item\label{funcop1:8} If $A\neq\emptyset$, then $\rng(A\constantto x)=\{x\}$.
\item\label{funcop1:9} If $\rng(f)=\{x\}$,
  then $f=\dom(f)\constantto x$.
\end{thm}

Observe $\emptyset\constantto x$ is empty, and when $A$ is nonempty we
see $A\constantto x$ is nonempty.

We have the following four results:
\begin{thm}
\item\label{funcop1:10} $\dom(\emptyset\constantto x)=\emptyset$
  and $\rng(\emptyset\constantto x)=\emptyset$.
\item\label{funcop1:11} Suppose every object $z\in\dom(f)$ satisfies $f(z)=x$.
  Then $f=\dom(f)\constantto x$.
\item\label{funcop1:12} $(A\constantto x)|_{B}=A\cap B\constantto x$.
\item\label{funcop1:13} $\dom(A\constantto x)=A$ and
  $\rng(A\constantto x)\subset\{x\}$.
\end{thm}

Observe we can reduce $\dom(X\constantto a)$ to $X$ for any set $X$.
Also observe, for any set $D$, $D\constantto\emptyset$ is empty-yielding.

We can prove the following seven propositions:
\begin{thm}
\item\label{funcop1:14} If $x\in B$, then $(A\constantto x)^{-1}B=A$.
\item\label{funcop1:15} $(A\constantto x)^{-1}\{x\}=A$.
\item\label{funcop1:16} If $x\notin B$, then $(A\constantto x)^{-1}B=\emptyset$.
\item\label{funcop1:17} If $x\in\dom(h)$,
  then $h\circ(A\constantto x)=A\constantto h(x)$.
\item\label{funcop1:18} If $A\neq\emptyset$ and $x\in\dom(h)$,
  then $\dom(h\circ(A\constantto x))\neq\emptyset$.
\item\label{funcop1:19} $(A\constantto x)\circ h=h^{-1}(A)\constantto x$.
\item\label{funcop1:20} $(A\constantto(x,y))^{\smile}=A\constantto(y,x)$.
\end{thm}

\begin{definition}\index{$F(f,g)$}%
Let $F$, $f$, $g$ be functions.
We define the term $F(f,g)$ (Mizar: ``\verb#F .: (f,g)#'') to be the set equal to
\begin{defn}
\item $F(f,g)=F\circ\langle f,g\rangle$.
\end{defn}
\end{definition}

Observe $F(f,g)$ is function-like and relation-like.

We can prove the following propositions:
\begin{thm}
\item\label{funcop1:21} Let $h$ be a function.
  Suppose every set $z\in\dom(F(f,g))$ satisfies $h(z)=F(f(z),g(z))$.
  Then $h=F(f,g)$.
\item\label{funcop1:22} If $x\in\dom(F(f,g))$,
  then $(F(f,g))(x)=F(f(x),g(x))$.
\item\label{funcop1:23} If $f|_{A}=g|_{A}$,
  then $(F(f,h))|_{A}=(F(g,h))|_{A}$.
\item\label{funcop1:24} If $f|_{A}=g|_{A}$,
  then $(F(h,f))|_{A}=(F(h,g))|_{A}$.
\item\label{funcop1:25} $F(f,g)\circ h=F(f\circ h,g\circ h)$.
\end{thm}

\begin{definition}\index{$F(f,x)$}\index{\texttt{F [:] (f,x)}}%
Let $F$, $f$ be functions, let $x$ be an object.
We define the term $F(f,x)$ (Mizar: ``\verb#F[:](f,x)#'') to be the set equal to
\begin{defn}
\item\label{funcop1:def4}%
$F(f,x)=F\circ\langle f,\dom(f)\constantto x\rangle$.
\end{defn}
\end{definition}

Observe $F(f,x)$ is function-like and relation-like.

We now can prove the following five propositions:
\begin{thm}
\item\label{funcop1:26} $F(f,x)=F(f,\dom(f)\constantto x)$.
\item\label{funcop1:27} If $x\in\dom(F(f,z))$,
  then $(F(f,z))(x)=F(f(x),z)$.
\item\label{funcop1:28} If $f|_{A}=g|_{A}$,
  then $(F(f,x))|_{A}=(F(g,x))|_{A}$.
\item\label{funcop1:29} $F(f,x)\circ h=F(f\circ h,x)$
\item\label{funcop1:30} $F(f,x)\circ\id_{A}=F(f|_{A},x)$.
\end{thm}

\begin{definition}\index{$F(x,g)$}\index{\texttt{F [;] (x,g)}}%
Let $F$ be a function, let $x$ be an object, let $g$ be a function.
We define the term $F(x,g)$ (Mizar: ``\verb#F [;] (x,g)#'') to be the set equal to
\begin{defn}
\item\label{funcop1:def5}%
$F(x,g)=F\circ\langle\dom(g)\constantto x,g\rangle$.
\end{defn}
\end{definition}

Observe $F(x,g)$ is function-like and relation-like.

We can prove the following five propositions:
\begin{thm}
\item\label{funcop1:31} $F(x,g)=F(\dom(g)\constantto x,g)$
\item\label{funcop1:32} If $x\in\dom(F(z,f))$, then $(F(z,f))(x)=F(z,f(x))$.
\item\label{funcop1:33} If $f|_{A}=g|_{A}$,
  then $(F(x,f))|_{A}=(F(x,g))|_{A}$.
\item\label{funcop1:34} $F(x,f)\circ h=F(x,f\circ h)$.
\item\label{funcop1:35} $F(x,f)\circ\id_{A}=F(x,f|_{A})$.
\end{thm}

Let $X$ be a nonempty set, $Y$ be a set, $F$ be a binary operator of $X$.
Let $f$, $g$, $h\colon Y\to X$ be functions. Let $x$, $x_{1}$, $x_{2}$
be elements of $X$.

Then we have the following result:
\begin{thm}
\item\label{funcop1:36} $F(f,g)$ is a function from $Y$ to $X$.
\end{thm}

\begin{definition}
Let $X$ be a nonempty set, let $Z$ be a set, let $F$ be a binary
operator of $X$. Let $f,g\colon Z\to X$.
We redefine the type of $F(f,g)$ to be a function from $Z$ to $X$.
\end{definition}

Let $Y$ be a nonempty set, $F$ be a binary operator of $X$.
Let $f$, $g$, $h\colon Y\to X$ be functions. Let $x$, $x_{1}$, $x_{2}$
be elements of $X$.

We can prove the following results:
\begin{thm}
\item\label{funcop1:37} For any element $z$ of $Y$, we have $(F(f,g))(z)=F(f(z),g(z))$.
\item\label{funcop1:38} Let $h\colon Y\to X$.
  Suppose every element $z$ of $Y$ satisfies $h(z)=F(f(z),g(z))$.
  Then $h=F(f,g)$.
\item\label{funcop1:39} Let $g\colon X\to X$.
  Then $F(\id_{X},g)\circ f=F(f,g\circ f)$.
\item\label{funcop1:40} Let $g\colon X\to X$.
  Then $F(g,\id_{X})\circ f=F(g\circ f,f)$.
\item\label{funcop1:41} $F(\id_{X},\id_{X})\circ f=F(f,f)$.
\item\label{funcop1:42} Let $g\colon X\to X$.
  Then $(F(\id_{X},g))(x)=F(x,g(x))$.
\item\label{funcop1:43} Let $g\colon X\to X$.
  Then $(F(g,\id_{X}))(x)=F(g(x),x)$.
\item\label{funcop1:44} $(F(\id_{X},\id_{X}))(x)=F(x,x)$. 
\item\label{funcop1:45} If $x\in B$, then $A\constantto x$ is a function
  from $A$ to $B$.
\end{thm}

\begin{definition}
Let $I$ be a set, let $i$ be an object.
We redefine the type of $I\constantto i$ to be a function from $I$ to $\{i\}$.
\end{definition}

\begin{definition}
Let $B$ be a nonempty set, let $A$ be a set, let $b$ be an element of $B$.
We redefine the type of $A\constantto b$ to be a function from $A$ to $B$.
\end{definition}

We can prove the following results:
\begin{thm}
\item\label{funcop1:46} $A\constantto x$ is a function from $A$ to $X$.
\item\label{funcop1:47} Let $Y$ be a set, $f\colon Y\to X$. Then
  $F(f,x)$ is a function from $Y$ to $X$.
\end{thm}

\begin{definition}
Let $X$ be a nonempty set, let $Z$ be a set.
Let $F$ be a binary operator of $X$, $f\colon Z\to X$, and $x$ be an
element of $X$.
We redefine the type of $F(f,x)$ to be a function from $Z$ to $X$.
\end{definition}

Suppose $Y$ is a nonempty set in the following four propositions:
\begin{thm}
\item\label{funcop1:48} For any element $y$ of $Y$, we have
  $(F(f,x))(y)=F(f(y),x)$.
\item\label{funcop1:49} For any element $y$ of $Y$, we have
  $(F(x,f))(y)=F(x,f(y))$.
\item\label{funcop1:50} $F(\id_{X},x)\circ f=F(f,x)$
\item\label{funcop1:51} $(F(\id_{X},x))(x)=F(x,x)$.
\end{thm}

Now suppose $Y$ is any arbitrary set (possibly empty). We can prove the
following result:
\begin{thm}
\item\label{funcop1:52} $F(x,g)$ is a function from $Y$ to $X$.
\end{thm}

\begin{definition}
Let $X$ be a nonempty set and $Z$ be any set.
Let $f$ be a binary operator of $X$, let $x$ be an element of $X$,
let $g\colon Z\to X$.
We redefine the type of $F(x,g)$ to be a function from $Z$ to $X$.
\end{definition}

Let $Y$ be a nonempty set. We can prove the following propositions:
\begin{thm}
\item\label{funcop1:53} For any element $y$ of $Y$, $(F(x,f))(y)=F(x,f(y))$.
\item\label{funcop1:54} Suppose every element $y$ of $Y$ satisfies $g(y)=F(x,f(y))$.
  Then $g=F(x,f)$.
\end{thm}

Let $Y$ be an arbitrary (possibly empty) set. We can prove the following
three propositions:
\begin{thm}
\item\label{funcop1:55} $F(x,\id_{X})\circ f=F(x,f)$
\item\label{funcop1:56} $(F(x,\id_{X}))(x)=F(x,x)$.
\item\label{funcop1:57} For any nonempty sets $X$, $Y$, $Z$, for any
  function $f\colon X\to Y\times Z$, and for any element $x$ of $X$,
  we have $f^{\smile}(x)=(f(x)_{2},f(x)_{1})$.
\end{thm}

\begin{definition}
Let $X$, $Y$, $Z$ be nonempty sets, let $f\colon X\to Y\times Z$.
We redefine the type of $\rng(f)$ to be a relation of $Y$ and $Z$.
\end{definition}

\begin{definition}
Let $X$, $Y$, $Z$ be nonempty sets, let $f\colon X\to Y\times Z$.
We redefine the type of $f^{\smile}$ to be a function from $X$ to $Z\times Y$.
\end{definition}

We can prove the following proposition:
\begin{thm}
\item\label{funcop1:58} Let $X$, $Y$, $Z$ be nonempty sets, let $f\colon X\to Y\times Z$.
  Then $\rng(f^{\smile})=(\rng(f))^{\sim}$.
\end{thm}

Let $y$ be an arbitrary element of $Y$. We can now prove the following
eight propositions:
\begin{thm}
\item\label{funcop1:59} If $F$ is associative,
  then $F(F(x_{1},f),x_{2})=F(x_{1},F(f,x_{2}))$.
\item\label{funcop1:60} If $F$ is associative,
  then $F(F(f,x),g)=F(f,F(x,g))$.
\item\label{funcop1:61} If $F$ is associative,
  then $F(F(f,g),h)=F(f,F(g,h))$.
\item\label{funcop1:62} If $F$ is associative,
  then $F(F(x_{1},x_{2}),f)=F(x_{1},F(x_{2},f))$.
\item\label{funcop1:63} If $F$ is associative,
  then $F(f,F(x_{1},x_{2}))=F(F(f,x_{1}),x_{2})$.
\item\label{funcop1:64} If $F$ is commutative,
  then $F(x,f)=F(f,x)$.
\item\label{funcop1:65} If $F$ is commutative,
  then $F(f,g)=F(g,f)$.
\item\label{funcop1:66} If $F$ is idempotent,
  then $F(f,f)=f$.
\end{thm}

Now suppose $Y$ is a nonempty set. Let $x$ be an element of $X$, $y$ be
an element of $Y$. We can prove the following propositions:
\begin{thm}
\item\label{funcop1:67} If $F$ is idempotent,
  then $(F(f(y),f))(y)=f(y)$.
\item\label{funcop1:68} If $F$ is idempotent,
  then $(F(f,f(y)))(y)=f(y)$.
\item\label{funcop1:69} For any functions $F$, $f$, $g$,
  if $\rng(f)\times\rng(g)\subset\dom(F)$, then $\dom(F(f,g))=\dom(f)\cap\dom(g)$.
\end{thm}

\begin{definition}
Let $F$ be a function.
We define the attribute $F$ is \define{Function-yielding} to mean
\begin{defn}
\item for every object $x\in\dom(F)$, we have $F(x)$ is a function.
\end{defn}
\end{definition}

Observe there exists a function-yielding function.
Observe when $F$ is a function-yielding function, and $f$ is a function,
then $F\circ f$ is function-yielding.

We can prove the following proposition:
\begin{thm}
\item\label{funcop1:70} $(X\times Y\constantto z)(x,y)=z$.
\end{thm}

\begin{definition}\index{\texttt{(a,b) .--> c}}%
Let $a$, $b$, $c$ be objects.
We define the term $(a,b)\constantto c$ (Mizar: ``\verb#(a,b) .--> c#'') to be the function equal to
\begin{defn}
\item $(a,b)\constantto c = \{(a,b)\}\constantto c$.
\end{defn}
\end{definition}

We can prove the following proposition:
\begin{thm}
\item\label{funcop1:71} For any objects $a$, $b$, and $c$, we have
  $((a,b)\constantto c)(a,b)=c$.
\end{thm}

\begin{definition}\index{\texttt{IFEQ}}%
Let $x$, $y$, $a$, $b$ be objects.
We define the term $\IfEq{x}{y}{a}{b}$ (Mizar: ``\verb#IFEQ(x,y,a,b)#'')
to be the object equal to
\begin{defn}
\item $a$ if $x=y$, otherwise $b$.
\end{defn}
\end{definition}

\begin{definition}
Let $x$, $y$ be objects, let $a$ and $b$ be sets.
We redefine the type of the term $\IfEq{x}{y}{a}{b}$ to be a set.
\end{definition}

\begin{definition}
Let $D$ be a set, let $x$ and $y$ be objects, let $a$ and $b$ be
elements of $D$.
We redefine the type of the term $\IfEq{x}{y}{a}{b}$ to be an element of
$D$.
\end{definition}

\begin{definition}\index{\texttt{x .--> y}}%
Let $x$, $y$ be objects.
We define the term $x\constantto y$ (Mizar: ``\verb#x .--> y#'')
to be the set equal to
\begin{defn}
\item $\{x\}\constantto y$.
\end{defn}
\end{definition}

Observe $x\constantto  y$ is function-like and relation-like and one-to-one.

We can prove the following two propositions:
\begin{thm}
\item\label{funcop1:72} For any objects $x$ and $y$, we have
  $(x\constantto y)(x)=y$.
\item\label{funcop1:73} For any objects $a$ and $b$, for any function
  $f$, we have $a\constantto b\subset f$ if and only if $a\in\dom(f)$
  and $f(a)=b$.
\end{thm}

\begin{notation}
Let $a$, $b$, $c$ be objects.
We define the synonym $(a,b)\mapsto c$ (Mizar: ``\verb#(a,b) :-> c#'')
for $(a,b)\constantto c$.
\end{notation}

\begin{definition}
Let $a$, $b$, $c$ be objects.
We redefine $(a,b)\mapsto c$ (Mizar: ``\verb#(a,b) :-> c#'')
to have its type be a function from $\{a\}\times\{b\}$ to $\{c\}$.
\end{definition}

We can prove the following three propositions:
\begin{thm}
\item\label{funcop1:74} $x\in\dom(x\constantto y)$
\item\label{funcop1:75} If $z\in\dom(x\constantto y)$, then $z=x$.
\item\label{funcop1:76} If $x\notin A$, then $(x\constantto y)|_{A}=\emptyset$.
\end{thm}

\begin{notation}\index{\texttt{x :-> y}}%
Let $x$ and $y$ be objects. We introduce $x\mapsto y$ (Mizar:
``\verb#x :-> y#'') as a synonym for $x\constantto y$.
\end{notation}

\begin{definition}
Let $x$, $y$ be objects.
We redefine the type of the term $x\mapsto y$ to be a function from
$\{x\}$ to $\{y\}$.
\end{definition}

We can prove the following proposition:
\begin{thm}
\item\label{funcop1:77} Let $x$ be an element of $\{a\}$ and $y$ be an
  element of $\{b\}$. Then $((a,b)\mapsto c)(x,y)=c$.
\end{thm}

Observe when $F$ is function-yielding and $C$ is a set, that $F|_{C}$ is function-yielding.
When $A$ is a set and $f$ is a function, we see $A\constantto f$ is
function-yielding.
When $X$ is a set and $a$ is any object, $X\constantto a$ is constant.

We can prove the following three propositions:
\begin{thm}
\item\label{funcop1:78} Let $f$ be a nonempty constant function.
  Then there exists an object $y$ such that for every object $x\in\dom(f)$
  we have $f(x)=y$.
\item\label{funcop1:79} Let $X$ be a nonempty set, let $x$ be any set.
  Then the value of $(X\constantto x)$ is equal to $x$.
\item\label{funcop1:80} Let $f$ be a constant function.
  Then $f=\dom(f)\constantto\mbox{the value of }f$.
\item\label{funcop1:81} $(A\constantto x)(B)\subset\{x\}$.
\item\label{funcop1:82} $x\constantto y$ is an isomorphism of
  $\{(x,x)\}$ with $\{(y,y)\}$.
\item\label{funcop1:83} $\{(x,x)\}$ is isomorphic to $\{(y,y)\}$.
\item\label{funcop1:84} For any function $f$, if $x\in\dom(f)$, then
  $x\constantto f(x)\subset f$.
\item\label{funcop1:85} If $F$ is associative,
  then $F(F(x,f),g)=F(x,F(f,g))$.
\item\label{funcop1:86} Let $x$, $y$, $A$ be sets. If $x\in A$, then
  $(x\constantto y)|_{A}=x\constantto y$.
\end{thm}

\begin{definition}
Let $F$ be a function.
We define the attribute $F$ is \define{Relation-yielding} to mean
\begin{defn}
\item for every set $x\in\dom(F)$, we have $F(x)$ is a relation.
\end{defn}
\end{definition}

Observe every function-yielding function is automatically relation-yielding.

\begin{thm}
\item\label{funcop1:87} Let $X$, $Y$ be sets, let $x$, $y$ be objects.
  Then $X\constantto x$ tolerates $Y\constantto y$ if and only if
  $x=y$ or $X$ misses $Y$.
\item\label{funcop1:88} $\rng(x\constantto y)=\{y\}$.
\item\label{funcop1:89} If $z\in A$, then $(A\constantto x)\circ(y\constantto z)=y\constantto x$.
\end{thm}


\end{document}