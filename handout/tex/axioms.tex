\documentclass{article}
\title{Tarski--Grothendieck Axioms}
\author{Andrzej Trybulec}
%% \makeatletter
%% \@ifclassloaded{combine}
%%   {\let\@begindocumenthook\@empty}
%%   {}
%% \makeatother
\begin{document}
\maketitle
No one refers to these axioms explicitly, but I would like to review
them here. They specify the behaviour of the two predicates:
\begin{enumerate}
\item Equality, defined for any objects $x$ and $y$, we have $x=y$ be primitive
\item Set membership $x\in X$ defined for any object $x$ and set $X$.
\end{enumerate}
The first axiom is implicit in ZF set theory, that Mizar makes
explicit: 

\begin{axiom}[Set axiom]
Every object $x$ is a set.
\end{axiom}

\noindent The next 4 axioms are in ZF.

\begin{axiom}[Extensionality]
Let $X$, $Y$ be sets.
If, for every object $x$ we have $x\in X$ if and only if $x\in Y$,
then $X=Y$.
\end{axiom}

\begin{axiom}[Pair]
Let $x$, $y$ be any object.
There exists a set $Z$ such that for any object $a$,
$a\in Z$ if and only if either $a=x$ or $a=y$.
\end{axiom}

\begin{axiom}[Union]
Let $\mathcal{F}$ be a set.
There exists a set $Z$ such that for any object $x$,
$x\in Z$ if and only if there exists some set $X$ such that $x\in X$ and $X\in\mathcal{F}$.
\end{axiom}

%% theorem :: TARSKI_0:5
%%   for x being object
%%   for X being set st x in X holds
%%   ex Y being set st
%%   (Y in X & ( for x being object holds
%%               ( not x in X or not x in Y ) ) ) ;

\begin{axiom}[Regularity]
Let $X$ be a set. For any object $x$ such that $x\in X$,
there exists a set $Y$ such that $Y\in X$ and there is no object $z$
such that $z\in Y$ and $z\in X$.
\end{axiom}

\begin{axiom}[Schema of Replacement]
Let $P[-,-]$ be a binary predicate, and $\mathcal{F}_{1}$ be a set.
Provided:
\begin{itemize}
\item For any objects $x$, $y$, $z$, if $P[x,y]$ and $P[x,z]$, then $y=z$;
\end{itemize}
Then there exists a set $X$ such that for any object $x$, $x\in X$ if
and only if there exists some object $y$ such that $y\in\mathcal{F}_{1}$
and $P[y,x]$.
\end{axiom}

\noindent The last axiom implies the missing axioms from ZFC (including, yes, choice).

\begin{axiom}[Universe]
``Every set is contained in a Grothendieck universe.''
  Let $N$ be a set. Then there exists a set $U$ such that:
  \begin{enumerate}
  \item $N\in U$;
  \item For any sets $X$ and $Y$ such that $X\in N$ and $Y\subset X$,
    we have $Y\in U$;
  \item For any set $X$ such that $X\in U$ there exists a set $Z$ such that $Z\in U$ and
    every set $Y$ for which $Y\subset X$ has $Y\in Z$;
  \item For any set $X$, if $X\subset U$, then either $X$ is equipotent
    to $U$ or else $X\in U$.
  \end{enumerate}
\end{axiom}

\begin{definition}
  We define equality on any two objects $x$, $y$ as a predicate $x=y$.
  We take equality to be reflexive (for any object $x$, we always have $x=x$)
and symmetric (for any objects $x$ and $y$, if $x=y$, then $y=x$).

We define $x\neq y$ as the antonym of $x=y$.
\end{definition}

\begin{notation}
We use the \textbf{informal notation} $x\notin X$ for $\neg(x\in X)$. This is
later formally defined in \hyperlink{notation:xboole0:nin}{XBOOLE-0}.
\end{notation}
\end{document}