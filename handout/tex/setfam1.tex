\documentclass{article}
\title{Families of Sets (SETFAM-1)}
\author{Beata Padlewska}
%% \makeatletter
%% \@ifclassloaded{combine}
%%   {\let\@begindocumenthook\@empty}
%%   {}
%% \makeatother
\begin{document}
\maketitle
\begin{definition}
Let $X$ be a set.
We define the term $\meet X$ (Mizar: ``\verb#meet X#'') to be the set satisfying:
\begin{defn}
\item if $X\neq\emptyset$ we have for any object $x$, we have $x\in\meet X$ if and only if 
  every set $Y$ such that $Y\in X$ satisfies $x\in Y$; and otherwise
  (when $X$ is empty) $\meet X=\emptyset$.
\end{defn}
\end{definition}

Let $X$, $Y$, $Z$, $Z_{1}$ be sets. We have the following results:
\begin{thm}
\item\label{setfam1:1} $\meet\emptyset=\emptyset$
\item\label{setfam1:2} $\meet X\subset\union X$
\item\label{setfam1:3} If $Z\in X$, then $\meet X\subset Z$.
\item\label{setfam1:4} If $\emptyset\in X$, then $\meet X=\emptyset$.
\item\label{setfam1:5} If $X\neq\emptyset$ and every set $Z_{1}\in X$ contains
  $Z\subset Z_{1}$, then $Z\subset\meet X$.
\item\label{setfam1:6} If $X\neq\emptyset$ and $X\subset Y$, then $\meet Y\subset\meet X$.
\item\label{setfam1:7} If $X\in Y$ and $X\subset Z$, then $\meet Y\subset Z$.
\item\label{setfam1:8} If $X\in Y$ and $X$ misses $Z$,
  then $\meet Y$ misses $Z$.
\item\label{setfam1:9} If $X\neq\emptyset$ and $Y\neq\emptyset$,
  then $\meet(X\cup Y)=(\meet X)\cap(\meet Y$.
\item\label{setfam1:10} $\meet\{X\}=X$.
\item\label{setfam1:11} $\meet\{X,Y\}=X\cap Y$.
\end{thm}

\begin{definition}
Let $F_{x}$ and $F_{y}$ be sets.
\begin{itemize}
\item We define the predicate $F_{x}$ \define{is finer than} $F_{y}$ to mean
\begin{defn}
\item for each set $X$ with $X\in F_{x}$, there exists a set $Y$ such
  that $Y\in F_{y}$ and $X\subset Y$.
\end{defn}
Observe this is a reflexive predicate.
\item We define the predicate $F_{y}$ \define{is coarser than} $F_{x}$ to mean
\begin{defn}
\item for each set $Y$ with $Y\in F_{y}$, there exists a set $X$ such
  that $X\in F_{x}$ and $X\subset Y$.
\end{defn}
Observe this is a reflexive predicate.
\end{itemize}
\end{definition}

Let $F_{x}$, $F_{y}$, $F_{z}$ be sets. We have the following eight theorems:
\begin{thm}
\item\label{setfam1:12} If $F_{x}\subset F_{y}$, then $F_{x}$ is finer
  than $F_{y}$.
\item\label{setfam1:13} If $F_{x}$ is finer than $F_{y}$, then $\union F_{x}\subset \union F_{y}$.
\item\label{setfam1:14} If $F_{y}\neq\emptyset$ and $F_{y}$ is coarser
  than $F_{x}$, then $\meet F_{x}\subset\meet F_{y}$.
\item\label{setfam1:15} $\emptyset$ is finer than $F_{x}$.
\item\label{setfam1:16} If $F_{x}$ is finer than $\emptyset$, then $F_{x}=\emptyset$.
\item\label{setfam1:17} If $F_{x}$ is finer than $F_{y}$ and $F_{y}$ is
  finer than $F_{z}$, then $F_{x}$ is finer than $F_{z}$.
\item\label{setfam1:18} If $F_{x}$ is finer than $\{Y\}$, then every set $X$
  if $X\in F_{x}$ then $X\subset Y$. 
\item\label{setfam1:19} If $F_{x}$ is finer than $\{X,Y\}$, then for any
  set $Z$ if $Z\in F_{x}$ then either $Z\subset X$ or $Z\subset Y$.
\end{thm}

\begin{definition}
  Let $F_{x}$, $F_{y}$ be sets.
We define the new term $\UNION(F_{x},F_{y})$ (Mizar:
  ``\verb#UNION(FX, FY)#'')
to be the set satisfying
\begin{defn}
\item for any set $Z$, $Z\in\UNION(F_{x},F_{y})$ if and only if there
  exists sets $X$ and $Y$ such that $X\in F_{x}$ and $Y\in F_{y}$
  and $Z=X\cup Y$.
\end{defn}
Observe this is commutative in its arguments.

We define the new term $\INTERSECTION(F_{x},F_{y})$ (Mizar:
  ``\verb#INTERSECTION(FX, FY)#'') to be the set satisfying
to be the set satisfying
\begin{defn}
\item for any set $Z$, $Z\in\UNION(F_{x},F_{y})$ if and only if there
  exists sets $X$ and $Y$ such that $X\in F_{x}$ and $Y\in F_{y}$
  and $Z=X\cap Y$.
\end{defn}
Observe this is commutative in its arguments.

We lastly define the new term $\DIFFERENCE(F_{x},F_{y})$ (Mizar:
  ``\verb#DIFFERENCE(FX, FY)#'') to be the set
  satisfying
\begin{defn}
\item for any set $Z$, $Z\in\UNION(F_{x},F_{y})$ if and only if there
  exists sets $X$ and $Y$ such that $X\in F_{x}$ and $Y\in F_{y}$
  and $Z=X\setminus Y$.
\end{defn}
\end{definition}

\begin{remark}
These definitions amount to $\{X~\langle op\rangle~Y\mid X\in F_{x},Y\in F_{y}\}$
where $\langle op\rangle$ is $\cup$, $\cap$, $\setminus$ for UNION,
INTERSECTION, DIFFERENCE.
\end{remark}

Now we can prove the following results:
\begin{thm}
\item\label{setfam1:20} $F_{x}$ is finer than $\UNION(F_{x},F_{x})$.
\item\label{setfam1:21} $\INTERSECTION(F_{x},F_{x})$ is finer than $F_{x}$.
\item\label{setfam1:22} $\DIFFERENCE(F_{x},F_{x})$ is finer than $F_{x}$.
\item\label{setfam1:23} If $F_{x}$ meets $F_{y}$,
  then $(\meet F_{x})\cap(\meet F_{y}) = \meet\INTERSECTION(F_{x},F_{y})$
\item\label{setfam1:24} If $F_{y}\neq\emptyset$, then
  $X\cup\meet F_{y}=\meet\UNION(\{X\},F_{y})$.
\item\label{setfam1:25} $X\cap\union(F_{y})=\union\INTERSECTION(\{X\},F_{y})$.
\item\label{setfam1:26} If $F_{y}\neq\emptyset$,
  then $X\setminus(\union F_{y})=\meet\DIFFERENCE(\{X\},F_{y})$.
\item\label{setfam1:27} If $F_{y}\neq\emptyset$,
  then $X\setminus(\meet F_{y})=\union\DIFFERENCE(\{X\},F_{y})$.
\item\label{setfam1:28} $\union\INTERSECTION(F_{x},F_{y})=(\union F_{x})\cap(\union F_{y})$.
\item\label{setfam1:29} If $F_{x}\neq\emptyset$ and
  $F_{y}\neq\emptyset$, then
  $(\meet F_{x})\cup(\meet F_{y})\subset\meet\UNION(F_{x},F_{y})$.
\item\label{setfam1:30}
  $\meet\DIFFERENCE(F_{x},F_{y})\subset(\meet F_{x})\setminus(\meet F_{y})$.
\end{thm}

\begin{definition}
Let $D$ be a set. We define the new mode \define{Subset-Family of $D$}
(Mizar: ``\verb#Subset-Family of D#'')
is a Subset of $\powerset(D)$.
\end{definition}

Let $D$ be a set.
Let $F$, $G$ be a Subset-Family of $D$. Let $P$ be a subset of $D$.

\begin{definition}
Let $D$ be a set and $F$ a subset-family of $D$.
We redefine the types of terms $\union F$ and $\meet F$ to be a subset of $D$.
\end{definition}

\begin{thm}
\item\label{setfam1:31} If every subset $P$ of $D$ satisfies $P\in F$
  iff $P\in G$, then $F=G$.
\end{thm}

\begin{definition}
Let $D$ be a set and $F$ a subset-family of $D$.
We define the new term $\COMPLEMENT{F}$ (Mizar: ``\verb#COMPLEMENT(F)#'') to be the subset-family of $D$ satisfying
\begin{defn}
\item for every subset $P$ of $D$, we have $P\in\COMPLEMENT{F}$ iff
  $P^{\complement}\in F$.
\end{defn}
Observe this is involutive (i.e., $\COMPLEMENT{(\COMPLEMENT{F})}=F$).
\end{definition}

We have the following three theorems, assuming $F\neq\emptyset$.
\begin{thm}
\item\label{setfam1:32} $\COMPLEMENT{F}\neq\emptyset$.
\item\label{setfam1:33} $\Omega_{D}\setminus(\union F)=\meet\COMPLEMENT{F}$.
\item\label{setfam1:34} $\union\COMPLEMENT{F}=\Omega_{D}\setminus\meet F$.
\end{thm}

Let $X$ be a set, let $F$ and $G$ be subset-families of $X$.
\begin{thm}
\item\label{setfam1:35} Let $P$ be a subset of $X$. Then
  $P^{\complement}\in\COMPLEMENT{F}$ iff $P\in F$.
\item\label{setfam1:36} If $\COMPLEMENT{F}\subset\COMPLEMENT{G}$, then
  $F\subset G$.
\item\label{setfam1:37} $\COMPLEMENT{F}\subset G$ iff $F\subset\COMPLEMENT{G}$.
\item\label{setfam1:38} If $\COMPLEMENT{F}=\COMPLEMENT{G}$, then $F=G$.
\item\label{setfam1:39} $\COMPLEMENT{F\cup G}=(\COMPLEMENT{F})\cup(\COMPLEMENT{G})$.
\item\label{setfam1:40} If $F=\{X\}$, then $\COMPLEMENT{F}=\{\emptyset\}$.
\end{thm}

Observe the complement of an empty subset-family is empty.

\begin{definition}
Let $X$ be a set. We define the attribute $X$ is \define{with non-empty elements}
(Mizar: ``\verb#with_non-empty_elements#'')
to mean
\begin{defn}
\item $\emptyset\notin X$.
\end{defn}
\end{definition}

Observe there exists a nonempty `with non-empty elements' set.
Also observe the enumerated sets $\{x_{1},\dots,x_{n}\}$ are with
non-empty elements for $n=1,\dots,9$.

We can prove the following two propositions:
\begin{thm}
\item\label{setfam1:41} If $\union Y\subset Z$ and $X\in Y$, then
  $X\subset Z$.
\item\label{setfam1:42} For any sets $A$, $B$, and $X$, if
  $X\subset\union(A\cup B)$ and every set $Y$ in $Y\in B$ has $Y$ misses $X$,
  then $X\subset\union A$.
\end{thm}

\begin{definition}
Let $M$ be a set. Let $B$ be a subset-family of $M$.
We define the term $\Intersect\ B$ (Mizar: ``\verb#Intersect B#'') to be
the subset of $M$ equal to
\begin{defn}
\item $\meet B$ if $B\neq\emptyset$, otherwise $M$.
\end{defn}
\end{definition}

We have the following two theorems:
\begin{thm}
\item\label{setfam1:43} Let $X$ and $x$ be sets, let $R$ be a
  subset-family of $X$. If $x\in X$, then $x\in\Intersect\ R$ iff every
  set $Y$ with $Y\in R$ has $x\in Y$.
\item\label{setfam1:44} Let $X$ be a set, $H$ and $J$ be subset-families
  of $X$. If $H\subset J$, then $\Intersect\ J\subset\Intersect\ H$.
\end{thm}

\begin{definition}
Let $E$ be a set. We define the attribute $E$ is \define{empty-membered}
to mean:
\begin{defn}
\item there does not exist a nonempty set $x$ with $x\in E$.
\end{defn}
\end{definition}

\begin{notation}
Observe the antonym of $E$ is empty-membered is $E$ is with nonempty element.
\end{notation}

\begin{definition}
Let $X$ be a set.
We define a new mode \define{Cover of $X$} (Mizar:
``\verb#Cover of X#'') to be the set satisfying
\begin{defn}
\item $X\subset\union(\mbox{it})$.
\end{defn}
\end{definition}

We can prove the following two theorems:
\begin{thm}
\item\label{setfam1:45} Let $X$ be a set, $F$ be a subset-family of $X$.
Then $F$ is a cover of $X$ if and only if $\union F=X$.
\item\label{setfam1:46} $\{\emptyset\}$ is a subset-family of $X$.
\end{thm}

\begin{definition}
Let $X$ be a set, $F$ a subset-family of $X$.
We define the attribute $F$ is \define{with proper subsets} to mean
\begin{defn}
\item $X\notin F$.
\end{defn}
\end{definition}

We can prove the following couple of theorems:
\begin{thm}
\item\label{setfam1:47} Let $T$ be a set, let $F$ and $G$ be
  subset-families of $T$. If $F$ is with proper subsets and $G\subset F$,
  then $G$ is with proper subsets.
\item\label{setfam1:48} Let $T$ be a nonempty set, let $A$ and $B$ be
  `with proper subsets' subset-families of $T$. Then $A\cup B$ is with
  proper subsets.
\end{thm}

\begin{definition}
Let $X$ be a set. We redefine the type of $\powerset(X)$ to be a
subset-family of $X$.
\end{definition}

We conclude by stating we can prove the following theorem:
\begin{thm}
\item\label{setfam1:49} Let $A$ be a nonempty set, let $b$ be any object.
If $A\neq\{b\}$, then there exists an element $a$ of $A$ such that
$a\neq b$.
\end{thm}
\end{document}