\documentclass{article}

\title{Binary Operations (BINOP-1)}
\author{Czes{\l}aw Byli\'nski}
\date{April 14, 1989}

\begin{document}
\maketitle

\begin{definition}
Let $f$ be a \hyperlink{definition:funct1:nm1}{function},
let $a$ and $b$ be objects.
We define the term $f(a,b)$ (Mizar: ``\verb#f . (a, b)#'') to equal
\begin{defn}
\item $f(a,b)=f((a,b))$ (Mizar: ``\verb#f.(a,b) = f.[a,b]#'').
\end{defn}
\end{definition}

\begin{definition}
Let $A$, $B$ be nonempty sets, let $C$ be a set, let $f\colon A\times B\to C$.
Let $a$ be an element of $A$, let $b$ be an element of $B$.
We redefine the type of the term $f(a,b)$ to be an element of $C$.
\end{definition}

Let $x$, $x_{1}$, $y$, $z$ be objects.
We can prove the following two propositions:
\begin{thm}
\item\label{binop1:1} Let $X$, $Y$, $Z$ be sets,
  let $f_{1},f_{2}\colon X\times Y\to Z$.
  If every sets $x$ and $y$ such that $x\in X$ and $y\in Y$ satisfy
  $f_{1}(x,y)=f_{2}(x,y)$,
  then $f_{1}=f_{2}$.
\item\label{binop1:2} Let $X$, $Y$, $Z$ be sets,
  let $f_{1},f_{2}\colon X\times Y\to Z$.
  If every element $a$ of $A$ and $b$ of $B$ satisfies $f_{1}(a,b)=f_{2}(a,b)$,
  then $f_{1}=f_{2}$.
\end{thm}

\begin{definition}
  Let $A$ be a set.
  We define the mode \define{Unary Operator of $A$} to be a function
  from $A$ to $A$.
  We define the mode \define{Binary Operator of $A$} to be a function
  from $A\times A$ to $A$.
\end{definition}

\begin{scheme}[FuncEx2]
Let $\mathcal{X}$, $\mathcal{Y}$, $\mathcal{Z}$ be sets, let $P[-,-,-]$
be a ternary predicate of objects.
There exists a function $f\colon\mathcal{X}\times\mathcal{Y}\to\mathcal{Z}$
such that for all objects $x\in\mathcal{X}$ and $y\in\mathcal{Y}$, we
have $P[x,y,f(x,y)]$; provided
\begin{enumerate}
\item for all objects $x\in\mathcal{X}$ and $y\in\mathcal{Y}$, there
  exists an object $z\in\mathcal{Z}$ such that $P[x,y,z]$.
\end{enumerate}
\end{scheme}

\begin{scheme}[Lambda2]
Let $\mathcal{X}$, $\mathcal{Y}$, $\mathcal{Z}$ be sets, let
$\mathcal{F}(-,-)$ be an object parametrized by any pair of objects.
There exists a function $f\colon\mathcal{X}\times\mathcal{Y}\to\mathcal{Z}$
such that for all objects $x\in\mathcal{X}$ and $y\in\mathcal{Y}$, we
have $f(x,y)=\mathcal{F}(x,y)$; provided
\begin{enumerate}
\item for all objects $x\in\mathcal{X}$ and $y\in\mathcal{Y}$,
  we have $\mathcal{F}(x,y)\in\mathcal{Z}$.
\end{enumerate}
\end{scheme}

\begin{scheme}[FuncEx2D]
Let $\mathcal{X}$, $\mathcal{Y}$, $\mathcal{Z}$ be sets, let $P[-,-,-]$
be a ternary predicate of objects.
There exists a function $f\colon\mathcal{X}\times\mathcal{Y}\to\mathcal{Z}$
such that for elements $x$ of $\mathcal{X}$ and $y$ of $\mathcal{Y}$, we
have $P[x,y,f(x,y)]$; provided
\begin{enumerate}
\item for elements $x$ of $\mathcal{X}$ and $y$ of $\mathcal{Y}$, there
  exists an element $z$ of $\mathcal{Z}$ such that $P[x,y,z]$.
\end{enumerate}
\end{scheme}

\begin{scheme}[Lambda2D]
Let $\mathcal{X}$, $\mathcal{Y}$, $\mathcal{Z}$ be nonempty sets, let
$\mathcal{F}(-,-)$ be an element of $\mathcal{Z}$ parametrized by an
element of $\mathcal{X}$ and an element of $\mathcal{Y}$.
There exists a function $f\colon\mathcal{X}\times\mathcal{Y}\to\mathcal{Z}$
such that for all elements $x$ of $\mathcal{X}$ and $y$ of $\mathcal{Y}$, we
have $f(x,y)=\mathcal{F}(x,y)$.
\end{scheme}

\begin{definition}
  Let $A$ be a set, let $f$ be a binary operator of $A$.
  We define the attribute $f$ is \define{commutative} means
  \begin{defn}
  \item for all elements $a$, $b$ of $A$, we have $f(a,b)=f(b,a)$.
  \end{defn}
  We define the attribute $f$ is \define{associative} means
  \begin{defn}
  \item for all elements $a$, $b$, $c$ of $A$, we have $f(a,f(b,c))=f(f(a,b),c)$.
  \end{defn}
  We define the attribute $f$ is \define{idempotent} means
  \begin{defn}
  \item for all elements $a$ of $A$, we have $f(a,a)=a$.
  \end{defn}
\end{definition}

Observe every binary operator of $\emptyset$ is empty, associative, and
commutative.

\begin{definition}
Let $A$ be a set, let $e$ be an object, let $f$ be a binary operator of
$A$.
We define the predicate \define{$e$ is a left unity with respect to $f$}
(Mizar: ``\verb#is_a_left_unity_wrt#'') to mean
\begin{defn}
\item for all elements $a$ of $A$, we have $f(e,a)=a$.
\end{defn}
We define the predicate \define{$e$ is a right unity with respect to $f$}
(Mizar: ``\verb#is_a_right_unity_wrt#'') to mean
\begin{defn}
\item for all elements $a$ of $A$, we have $f(a,e)=a$.
\end{defn}
\end{definition}

\begin{definition}
Let $A$ be a set, let $e$ be an object, let $f$ be a binary operator of $A$.
We define the predicate \define{$e$ is a unity with respect to $f$}
(Mizar: ``\verb#is_a_unity_wrt#'') to mean
\begin{defn}
\item $e$ is a left unity with respect to $f$, and $e$ is a right unity
  with respect to $f$.
\end{defn}
\end{definition}

Let $A$ be a set.
Let $e$, $e_{1}$, $e_{2}$ be elements of $A$,
let $f$ be a binary operator of $A$.
We can prove the following results:
\begin{thm}
\item\label{binop1:3} $e$ is a unity with respect to $f$ if and only if
  for all elements $a$ of $A$ we have $f(e,a)=a$ and $f(a,e)=a$.
\item\label{binop1:4} If $f$ is commutative,
  then $e$ is a unity with respect to $f$ iff every element $a$ of $A$
  satisfies $f(e,a)=a$.
\item\label{binop1:5} If $f$ is commutative,
  then $e$ is a unity with respect to $f$ iff every element $a$ of $A$
  satisfies $f(a,e)=a$.
\item\label{binop1:6} If $f$ is commutative,
  then $e$ is a unity with respect to $f$ iff $e$ is a left unity with
  respect to $f$.
\item\label{binop1:7} If $f$ is commutative,
  then $e$ is a unity with respect to $f$ iff $e$ is a right unity with
  respect to $f$.
\item\label{binop1:8} If $f$ is commutative,
  then $e$ is a left unity with respect to $f$ iff $e$ is a right unity with
  respect to $f$.
\item\label{binop1:9} If $e_{1}$ is a left unity with respect to $f$,
  and if $e_{2}$ is a right unity with respect to $f$, then $e_{1}=e_{2}$.
\item\label{binop1:10} If $e_{1}$ and $e_{2}$ are unities with respect
  to $f$, then $e_{1}=e_{2}$.
\end{thm}

\begin{definition}
Let $A$ be a set, let $f$ be a binary operator of $A$.
Assume there exists an element $e$ of $A$ which is a unity with respect
to $f$.
We define the term \define{the unity with respect to $f$} (Mizar: ``\verb#the_unity_wrt f#'')
to be the element of $A$ satisfying
\begin{defn}
\item it is a unity with respect to $f$.
\end{defn}
\end{definition}

\begin{definition}
Let $A$ be a set, let $f_{1}$, $f_{2}$ be binary operators of $A$.
We define the predicate $f_{1}$ is \define{left distributive with respect to} $f_{2}$
(Mizar: ``\verb#f1 is_left_distributive_wrt f2#'') means
\begin{defn}
\item for all elements $a$, $b$, $c$ of $A$, we have
  $f_{1}(a,f_{2}(b,c))=f_{2}(f_{1}(a,b), f_{1}(a,c))$.
\end{defn}
We define the predicate $f_{1}$ is \define{right distributive with respect to} $f_{2}$
(Mizar: ``\verb#f1 is_right_distributive_wrt f2#'') means
\begin{defn}
\item for all elements $a$, $b$, $c$ of $A$, we have
  $f_{1}(f_{2}(a,b),c)=f_{2}(f_{1}(a,c), f_{1}(b,c))$.
\end{defn}
\end{definition}

\begin{definition}
Let $A$ be a set, let $f_{1}$, $f_{2}$ be binary operators of $A$.
We define the predicate $f_{1}$ is \define{distributive with respect to}
$f_{2}$ to mean
\begin{defn}
\item $f_{1}$ is both left and right distributive with respect to $f_{2}$.
\end{defn}
\end{definition}

Let $f_{1}$, $f_{2}$ be binary operators of $A$. We can prove the
following six propositions:
\begin{thm}
\item\label{binop1:11} $f_{1}$ is distributive with respect to $f_{2}$
  if and only if for all elements $a$, $b$, $c$ of $A$ we have
  $f_{1}(a,f_{2}(b,c))=f_{2}(f_{1}(a,b),f_{1}(a,c))$ and
  $f_{1}(f_{2}(a,b),c)=f_{2}(f_{1}(a,c),f_{1}(b,c))$.
\item\label{binop1:12} Let $A$ be a nonempty set. If $f_{1}$ be a
  commutative binary operator of $A$,
  then $f_{1}$ is distributive with respect to $f_{2}$ iff for all
  elements $a$, $b$, $c$ of $A$ we have $f_{1}(a,f_{2}(b,c))=f_{2}(f_{1}(a,b),f_{1}(a,c))$.
\item\label{binop1:13} Let $A$ be a nonempty set. If $f_{1}$ be a
  commutative binary operator of $A$,
  then $f_{1}$ is distributive with respect to $f_{2}$ iff for all
  elements $a$, $b$, $c$ of $A$ we have $f_{1}(f_{2}(a,b),c)=f_{2}(f_{1}(a,c),f_{1}(b,c))$.
\item\label{binop1:14} Let $A$ be a nonempty set. If $f_{1}$ is
  commutative, then $f_{1}$ is distributive with respect to $f_{2}$ iff
  $f_{1}$ is left distributive with respect to $f_{2}$. 
\item\label{binop1:15} Let $A$ be a nonempty set. If $f_{1}$ is
  commutative, then $f_{1}$ is distributive with respect to $f_{2}$ iff
  $f_{1}$ is right distributive with respect to $f_{2}$.
\item\label{binop1:16} Let $A$ be a nonempty set. If $f_{1}$ is
  commutative, then $f_{1}$ is left distributive with respect to $f_{2}$ iff
  $f_{1}$ is right distributive with respect to $f_{2}$.
\end{thm}

\begin{definition}
Let $A$ be a set, let $u$ be a unary operator of $A$, let $f$ be a
binary operator of $A$. We define the predicate
$u$ is \define{distributive with respect to} $f$ (Mizar: ``\verb#u is_distributive_wrt f#'')
to mean
\begin{defn}
\item for all elements $a$, $b$ of $A$, we have $u(f(a,b))=f(u(a),u(b))$.\addtocounter{defni}{3}
\end{defn}
\end{definition}

The next 3 definitions were cancelled.

\begin{definition}
Let $A$ be a nonempty set, let $e$ be an object, let $f$ be a binary
operator of $A$.
We redefine the predicate $e$ is a left unity with respect to $f$ to mean
\begin{defn}
\item for all elements $a$ of $A$ we have $f(e,a)=a$.
\end{defn}
We redefine the predicate $e$ is a right unity with respect to $f$ to
mean
\begin{defn}
\item\label{binop1:def17} for all elements $a$ of $A$ we have $f(a,e)=a$.
\end{defn}
\end{definition}

\begin{definition}
Let $A$ be a nonempty set, let $f_{1}$ and $f_{2}$ be binary operators
of $A$.
We redefine the predicate $f_{1}$ is left distributive with respect to
$f_{2}$ to mean
\begin{defn}
\item for all elements $a$, $b$, $c$ of $A$ we have $f_{1}(a,f_{2}(b,c))=f_{2}(f_{1}(a,b),f_{1}(a,c))$.
\end{defn}
We redefine the predicate $f_{1}$ is right distributive with respect to
$f_{2}$ to mean
\begin{defn}
\item for all elements $a$, $b$, $c$ of $A$ we have $f_{1}(f_{2}(a,b),)=f_{2}(f_{1}(a,c),f_{1}(b,c))$.
\end{defn}
\end{definition}

\begin{definition}
Let $A$ be a nonempty set, let $u$ be a unary operator of $A$, let $f$
be a binary operator of $A$. We redefine the predicate $u$ is
distributive with respect to $f$ to mean
\begin{defn}
\item for all elements $a$, $b$ of $A$, we have $u(f(a,b))=f(u(a),u(b))$.
\end{defn}
\end{definition}

We can prove the following four propositions:
\begin{thm}
\item\label{binop1:17} Let $x$ and $y$ be objects, let $f\colon X\times Y\to Z$.
  If $x\in X$ and $y\in Y$ and $Z\neq\emptyset$,
  then $f(x,y)\in Z$.
\item\label{binop1:18} Let $x$, $y$ be objects, let $X$, $Y$, $Z$ be
  sets,
  let $f\colon X\times Y\to Z$, let $g$ be a \hyperlink{definition:funct1:nm1}{function}.
  If $Z\neq\emptyset$, $x\in X$, and $y\in Y$,
  then $(g\circ f)(x,y)=g\bigl(f(x,y)\bigr)$.
\item\label{binop1:19}
  Let $f$ be a \hyperlink{definition:funct1:nm1}{function},
  suppose $\dom(f)=X\times Y$.
  Then $f$ is constant if and only if for all objects $x_{1},x_{2}\in X$
  and $y_{1},y_{2}\in Y$ we have $f(x_{1},y_{1})=f(x_{2},y_{2})$.
\item\label{binop1:20} Let $f_{1}$, $f_{2}$ be partial functions from
  $X\times Y$ to $Z$. Suppose $\dom(f_{1})=\dom(f_{2})$.
  If every $(x,y)\in\dom(f_{1})$ satisfies $f_{1}(x,y)=f_{2}(x,y)$,
  then $f_{1}=f_{2}$.
\end{thm}

\begin{scheme}[PartFuncEx2]
Let $\mathcal{X}$, $\mathcal{Y}$, $\mathcal{Z}$ be sets, let $P[-,-,-]$
be a ternary predicate of objects.
There exists a partial function $f$ from $\mathcal{X}\times\mathcal{Y}$
to $\mathcal{Z}$ such that
\begin{enumerate}[label=(\roman*)]
\item for all objects $x$ and $y$, we have $(x,y)\in\dom(f)$ if and only
  if $x\in\mathcal{X}$ and $y\in\mathcal{Y}$ and there exists an object
  $z$ such that $P[x,y,z]$; and
\item for all objects $x$ and $y$, if $(x,y)\in\dom(f)$, then $P[x,y,f(x,y)]$.
\end{enumerate}
provided:
\begin{enumerate}
\item for all objects $x$, $y$, and $z$, if $x\in\mathcal{X}$, $y\in\mathcal{Y}$, and $P[,y,z]$,
  then we have $z\in\mathcal{Z}$; and
\item for all objects $x$, $y$, $z_{1}$, $z_{2}$,
  if $x\in\mathcal{X}$ and $y\in\mathcal{Y}$ and $P[x,y,z_{1}]$ and
  $P[x,y,z_{2}]$, then $z_{1}=z_{2}$.
\end{enumerate}
\end{scheme}

\begin{scheme}[LambdaR2]
Let $\mathcal{X}$, $\mathcal{Y}$, $\mathcal{Z}$ be sets,
let $\mathcal{F}(-,-)$ be an object parametrized by two objects,
let $P[-,-]$ be a binary predicate of objects.
There exists a partial function $f$ from $\mathcal{X}\times\mathcal{Y}$
to $\mathcal{Z}$ such that
\begin{enumerate}[label=(\roman*)]
\item for all objects $x$ and $y$, we have $(x,y)\in\dom(f)$ if and only
  if $x\in\mathcal{X}$ and $y\in\mathcal{Y}$ and $P[x,y]$; and
\item for all objects $x$ and $y$, if $(x,y)\in\dom(f)$, then $f(x,y)=\mathcal{F}(x,y)$;
\end{enumerate}
provided
\begin{enumerate}
\item for all objects $x$, $y$, if $P[x,y]$, then $\mathcal{F}(x,y)\in\mathcal{Z}$.
\end{enumerate}
\end{scheme}

\begin{scheme}[PartLambda2]
Let $\mathcal{X}$, $\mathcal{Y}$, $\mathcal{Z}$ be sets,
let $\mathcal{F}(-,-)$ be an object parametrized by two objects,
let $P[-,-]$ be a binary predicate of objects.
There exists a partial function $f$ from $\mathcal{X}\times\mathcal{Y}$
to $\mathcal{Z}$ such that
\begin{enumerate}[label=(\roman*)]
\item for all objects $x$ and $y$, we have $(x,y)\in\dom(f)$ if and only
  if $x\in\mathcal{X}$ and $y\in\mathcal{Y}$ and $P[x,y]$; and
\item for all objects $x$ and $y$, if $(x,y)\in\dom(f)$, then $f(x,y)=\mathcal{F}(x,y)$;
\end{enumerate}
provided
\begin{enumerate}
\item for all objects $x$, $y$, if $x\in\mathcal{X}$ and
  $y\in\mathcal{Y}$ and $P[x,y]$, then $\mathcal{F}(x,y)\in\mathcal{Z}$.
\end{enumerate}
\end{scheme}

\begin{scheme}
Let $\mathcal{X}$, $\mathcal{Y}$ be nonempty sets,
let $\mathcal{Z}$ be sets,
let $\mathcal{F}(-,-)$ be an object parametrized by two objects,
let $P[-,-]$ be a binary predicate of objects.
There exists a partial function $f$ from $\mathcal{X}\times\mathcal{Y}$
to $\mathcal{Z}$ such that
\begin{enumerate}[label=(\roman*)]
\item for all elements $x$ of $\mathcal{X}$ and $y$ of $\mathcal{Y}$, we have $(x,y)\in\dom(f)$ if and only $P[x,y]$; and
\item for all elements $x$ of $\mathcal{X}$ and $y$ of $\mathcal{Y}$, if $(x,y)\in\dom(f)$, then $f(x,y)=\mathcal{F}(x,y)$;
\end{enumerate}
provided
\begin{enumerate}
\item for all elements $x$ of $\mathcal{X}$ and $y$ of $\mathcal{Y}$,
  if $P[x,y]$, then $\mathcal{F}(x,y)\in\mathcal{Z}$.
\end{enumerate}
\end{scheme}

\begin{definition}
Let $A$ be a set, let $f$ be a binary operator of $A$, let $a$ and $b$
be elements of $A$.
We redefine the type of $f(a,b)$ to be an element of $A$.
\end{definition}

\begin{definition}
Let $X$, $Y$, $Z$ be sets, let $f_{1},f_{2}\colon X\times Y\to Z$.
We redefine the predicate $f_{1}=f_{2}$ to mean
\begin{defn}
\item for all sets $x$ and $y$, if $x\in X$ and $y\in Y$, then $f_{1}(x,y)=f_{2}(x,y)$.
\end{defn}
\end{definition}

\begin{scheme}
Let $\mathcal{X}$, $\mathcal{Y}$, $\mathcal{Z}$ be sets, let $P[-,-,-]$
be a ternary predicate of objects.
There exists a function $f\colon\mathcal{X}\times\mathcal{Y}\to\mathcal{Z}$
such that for all sets $x$ and $y$, if $x\in\mathcal{X}$ and $y\in\mathcal{Y}$,
then $P[x,y,f(x,y)]$; provided:
\begin{enumerate}
\item for all sets $x$ and $y$, if $x\in\mathcal{X}$ and $y\in\mathcal{Y}$,
then there exists a set $z\in\mathcal{Z}$ such that $P[x,y,z]$.
\end{enumerate}
\end{scheme}

\end{document}