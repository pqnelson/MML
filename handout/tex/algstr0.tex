\documentclass{article}

\title{Basic Algebraic Structures (ALGSTR-0)}
\author{Library Committee}
\date{December 8, 2007}
\begin{document}
\maketitle

\section{Additive structures}

\begin{definition}\index{addMagma}
We define the structure \define{additive Magma} (Mizar: ``\verb#addMagma#'')
which extends 1-sorted structures as
\[\langle\mbox{carrier}, \mbox{an addF}\rangle\]
where addF is a binary operator of the carrier.
\end{definition}

\begin{definition}
Let $M$ be an additive magma, let $x$ and $y$ be elements of $M$.
We define the term $x+y$ to be an element of $M$ meaning
\begin{defn}
\item $x+y = (\mbox{the addF of $M$})(x,y)$.
\end{defn}
\end{definition}

\begin{definition}
We define the \define{Trivial additive magma}  (Mizar: ``\verb#Trivial-addMagma#'')
to be the additive magma such that
\begin{defn}
\item it is the additive magma $\langle\{0\},\mbox{op2}\rangle$.
\end{defn}
\end{definition}

\begin{remark}
Remember, there is only one binary operator on the set $\{0\}$ which is
named in FUNCT-5 as ``op2''.
\end{remark}

\begin{definition}
Let $M$ be an additive magma, let $x$ be an element of $M$.
We define the attribute $x$ is \define{left add-cancelable}  (Mizar: ``\verb#left_add-cancelable#'')
to mean
\begin{defn}
\item for all elements $y$ and $z$ of $M$, if $x+y=x+z$, then $y=z$.
\end{defn}
We define the attribute $x$ is \define{right add-cancelable}  (Mizar: ``\verb#right_add-cancelable#'')
to mean
\begin{defn}%def4
\item for all elements $y$ and $z$ of $M$, if $y+x=z+x$, then $y=z$.
\end{defn}
\end{definition}

\begin{definition}
Let $M$ be an additive magma, let $x$ be an element of $M$. We define
the attribute $x$ is \define{add-cancelable} (Mizar: ``\verb#add-cancelable#'')
to mean
\begin{defn}%def5
\item $x$ is both right and left add-cancelable.
\end{defn}
\end{definition}

\begin{definition}
Let $M$ be an additive magma.
We define the attribute $M$ is \define{left add-cancelable} (Mizar:
``\verb#left_add-cancelable#'') to mean
\begin{defn}%def6
\item every element of $M$ is left add-cancelable.
\end{defn}
We define the attribute $M$ is \define{right add-cancelable} (Mizar:
``\verb#right_add-cancelable#'') to mean
\begin{defn}%def7
\item every element of $M$ is right add-cancelable.
\end{defn}
\end{definition}

\begin{definition}
Let $M$ be an additive magma.
We define the attribute $M$ is \define{add-cancelable} (Mizar:
``\verb#add-cancelable#'') to mean
\begin{defn}%def8
\item $M$ is both left and right add-cancelable.
\end{defn}
\end{definition}

\begin{definition}\index{addLoopStr}%
We define the system of \define{additive loop structures} (Mizar: ``\verb#addLoopStr#'') which extends
the system of ZeroStr and additive magmas to consist of
\[\langle\mbox{carrier},\mbox{addF},\mbox{ZeroF}\rangle\]
where carrier is a set, addF is a binary operator of the carrier, and
ZeroF is an element of the carrier.
\end{definition}

\begin{remark}
Mizar uses the term ``addLoopStr'', but consulting Wanda Szmielew's
\textit{From Affine to Euclidean Geometry: An Axiomatic Approach} (PWN
Polish Scientific Publishers, 1983) we find footnote 5 on page 4
justifying the terminology.
\end{remark}

\begin{definition}\index{Trivial-addLoopStr}%
We define the constant \define{Trivial additive loop structure}
(Mizar: ``\verb#Trivial-addLoopStr#'') to be the additive loop equal to
\begin{defn}%def9
\item it = $\langle\{0\},\mbox{op2},\mbox{op0}\rangle$.
\end{defn}
\end{definition}

\begin{remark}
Recall from FUNCT-5 that ``op0'' is a synonym for 0, considered as an
element of $\{0\}$.
\end{remark}

\begin{definition}
Let $M$ be an additive loop structure, let $x$ be an element of $M$.
We define the attribute $x$ is \define{left complementable}
(Mizar: ``\verb#left_complementable#'') to mean
\begin{defn}
\item there exists an element $y$ of $M$ such that $x+y=0_{M}$.
\end{defn}
We define the attribute $x$ is \define{right complementable}
(Mizar: ``\verb#right_complementable#'') to mean
\begin{defn}
\item there exists an element $y$ of $M$ such that $y+x=0_{M}$.
\end{defn}
\end{definition}

\begin{definition}
Let $M$ be an additive loop structure, let $x$ be an element of $M$.
We define the attribute $x$ is \define{complementable} (Mizar: ``\verb#complementable#'')
to mean
\begin{defn}
\item $x$ is both right and left complementable.
\end{defn}
\end{definition}

\begin{definition}
Let $M$ be an additive loop structure, let $x$ be an element of $M$.
Assume $x$ is left complementable and right add-cancelable.
We define the term $-x$ (Mizar: ``\verb#-x#'') to be an element of $M$
such that
\begin{defn}
\item $(-x) + x=0_{M}$.
\end{defn}
\end{definition}

\begin{definition}
Let $V$ be an additive loop structure, let $v$ and $w$ be elements of $V$.
We define the term $v-w$ (Mizar: ``\verb#v - w#'') to be an element of $V$
such that
\begin{defn}
\item $v-w=v+(-w)$.
\end{defn}
\end{definition}

\begin{definition}
Let $M$ be an additive loop structure.
We define the attribute $M$ is \define{left complementable}
(Mizar: ``\verb#left_complementable#'') to mean
\begin{defn}
\item every element $x$ of $M$ is left cancelable.
\end{defn}
We define the attribute $M$ is \define{right complementable}
(Mizar: ``\verb#right_complementable#'') to mean
\begin{defn}
\item every element $x$ of $M$ is right cancelable.
\end{defn}
\end{definition}

\begin{definition}
Let $M$ be an additive loop structure.
We define the attribute $M$ is \define{complementable}
(Mizar: ``\verb#complementable#'') to mean
\begin{defn}
\item $M$ is both right and left complementable.
\end{defn}
\end{definition}

\section{Multiplicative structures}

\begin{definition}
We define the system of \define{multiplicative magma} structures to
extend 1-sorted structures as
\[\langle\mbox{a carrier}, \mbox{a multF}\rangle\]
where the carrier is a set, and the multF is a binary operator of the
carrier.
\end{definition}

\begin{definition}
Let $M$ be a multiplicative magma, let $x$ and $y$ be elements of $M$.
We define the term $x\cdot y$ (Mizar: ``\verb#x * y#'')
to be an element of $M$ meaning
\begin{defn}%def18
\item $x\cdot y=(\mbox{the multF of $M$})(x,y)$.
\end{defn}
\end{definition}

\begin{definition}
We define the \define{trivial multiplicative magma} (Mizar: ``\verb#Trivial-multMagma#'')
to be a multiplicative magma equal to
\begin{defn}
\item $\langle\{0\}, op2\rangle$.
\end{defn}
\end{definition}

\begin{definition}
Let $M$ be a multiplicative magma, let $x$ be an element of $M$.
We define the attribute $x$ is \define{left multiplicative-cancelable} (Mizar: ``\verb#left_mult-cancelable#'')
to mean
\begin{defn}%def20
\item for all elements $y$ and $z$ of $M$, if $x\cdot y=x\cdot z$,
  then $y=z$.
\end{defn}
We define the attribute $x$ is \define{right multiplicative-cancelable} (Mizar: ``\verb#right_mult-cancelable#'')
to mean
\begin{defn}%def21
\item for all elements $y$ and $z$ of $M$, if $y\cdot x=z\cdot x$,
  then $y=z$.
\end{defn}
\end{definition}

\begin{definition}
Let $M$ be a multiplicative magma, let $x$ be an element of $M$.
We define the attribute $x$ is \define{multiplicative-cancelable} (Mizar: ``\verb#mult-cancelable#'')
to mean
\begin{defn}%def22
\item $x$ is right and left multiplicative-cancelable.
\end{defn}
\end{definition}

\begin{definition}\index{left\textunderscore mult-cancelable}\index{right\textunderscore mult-cancelable}%
Let $M$ be a multiplicative magma.
We define the attribute $M$ is \define{left multiplicative-cancelable} (Mizar: ``\verb#left_mult-cancelable#'')
to mean
\begin{defn}%def23
\item every element of $M$ is left multiplicative-cancelable.
\end{defn}
We define the attribute $M$ is \define{right multiplicative-cancelable} (Mizar: ``\verb#right_mult-cancelable#'')
to mean
\begin{defn}%def24
\item every element of $M$ is right multiplicative-cancelable.
\end{defn}
\end{definition}

\begin{definition}\index{mult-cancelable}
Let $M$ be a multiplicative magma.
We define the attribute $M$ is \define{multiplicative-cancelable} (Mizar: ``\verb#mult-cancelable#'')
to mean
\begin{defn}
\item $M$ is left and right multiplicative-cancelable.
\end{defn}
\end{definition}

\begin{definition}\index{multLoopStr}
We define the system of \define{multiplicative loop structures}
(Mizar: ``\verb#multLoopStr#'') extending the OneStr and multiplicative
magma structures as
\[\langle\mbox{a carrier},\mbox{a multF},\mbox{a OneF}\rangle\]
where the carrier is a set, the multF is a binary operator of the
carrier, and the OneF is an element of the carrier.
\end{definition}

\begin{definition}
We define the constant term \define{trivial multiplicative loop structure}
(Mizar: ``\verb#Trivial-multLoopStr#'')
to be the multiplicative loop equal to
\begin{defn}
\item $\langle\{0\},\mbox{op2},\mbox{op0}\rangle$.
\end{defn}
\end{definition}

\begin{definition}
Let $M$ be a multiplicative loop structure, let $x$ be an element of $M$.
We define the attribute $x$ is \define{left invertible}
(Mizar: ``\verb#left_invertible#'')
to mean
\begin{defn}
\item there exists an element $y$ of $M$ such that $y\cdot x=1_{M}$.
\end{defn}
We define the attribute $x$ is \define{right invertible}
(Mizar: ``\verb#right_invertible#'')
to mean
\begin{defn}
\item there exists an element $y$ of $M$ such that $x\cdot y=1_{M}$.
\end{defn}
\end{definition}

\begin{definition}
Let $M$ be a multiplicative loop structure, let $x$ be an element of $M$.
We define the attribute $x$ is \define{invertible}
(Mizar: ``\verb#invertible#'')
to mean
\begin{defn}
\item $x$ is both left and right invertible.
\end{defn}
\end{definition}

\begin{definition}
Let $M$ be a multiplicative loop structure, let $x$ be an element of $M$.
Assume $x$ is left invertible and right multiplicative-cancelable.
We define the term $/x$ (Mizar: ``\verb#/x#'')
to be an element of $M$ such that
\begin{defn}%def30
\item $(/x)\cdot x=1_{M}$.
\end{defn}
\end{definition}

\begin{definition}
Let $M$ be a multiplicative loop structure.
We define the attribute $M$ is \define{left invertible}
(Mizar: ``\verb#left_invertible#'')
\begin{defn}%def31
\item every element of $M$ is left invertible.
\end{defn}
We define the attribute $M$ is \define{right invertible}
(Mizar: ``\verb#right_invertible#'')
\begin{defn}%def32
\item every element of $M$ is right invertible.
\end{defn}
\end{definition}

\begin{definition}
Let $M$ be a multiplicative loop structure.
We define the attribute $M$ is \define{invertible}
(Mizar: ``\verb#invertible#'')
to mean
\begin{defn}
\item $M$ is both left and right invertible.
\end{defn}
\end{definition}

\section{Almost Cancellable Structures}

\begin{definition}
We define the system of \define{multiplicative loop structures with zero}
(Mizar: ``\verb#multLoopStr_0#'') to extend the multiplicative loop
structure and the ZeroOneStr structure, consisting of
\[\langle \mbox{a carrier}, \mbox{a multF}, \mbox{a ZeroF}, \mbox{a OneF}\rangle\]
where the carrier is a set, the multF is a binary operator of the
carrier, and the ZeroF and OneF are elements of the carrier.
\end{definition}

\begin{definition}
We define the constant \define{trivial multiplicative loop structure
  with zero} (Mizar: ``\verb#Trivial-multLoopStr_0#'') to be equal to
\begin{defn}%def34
\item $\langle\{0\},\mbox{op2},\mbox{op0},\mbox{op0}\rangle$.
\end{defn}
\end{definition}

\skipdefn{}%The \thedefni\ definition was cancelled.

\begin{definition}
Let $M$ be a multiplicative structure with zero.
We define the attribute $M$ is \define{almost left cancelable}
(Mizar: ``\verb#almost_left_cancelable#'')
to mean
\begin{defn}\stepcounter{defni}
\item for all elements $x$ of $M$, if $x\neq0_{M}$, then $x$ is left multiplicative-cancelable.
\end{defn}
We define the attribute $M$ is \define{almost right cancelable}
(Mizar: ``\verb#almost_right_cancelable#'')
to mean
\begin{defn}
\item for all elements $x$ of $M$, if $x\neq0_{M}$, then $x$ is right multiplicative-cancelable.
\end{defn}
\end{definition}

\begin{definition}
Let $M$ be a multiplicative structure with zero.
We define the attribute $M$ is \define{almost cancelable} (Mizar: ``\verb#almost_cancelable#'')
to mean
\begin{defn}
\item $M$ is almost left cancelable and almost right cancelable.
\end{defn}
\end{definition}

\begin{definition}
Let $M$ be a multiplicative structure with zero.
We define the attribute $M$ is \define{almost left invertible}
(Mizar: ``\verb#almost_left_invertible#'')
to mean
\begin{defn}%def39
\item for all elements $x$ of $M$, if $x\neq0_{M}$, then $x$ is left invertible.
\end{defn}
We define the attribute $M$ is \define{almost right invertible}
(Mizar: ``\verb#almost_right_invertible#'')
to mean
\begin{defn}%def40
\item for all elements $x$ of $M$, if $x\neq0_{M}$, then $x$ is right invertible.
\end{defn}
\end{definition}

\begin{definition}
Let $M$ be a multiplicative structure with zero.
We define the attribute $M$ is \define{almost invertible}
(Mizar: ``\verb#almost_invertible#'')
to mean
\begin{defn}
\item $M$ is almost right invertible and almost left invertible.
\end{defn}
\end{definition}

\section{Double Loop Structures}

\begin{definition}\index{\texttt{doubleLoopStr}}%
We define the system of \define{double loop structures}
(Mizar: ``\verb#doubleLoopStr#'') extending additive loop structures and
multiplicative loop structures with zero, consisting of
\[\langle\mbox{a carrier}, \mbox{an addF}, \mbox{a multF}, \mbox{a ZeroF}, \mbox{a OneF}\rangle\]
where the carrier is a set, the addF and multF are binary operators of
the carrier, and the ZeroF and OneF are elements of the carrier.
\end{definition}

\begin{definition}
We define the constant term \define{trivial double loop structure} (Mizar: ``\verb#Trivial-doubleLoopStr#'')
to be equal to
\begin{defn}%def42
\item $\langle\{0\},\mbox{op2},\mbox{op2},\mbox{op0},\mbox{op0}\rangle$.
\end{defn}
\end{definition}

\begin{definition}
Let $M$ be a multiplicative loop structure, let $x$ and $y$ be elements
of $M$.
We define the term $x/y$
(Mizar: ``\verb#x / y#'')
 to be an element of $M$ such that
\begin{defn}%def43
\item $x/y = x\cdot(/y)$.
\end{defn}
\end{definition}

\end{document}