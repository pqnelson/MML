\documentclass{article}

\title{Boolean Domains}
\author{Andrzej Trybulec and Agata Darmochwa\l}
\date{April 14, 1989}
\begin{document}
\maketitle

\begin{definition}
Let $A$ be a set.
We define the attribute that $X$ is \define{cup-closed} to mean
\begin{defn}
\item For all sets $X$ and $Y$, if $X\in A$ and $Y\in A$, the $X\cup Y\in A$.
\end{defn}
We defne the attribute that $A$ is \define{cap-closed} to mean
\begin{defn}
\item For all sets $X$ and $Y$, if $X\in A$ and $Y\in A$, the $X\cap Y\in A$.
\end{defn}
We define the attribute that $A$ is \define{diff-closed} to mean
\begin{defn}
\item For all sets $X$ and $Y$, if $X\in A$ and $Y\in A$, the $X\setminus Y\in A$.
\end{defn}
\end{definition}
\end{document}

\begin{definition}
Let $A$ be a set.
We define the attribute that $A$ is \define{pre-Boolean} (Mizar:
``\verb#preBoolean#'') to mean
\begin{defn}
\item $A$ is cup-closed and diff-closed.
\end{defn}
\end{definition}

We can prove the following result:
\begin{thm}
\item\label{finsub1:1} Let $A$ be a set. Then $A$ is pre-Boolean if and
  only if all sets $X\in A$ and $Y\in A$ satisfy $X\cup Y\in A$ and
  $X\setminus Y\in A$.
\end{thm}

\begin{definition}
Let $A$ be a nonempty pre-Boolean set.
Let $X$, $Y$ be elements of $A$.
We redefine the type of $X\cup Y$ to be an element of $A$, and we
redefine the type of $X\setminus Y$ to be an element of $A$.
\end{definition}

Let $X$, $Y$ be sets.
We can prove the following results:
\begin{thm}
\item\label{finsub1:2} If $X$ and $Y$ are elements of $A$, then $X\cap Y$
  is an element of $A$.
\item\label{finsub1:3} If $X$ and $Y$ are elements of $A$, then
  $X\symdiff Y$ is an element of $A$.
\item\label{finsub1:4} Let $A$ be a nonempty set. Suppose all elements
  $X$ and $Y$ of $A$ satisfy $X\symdiff Y\in A$ and $X\setminus Y\in A$.
  Then $A$ is pre-Boolean.
\item\label{finsub1:5} Let $A$ be a nonempty set. Suppose all elements
  $X$ and $Y$ of $A$ satisfy $X\symdiff Y\in A$ and $X\cap Y\in A$.
  Then $A$ is pre-Boolean.
\item\label{finsub1:6} Let $A$ be a nonempty set. Suppose all elements
  $X$ and $Y$ of $A$ satisfy $X\symdiff Y\in A$ and $X\cup Y\in A$.
  Then $A$ is pre-Boolean.
\end{thm}

\begin{definition}
Let $A$ be a nonempty pre-Boolean set, let $X$ and $Y$ be elements of $A$.
We redefine the type of $X\cap Y$ to be an element of $A$, and we
redefine the type of $X\symdiff Y$ to be an element of $A$.
\end{definition}

Now we can prove the following two propositions:
\begin{thm}
\item\label{finsub1:7} $\emptyset\in A$.
\item\label{finsub1:8} Let $A$ be a set. Then $\powerset(A)$ is pre-Boolean.
\end{thm}

Observe $\powerset(A)$ is pre-Boolean.

We can prove the following result:
\begin{thm}
\item\label{finsub1:9} Let $A$, $B$ be nonempty pre-Boolean sets.
  Then $A\cap B$ is a nonempty pre-Boolean set.
\end{thm}

\begin{definition}
Let $A$ be a set. We define the term $\Fin(A)$ to be the pre-Boolean set
satisfying
\begin{defn}
\item for all sets $X$, we have $X\in\Fin(A)$ if and only if $X\subset A$ and
  $X$ is finite.
\end{defn}
\end{definition}

We have the following results.
\begin{thm}
\item\label{finsub1:10} Let $A$, $B$ be sets. If $A\subset B$, then $\Fin(A)\subset\Fin(B)$.
\item\label{finsub1:11} Let $A$, $B$ be sets. Then $\Fin(A\cap B)=\Fin(A)\cap\Fin(B)$.
\item\label{finsub1:12} Let $A$, $B$ be sets. Then
  $\Fin(A)\cup\Fin(B)\subset\Fin(A\cup B)$.
\item\label{finsub1:13} Let $A$ be a set. Then $\Fin(A)\subset\powerset(A)$.
\item\label{finsub1:14} Let $A$ be a set. If $A$ is finite, then $\Fin(A)=\powerset(A)$.
\item\label{finsub1:15} $\Fin(\emptyset)=\{\emptyset\}$.
\item\label{finsub1:16} Let $A$ be a set, let $X$ be an element of $\Fin(A)$.
  Then $X$ is a subset of $A$.
\item\label{finsub1:17} Let $A$ be a set, let $X$ be a subset of $A$.
  If $A$ is finite, then $X$ is an element of $\Fin(A)$.
\end{thm}