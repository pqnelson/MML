\documentclass{article}

\title{Subsets of Complex Numbers (NUMBERS)}
\author{Andrzej Trybulec}
\date{November 7, 2003}
\begin{document}
\maketitle

\begin{notation}\index{$\NN$}\index{\texttt{NAT}}%
We define the term $\NN$ (Mizar: ``\verb#NAT#'') to be a synonym for the
set $\omega$.
\end{notation}

\begin{definition}\index{Reals}\index{$\RR$}\index{\texttt{REAL}}%
We define the term $\RR$ (Mizar: ``\verb#REAL#'') to be the set equal to
\begin{defn}
\item $\RR := (\RR_{+}\cup(\{0\}\times\RR_{+}))\setminus\{(0,0)\}$.
\end{defn}
\end{definition}

\begin{definition}\index{Complex numbers}\index{$\CC$}\index{\texttt{COMPLEX}}%
We define the term $\CC$ (Mizar: ``\verb#COMPLEX#'') to be the set equal to
\begin{defn}
\item $\CC:=(\Funcs(\{0,1\},\RR)\setminus\{x\in\Funcs(\{0,1\},\RR)\mid x(1)=0\})\cup\RR$.
\end{defn}\index{Rational numbers}\index{$\QQ$}\index{\texttt{RAT}}%
We define the term $\QQ$ (Mizar: ``\verb#RAT#'') to be the set equal to
\begin{defn}
\item $\QQ:=(\QQ_{+}\cup(\{0\}\times\QQ_{+}))\setminus\{(0,0)\}$.
\end{defn}\index{Integers}\index{$\ZZ$}\index{\texttt{INT}}%
We define the term $\ZZ$ (Mizar: ``\verb#INT#'') to be the set equal to
\begin{defn}
\item $\ZZ:=(\NN\cup(\{0\}\times\NN))\setminus\{(0,0)\}$.
\end{defn}
\end{definition}
\begin{remark}
For the rationals and integers, we interpret $(0,x)$ as $-x$.
For the complex numbers, we interpret $z_{0}+z_{1}\cdot\sqrt{-1}$
as a function $z\colon\{0,1\}\to\RR$.
We ``surgically remove'' subsets to make sure $\NN\subset\ZZ\subset\QQ\subset\RR\subset\CC$.
\end{remark}

\begin{remark}\index{Number!Naming conventions}\index{Idiom!Number names}%
It is idiomatic Mizar to use \texttt{SCREAMING CAP CASE} for sets of
numbers, and \texttt{PascalCase} for the type of numbers. For example,
in the article \texttt{QUATERNI}, the set of Quaternions is defined as
\texttt{QUATERNION} and the type is \texttt{Quaternion}.

The only exception to this rule is the extended reals, whose set is
\verb#ExtREAL#. The set of extended naturals (found in the
\verb#COUNTERS# article) follows this pattern of prefixing \verb#Ext# to
the set name (\verb#ExtNAT# for the set of extended natural numbers).

It will be useful to define attributes for \verb#Number# to reflect the
type of number it is (real, ext-real, complex, rational, etc.). These
attributes are in \texttt{kebab-lower-case}.
\end{remark}

\begin{definition}
We redefine the type of the term $0$ to be an element of $\omega$.
\end{definition}

We can prove the following results:
\begin{thm}
\item\label{numbers:1} $\RR\properSubset\CC$.
\item\label{numbers:2} $\QQ\properSubset\RR$
\item\label{numbers:3} $\QQ\properSubset\CC$
\item\label{numbers:4} $\ZZ\properSubset\QQ$
\item\label{numbers:5} $\ZZ\properSubset\RR$
\item\label{numbers:6} $\ZZ\properSubset\CC$
\item\label{numbers:7} $\NN\properSubset\ZZ$
\item\label{numbers:8} $\NN\properSubset\QQ$
\item\label{numbers:9} $\NN\properSubset\RR$
\item\label{numbers:10} $\NN\properSubset\CC$
\item\label{numbers:11} $\RR\subset\CC$
\item\label{numbers:12} $\QQ\subset\RR$
\item\label{numbers:13} $\QQ\subset\CC$
\item\label{numbers:14} $\ZZ\subset\QQ$
\item\label{numbers:15} $\ZZ\subset\RR$
\item\label{numbers:16} $\ZZ\subset\CC$
\item\label{numbers:17} $\NN\subset\ZZ$
\item\label{numbers:18} $\NN\subset\QQ$
\item\label{numbers:19} $\NN\subset\RR$
\item\label{numbers:20} $\NN\subset\CC$.
\item\label{numbers:21} $\RR\neq\CC$
\item\label{numbers:22} $\QQ\neq\RR$
\item\label{numbers:23} $\QQ\neq\CC$
\item\label{numbers:24} $\ZZ\neq\QQ$
\item\label{numbers:25} $\ZZ\neq\RR$
\item\label{numbers:26} $\ZZ\neq\CC$
\item\label{numbers:27} $\NN\neq\ZZ$
\item\label{numbers:28} $\NN\neq\QQ$
\item\label{numbers:29} $\NN\neq\RR$
\item\label{numbers:30} $\NN\neq\CC$.
\end{thm}

\begin{definition}\index{Reals!Extended}\index{Extended Reals}\index{$\ExtRR$}%
We define the term $\ExtRR$ (Mizar: ``\verb#ExtREAL#'') to be the set
equal to
\begin{defn}
\item $\ExtRR := \RR\cup\{\RR,(0,\RR)\}$.
\end{defn}
\end{definition}
\begin{remark}
We interpret $(0,\RR)$ as $-\infty$ and $\RR$ as $+\infty$.
\end{remark}

We have the following results:
\begin{thm}
\item\label{numbers:31} $\RR\subset\ExtRR$
\item\label{numbers:32} $\RR\neq\ExtRR$
\item\label{numbers:33} $\RR\properSubset\ExtRR$
\end{thm}

Observe $\ZZ$, $\QQ$, $\RR$, $\CC$ are all infinite sets.

\end{document}