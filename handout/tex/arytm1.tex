\documentclass{article}

\title{Non negative real numbers. Part II (ARYTM-1)}
\author{Andrzej Trybulec}
\date{March 7, 1998}
\begin{document}
\maketitle

Let $x$, $y$, $z$ be elements of $\RR_{+}$. We can prove the following:
\begin{thm}
\item\label{arytm1:1} If $x+y=y$, then $x=0$.
\item\label{arytm1:2} If $x\cdot y=0$, then either $x=0$ or $y=0$.
\item\label{arytm1:3} If $x\leq y$ and $y\leq z$, then $x\leq z$.
\item\label{arytm1:4} If $x\leq y$ and $y\leq x$, then $x=y$.
\item\label{arytm1:5} If $x\leq y$ and $y=0$, then $x=0$.
\item\label{arytm1:6} If $x=0$, then $x\leq y$.
\item\label{arytm1:7} $x\leq y$ if and only if $x+z\leq y+z$.
\item\label{arytm1:8} If $x\leq y$, then $x\cdot z\leq y\cdot z$.
\end{thm}

\begin{definition}\index{\texttt{-\textquotesingle}}\index{Subtraction!of Positive Reals}\index{Subtraction!\texttt{-\textquotesingle}}%
Let $x$, $y$ be elements of $\RR_{+}$.
We define the term $x-y$ (Mizar: ``\verb#x -' y#'') to be the element of
$\RR_{+}$ satisfying
\begin{defn}
\item $(x-y)+y=x$ if $y\leq x$; otherwise $x-y:=0$.
\end{defn}
\end{definition}

\begin{remark}
In Mizar, it is idiomatic to use \verb#-'# for subtraction between
positive numbers. We will have, in future articles, theorems
establishing ``compatibility'' of this subtraction with the ``usual''
subtraction operator. 
\end{remark}

We have the following results
\begin{thm}
\item\label{arytm1:9} $x\leq y$ or $x - y\neq0$.
\item\label{arytm1:10} If $x\leq y$ or $y-x=0$, then $x=y$.
\item\label{arytm1:11} $x - y\leq x$.
\item\label{arytm1:12} If $y\leq x$ and $y\leq z$, then $x+(z-y)=(x-y)+z$
\item\label{arytm1:13} If $z\leq y$, then $x + (y - z) = (x + y) - z$.
\item\label{arytm1:14} If $z\leq x$ and $y\leq z$, then $(x - z)+y=x-(z-y)$.
\item\label{arytm1:15} If $y\leq x$ and $y\leq z$, then $(z - y)+x=(x-y)+z$.
\item\label{arytm1:16} If $x\leq y$, then $z-y\leq z-x$.
\item\label{arytm1:17} If $x\leq y$, then $x-z\leq y-z$.
\end{thm}

\begin{definition}
Let $x$, $y$ be elements of $\RR_{+}$.
We define the term $x-y$ (Mizar: ``\verb#x - y#'') to be the set equal
to
\begin{defn}
\item $x - y$ (Mizar: ``\verb#x -' y#'') if $y\leq x$; otherwise it is $(0,y-x)$.
\end{defn}
\end{definition}

We can prove the following results:
\begin{thm}
\item\label{arytm1:18} $x-x=0$.
\item\label{arytm1:19} If $x=0$ and $y\neq0$, then $x-y=(0,y)$.
\item\label{arytm1:20} If $z\leq y$, then $x+(y-z)=(x+y)-z$.
\item\label{arytm1:21} If $y<z$, then $x-(z-y)=x+y-z$.
\item\label{arytm1:22} If $y\leq x$ and $z<y$, then $x-(y-z)=x-y+z$.
\item\label{arytm1:23} If $x<y$ and $z<y$, then $x-(y-z)=z-(y-x)$.
\item\label{arytm1:24} If $y\leq x$, then $x-(y+z)=x-y-z$.
\item\label{arytm1:25} If $x\leq y$ and $z\leq y$, then $(y-z)-x=(y-x)-z$.
\item\label{arytm1:26} If $z\leq y$, then $x\cdot(y-z)=(x\cdot y)-(x\cdot z)$.
\item\label{arytm1:27} If $y<z$ and $x\neq0$, then $(0,\cdot(z-y))=(x\cdot y)-(x\cdot z)$.
\item\label{arytm1:28} If $y-z\neq0$, $z\leq y$, and $x\neq0$,
  then $(x\cdot z)-(x\cdot y)=(0,x\cdot(y-z))$.
\end{thm}

\end{document}