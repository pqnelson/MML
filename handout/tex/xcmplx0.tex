\documentclass{article}
\title{Complex Numbers --- Basic Definitions (XCMPLX-0)}
\author{Library Committee}
\date{March 7, 2003}
\begin{document}
\maketitle

\begin{definition}
We define the term $\I$ to be the number equal to
\begin{defn}
\item $\I := (0,1)\constantto(0,1)$
\end{defn}
Let $c$ be a Number.
We define the attribute $c$ is \define{complex} to mean
\begin{defn}
\item $c\in\CC$.
\end{defn}
\end{definition}

Observe $\I$ is complex. Observe there exist complex numbers (and
complex Numbers).

\begin{definition}
We define the mode, a \define{Complex} is a complex Number.
\end{definition}

Observe Complex is a set.

\skipdefn

\begin{definition}
Let $x$, $y$ be Complex.
We define the term $x + y$ (Mizar: ``\verb#x + y#'') to be the number satisfying
\begin{defn}[start=4]
\item There exists elements $x_{1}$, $x_{2}$, $y_{1}$, $y_{2}$ of $\RR$
  such that $x=(x_{1},x_{2})$, $y=(y_{1},y_{2})$, and $x+y=(+(x_{1},y_{1}),+(x_{2},y_{2}))$.
\end{defn}
Observe this is commutative (i.e., $x+y=y+x$).

We define the term $x\cdot y$ (Mizar: ``\verb#x * y#'') to be the number
satisfying
\begin{defn}
\item There exists elements $x_{1}$, $x_{2}$, $y_{1}$, $y_{2}$ of $\RR$
  such that $x=(x_{1},x_{2})$, $y=(y_{1},y_{2})$, and $x+y=(+(\times(x_{1},y_{1}),-(x_{2},y_{2})),+(\times(x_{1},y_{2}),\times(x_{2},y_{1})))$.
\end{defn}
Observe this is commutative (i.e., $x\cdot y=y\cdot x$).
\end{definition}

Observe for any Complex $z$, $z'$, we have $z+z'$ is complex and $z\cdot z'$
is complex.

\begin{definition}
Let $z$ be Complex.
We define the term $-z$ (Mizar: ``\verb#- z#'') to be the Complex satisfying
\begin{defn}
\item $z + (-z)=0$.
\end{defn}
Observe this is involutive (i.e., $-(-z)=z$).

We define the term $z^{-1}$ (Mizar: ``\verb#z "#'') to be the Complex
satisfying
\begin{defn}
\item $z\cdot(z^{-1})=1$ if $z\neq0$, otherwise $z^{-1}=0$.
\end{defn}
Observe this is involutive (i.e., $(z^{-1})^{-1}=z$).
\end{definition}

\begin{definition}
Let $x$, $y$ be Complex.
We define the term $x - y$ (Mizar: ``\verb#x - y#'') to be the number
equal to
\begin{defn}
\item $x - y := x + (-y)$.
\end{defn}
We define the term $x/y$ (Mizar: ``\verb#x / y#'') to be the number
equal to
\begin{defn}
\item $x/y := x\cdot(y^{-1})$.
\end{defn}
\end{definition}

Observe for any Complex $x$ and $y$, we have $x-y$ is complex and $x/y$
is complex.

Observe, if an object is natural, then it is automatically complex.

We observe there exists a zero Complex, and a nonzero Complex.

Observe when $x$ is a nonzero Complex, that $-x$ and $x^{-1}$ are both nonzero.
Further, when $y$ is a nonzero Complex, that $x\cdot y$ is nonzero and
$x/y$ is nonzero.

Observe any element of $\RR$ is automatically complex, and any element
of $\CC$ is automatically complex.

When we have a Complex $z$, we reduce $\In{z}{\CC}$ to $z$ automatically.

\end{document}