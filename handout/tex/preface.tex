\chapter*{Preface}

This is an attempt to organize the results of the Mizar Mathematical
Library in conventional notation, in the order which the Mizar computer
program reads the contents of the library. The library is presented as a
series of articles, the table of contents should be read as if it were a
collection of articles presented at a hypothetical conference. The
authors of each article has been credited with their work, except in the
case when the ``Encyclopedia of Mathematics'' entries are credited to
the Library Committee (whomever they may be).

I have taken the liberty to insert remarks, which are asides to clarify
matters which confounded me when I started learning Mizar. With the
exception of the first article, the contents are precisely that which
could be found in the MML article as distributed with Mizar. The first
article discusses the axiomatization of set theory, which most students
are not really familiar with in the first place. Since the article
\verb#TARSKI_0# (and \verb#TARSKI_A#) is fairly spartan, I have included more
commentary \emph{outside} of remarks, reordered the axioms, and
explicitly compared it to the usual \textsf{ZF} axiomatization.

The numbering of theorems, definitions, and schemes are faithful to the
numbering found in the Mizar Mathematical Library. Proofs are omitted,
because the reader may consult the MML to find the proofs in
surprisingly readable format. When definitions are introduced, I almost
always include a parenthetic to note what the Mizar code looks like. The
only exceptions are when ``\verb#snake_case#'' is used by the Mizar code
which reflects a sequence of words that should be separated by spaces
--- e.g., ``\verb#being_of_order_0#'' clearly encodes the mathematical
notion ``being of order $0$'' and that's how Mizar formalizes it.

I have omitted registrations haphazardly. This is done out of laziness
and taste, rather than anything else.

The goal of this handout is to accelerate the reader's familiarization
with the vast Mathematical Library. As I mentioned, the articles are
presented in linear order. Mostly. There is some important exception
when I deviated with this principle, and that's when different fields
``branch'' into their own. For example, Group Theory is formalized
largely independent of Point-set Topology. Therefore, I separated the
Group Theory articles from the Topology articles, so the content may be
``fresh'' in the reader's mind. These are separated into their own
parts, and within each part the library is presented linearly (as
ordered by the contents of the \verb#mml.lar# file).

I hope this will prove useful to inquiring minds.

\aufm{Alex Nelson\\
  Los Angeles, California\\
  6 December 2023}
