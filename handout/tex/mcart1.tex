\documentclass{article}

\title{Tuples, Projections and Cartesian Products (MCART-1)}
\author{Andrzej Trybulec}
\date{March 30, 1989}
\begin{document}
\maketitle

Let $X$ be a set, let $v$, $x$, $y$ be objects.
\begin{thm}
\item\label{mcart1:1} (Cancelled)
\item\label{mcart1:2} (Cancelled)
\item\label{mcart1:3} (Cancelled)
\item\label{mcart1:4} (Cancelled)
\item\label{mcart1:5} (Cancelled)
\item\label{mcart1:6} (Cancelled)
\item\label{mcart1:7} (Cancelled)
\item\label{mcart1:8} (Cancelled)
\item\label{mcart1:9} If $X\neq\emptyset$,
  then there exists an object $v$ such that $v\in X$
  and for any objects $x$, $y$ we have
  either $x\notin X$ and $y\notin X$ or $v\neq(x,y)$. 
\item\label{mcart1:10} If $z\in X\times Y$, then $z_{1}\in X$ and
  $z_{2}\in Y$.
\item\label{mcart1:11} (Cancelled)
\item\label{mcart1:12} If $z\in\{x\}\times Y$, then $z_{1}=x$ and
  $z_{2}\in Y$.
\item\label{mcart1:13} If $z\in X\times\{y\}$, then $z_{1}\in X$ and $z_{2}=y$.
\item\label{mcart1:14} If $z\in\{x\}\times\{y\}$, then $z_{1}=x$ and $z_{2}=y$.
\item\label{mcart1:15} If $z\in\{x_{1},x_{2}\}\times Y$, then either
  $z_{1}=x_{1}$ or $z_{1}=x_{2}$ and
  $z_{2}\in Y$.
\item\label{mcart1:16} If $z\in X\times\{y_{1},y_{2},\}$, then $z_{1}\in X$
  and either $z_{2}=y_{1}$ or $z_{2}=y_{2}$.
\item\label{mcart1:17} If $z\in\{x_{1},x_{2}\}\times \{y\}$, then either
  $z_{1}=x_{1}$ or $z_{1}=x_{2}$ and
  $z_{2}=y$.
\item\label{mcart1:18} If $z\in \{x\}\times\{y_{1},y_{2},\}$, then $z_{1}=x$
  and either $z_{2}=y_{1}$ or $z_{2}=y_{2}$.
\item\label{mcart1:19} If $z\in \{x_{1},x_{2}\}\times\{y_{1},y_{2},\}$,
  then either $z_{1}=x_{1}$ or $z_{1}=x_{2}$ and either $z_{2}=y_{1}$ or $z_{2}=y_{2}$.
\item\label{mcart1:20} If there exists objects $y$ and $z$ such that
  $x=(y,z)$,
  then $x\neq x_{1}$ and $x\neq x_{2}$.
\end{thm}

Let $R$ be a relation. We have the following results:
\begin{thm}
\item\label{mcart1:21} If $x\in R$, then $x=(x_{1},x_{2})$.
\item\label{mcart1:22} If $X\neq\emptyset$ and $Y\neq\emptyset$,
  then every element $x$ of $X\times Y$ satisfies $x=(x_{1},x_{2})$.
\item\label{mcart1:23}
  $\{x_{1},x_{2}\}\times\{y_{1},y_{2}\}=\{(x_{1},y_{1}), (x_{1},y_{2}),(x_{2},y_{1}),(x_{2},y_{2})\}$.
\item\label{mcart1:24} If $X\neq\emptyset$ and $Y\neq\emptyset$,
  then every element $x$ of $X\times Y$ satisfies $x\neq x_{1}$ and
  $x\neq x_{2}$.
\end{thm}

\section{Triples, Quadruples, Cartesian products of three sets}

\begin{thm}
\item\label{mcart1:25} (Cancelled)
\item\label{mcart1:26} If $X\neq\emptyset$, then there exists an object
  $v$ such that $v\in X$ and there are no objects $x$, $y$, $z$ such
  that $v=(x,y,z)$ and either $x\in X$ or $y\in X$.
\item\label{mcart1:27} (Cancelled)
\item\label{mcart1:28} (Cancelled)
\item\label{mcart1:29} (Cancelled)
\item\label{mcart1:30} If $X\neq\emptyset$, then there exists an object
  $v$ such that $v\in X$ and there are no objects $x_{1}$, $x_{2}$,
  $x_{3}$, $x_{4}$ such
  that $v=(x_{1},x_{2},x_{3},x_{4})$ and either $x_{1}\in X$ or $x_{2}\in X$.
\item\label{mcart1:31} $X_{1}\neq\emptyset$ and $X_{2}\neq\emptyset$ and
  $X_{3}\neq\emptyset$ if and only if $X_{1}\times X_{2}\times X_{3}\neq\emptyset$.
\item\label{mcart1:32} Suppose $X_{1}\neq\emptyset$ and $X_{2}\neq\emptyset$ and
  $X_{3}\neq\emptyset$. If $X_{1}\times X_{2}\times X_{3}=Y_{1}\times Y_{2}\times Y_{3}$,
  then $X_{1}=Y_{1}$, $X_{2}=Y_{2}$, and $X_{3}=Y_{3}$.
\item\label{mcart1:33} If $X_{1}\times X_{2}\times X_{3}\neq\emptyset$
  and $X_{1}\times X_{2}\times X_{3}=Y_{1}\times Y_{2}\times Y_{3}$,
  then $X_{1}=Y_{1}$, $X_{2}=Y_{2}$, and $X_{3}=Y_{3}$.
\item\label{mcart1:34} If $X\times X\times X=Y\times Y\times Y$, then $X=Y$.
\item\label{mcart1:35} $\{x_{1}\}\times\{x_{2}\}\times\{x_{3}\}=\{(x_{1},x_{2},x_{3})\}$.
\item\label{mcart1:36} $\{x_1,y_1\}\times\{x_2\}\times\{x_3\}=\{ (x_1,x_2,x_3),(y_1,x_2,x_3) \}$
\item\label{mcart1:37} $\{x_{1}\}\times\{x_{2},y_{2}\}\times\{x_{3}\}=\{(x_{1},x_{2},x_{3}),(x_{1},y_{2},x_{3})\}$.
\item\label{mcart1:38} $\{x_{1}\}\times\{x_{2}\}\times\{x_{3},y_{3}\}=\{(x_{1},x_{2},x_{3}),(x_{1},x_{2},y_{3})\}$.
\item\label{mcart1:39} $\{x_1,y_1\}\times\{x_2,y_{2}\}\times\{x_3\}=\{ (x_1,x_2,x_3),(y_1,x_2,x_3),(x_1,y_2,x_3),(y_1,y_2,x_3) \}$
\item\label{mcart1:40} $\{x_1,y_1\}\times\{x_2\}\times\{x_3,y_{3}\}=\{ (x_1,x_2,x_3),(y_1,x_2,x_3),(x_1,x_2,y_3),(y_1,x_2,y_3) \}$
\item\label{mcart1:41} $\{x_{1}\}\times\{x_{2},y_{2}\}\times\{x_{3},y_{3}\}=\{(x_{1},x_{2},x_{3}),(x_{1},y_{2},x_{3}),(x_{1},x_{2},y_{3}),(x_{1},y_{2},y_{3})\}$.
\item\label{mcart1:42} $\{x_1,y_1\}\times\{x_2,y_{2}\}\times\{x_3,y_{3}\}=\{ (x_1,x_2,x_3),(x_1,y_2,x_3),(x_{1},x_{2},y_{3}),(x_{1},y_{2},y_{3}),(y_1,x_2,x_3),(y_1,y_2,x_3),(y_{1},x_{2},y_{3}),(y_{1},y_{2},y_{3}) \}$
\end{thm}

The first four definitions are cancelled.

\begin{definition}
  Let $X_{1}$, $X_{2}$, $X_{3}$ be nonempty sets.
  Let $x$ be an element of $X_{1}\times X_{2}\times X_{3}$.
  We redefine \eqref{xtuple0:defn6} the term $x_{\mathbf{1},3}$ (Mizar: ``\verb#x`1_3#'') to be an element of
  $X_{1}$ such that
  \begin{defn}[start=5]
  \item $x=(x_{1},x_{2},x_{3})$ and $x_{\mathbf{1},3}=x_{1}$.
  \end{defn}
  We redefine the term $x_{\mathbf{2},3}$ to be an element of
  $X_{2}$ such that
  \begin{defn}
  \item $x=(x_{1},x_{2},x_{3})$ and $x_{\mathbf{2},3}=x_{2}$.
  \end{defn}
  We redefine the term $x_{\mathbf{3},3}$ to be an element of
  $X_{3}$ such that
  \begin{defn}
  \item $x=(x_{1},x_{2},x_{3})$ and $x_{\mathbf{3},3}=x_{3}$.
  \end{defn}
\end{definition}

\begin{thm}
\item\label{mcart1:43} (Cancelled)
\item\label{mcart1:44} (Cancelled)
\item\label{mcart1:45} If $X\subset X\times Y\times Z$ or
  $X\subset Y\times Z\times X$ or $X\subset Z\times X\times Y$,
  then $X=\emptyset$.
\item\label{mcart1:46} (Cancelled)
\item\label{mcart1:47} Let $X_{1}$, $X_{2}$, $X_{3}$ be nonempty sets.
  Every element $x$ of $X_{1}\times X_{2}\times X_{3}$ satisfies
  $x\neq x_{\mathbf{1},3}$ and $x\neq x_{\mathbf{2},3}$ and
  $x\neq x_{\mathbf{3},3}$.
\item\label{mcart1:48} If $X_{1}\times X_{2}\times X_{3}$ meets
  $Y_{1}\times Y_{2}\times Y_{3}$, then $X_{1}$ meets $Y_{1}$ and
  $X_{2}$ meets $Y_{2}$ and $X_{3}$ meets $Y_{3}$.
\end{thm}

\section{Cartesian Products of Four Sets}
\begin{thm}
\item\label{mcart1:49} $X_{1}\times X_{2}\times X_{3}\times X_{4}=((X_{1}\times X_{2})\times X_{3})\times X_{4}$
\item\label{mcart1:50} $(X_{1}\times X_{2})\times X_{3}\times X_{4}=X_{1}\times X_{2}\times X_{3}\times X_{4}$.
\item\label{mcart1:51} $X_{1}\neq\emptyset$, $X_{2}\neq\emptyset$,
 $X_{3}\neq\emptyset$, and $X_{4}\neq\emptyset$ if and only if $X_{1}\times X_{2}\times X_{3}\times X_{4}\neq\emptyset$.
\item\label{mcart1:52} If $X_{1}\neq\emptyset$, $X_{2}\neq\emptyset$,
 $X_{3}\neq\emptyset$, $X_{4}\neq\emptyset$, and $X_{1}\times X_{2}\times X_{3}\times X_{4}=Y_{1}\times Y_{2}\times Y_{3}\times Y_{4}$,
then $X_{1}=Y_{1}$ and $X_{2}=Y_{2}$ and $X_{3}=Y_{3}$ and $X_{4}=Y_{4}$.
\item\label{mcart1:53} If $X_{1}\times X_{2}\times X_{3}\times X_{4}\neq\emptyset$
  and $X_{1}\times X_{2}\times X_{3}\times X_{4}=Y_{1}\times Y_{2}\times Y_{3}\times Y_{4}$,
then $X_{1}=Y_{1}$ and $X_{2}=Y_{2}$ and $X_{3}=Y_{3}$ and $X_{4}=Y_{4}$.
\item\label{mcart1:54} If $X\times X\times X\times X=Y\times Y\times Y\times Y$,
  then $X=Y$.
\item\label{mcart1:55} (Cancelled)
\item\label{mcart1:56} (Cancelled)
\item\label{mcart1:57} (Cancelled)
\end{thm}

\begin{definition}
Let $X_{1}$, $X_{2}$, $X_{3}$, $X_{4}$ be nonempty sets.
Let $x$ be an element of $X_{1}\times X_{2}\times X_{3}\times X_{4}$.
We redefine \eqref{xtuple0:def10} the term $x_{\mathbf{1},4}$ (Mizar: ``\verb#x`1_4#'')
to be an element of $X_{1}$ satisfying
\begin{defn}
\item If $x=(x_{1},x_{2},x_{3},x_{4})$, then $x_{\mathbf{1},4}=x_{1}$.
\end{defn}
We redefine the term $x_{\mathbf{2},4}$ (Mizar: ``\verb#x`2_4#'')
to be an element of $X_{2}$ satisfying
\begin{defn}
\item If $x=(x_{1},x_{2},x_{3},x_{4})$, then $x_{\mathbf{2},4}=x_{2}$.
\end{defn}
We redefine the term $x_{\mathbf{3},4}$ (Mizar: ``\verb#x`3_4#'')
to be an element of $X_{3}$ satisfying
\begin{defn}
\item If $x=(x_{1},x_{2},x_{3},x_{4})$, then $x_{\mathbf{3},4}=x_{3}$.
\end{defn}
We redefine the term $x_{\mathbf{4},4}$ (Mizar: ``\verb#x`4_4#'')
to be an element of $X_{4}$ satisfying
\begin{defn}
\item If $x=(x_{1},x_{2},x_{3},x_{4})$, then $x_{\mathbf{4},4}=x_{4}$.
\end{defn}
\end{definition}

\begin{thm}
\item\label{mcart1:58} Let $X_{1}$, $X_{2}$, $X_{3}$, $X_{4}$ be nonempty sets.
  Every element $x$ of $X_{1}\times X_{2}\times X_{3}\times X_{4}$
  satisfies
  $x\neq x_{\mathbf{1},4}$, $x\neq x_{\mathbf{2},4}$, $x\neq x_{\mathbf{3},4}$, and $x\neq x_{\mathbf{4},4}$.
\item\label{mcart1:59} If $X_{1}\subset X_{1}\times X_{2}\times X_{3}\times X_{4}$
  or $X_{1}\subset X_{2}\times X_{3}\times X_{4}\times X_{1}$
  or $X_{1}\subset X_{3}\times X_{4}\times X_{1}\times X_{2}$
  or $X_{1}\subset X_{4}\times X_{1}\times X_{2}\times X_{3}$,
  then $X_{1}=\emptyset$.
\item\label{mcart1:60} If $X_{1}\times X_{2}\times X_{3}\times X_{4}$ meets
  $Y_{1}\times Y_{2}\times Y_{3}\times Y_{4}$, then
  $X_{i}$ meets $Y_{i}$ for all $i=1,2,3,4$.
\item\label{mcart1:61} $\{x_{1}\}\times\{x_{2}\}\times\{x_{3}\}\times\{x_{4}\}=\{(x_{1},x_{2},x_{3},x_{4})\}$.
\end{thm}

We can prove the following two propositions concerning ordered pairs:
\begin{thm}
\item\label{mcart1:62} If $X\times Y\neq\emptyset$, then every element
  $x$ of $X\times Y$ satisfies $x\neq x_{1}$ and $x\neq x_{2}$.
\item\label{mcart1:63} If $x\in X\times Y$,
  then $x\neq x_{1}$ and $x\neq x_{2}$.
\end{thm}

\section{Triples}

Let $X_{1}$, $X_{2}$, $X_{3}$ be nonempty sets. Let $x$ be an element of
$X_{1}\times X_{2}\times X_{3}$.
\begin{thm}
\item\label{mcart1:64} If $x=(x_{1},x_{2},x_{3})$, then
  $x_{\mathbf{1},3}=x_{1}$ and
  $x_{\mathbf{2},3}=x_{2}$ and
  $x_{\mathbf{3},3}=x_{3}$.
\item\label{mcart1:65} Let $x_{1}$ be an element of $X_{1}$,
  $x_{2}$ be an element of $X_{2}$,
  $x_{3}$ be an element of $X_{3}$.
  Suppose $x=(x_{1},x_{2},x_{3})$ implies $y_{1}=x_{1}$.
  Then $y_{1}=x_{\mathbf{1},3}$.
\item\label{mcart1:66} Let $x_{1}$ be an element of $X_{1}$,
  $x_{2}$ be an element of $X_{2}$,
  $x_{3}$ be an element of $X_{3}$.
  Suppose $x=(x_{1},x_{2},x_{3})$ implies $y_{2}=x_{2}$.
  Then $y_{2}=x_{\mathbf{2},3}$.
\item\label{mcart1:67} Let $x_{1}$ be an element of $X_{1}$,
  $x_{2}$ be an element of $X_{2}$,
  $x_{3}$ be an element of $X_{3}$.
  Suppose $x=(x_{1},x_{2},x_{3})$ implies $y_{3}=x_{3}$.
  Then $y_{3}=x_{\mathbf{3},3}$.
\end{thm}

Now let $X_{1}$, $X_{2}$, $X_{3}$ be arbitrary sets.
\begin{thm}
\item\label{mcart1:68} If $z\in X_{1}\times X_{2}\times X_{3}$,
  then there exists objects $x_{1}$, $x_{2}$, $x_{3}$ such that
  $x_{1}\in X_{1}$ and $x_{2}\in X_{2}$ and $x_{3}\in X_{3}$ and
  $z=(x_{1},x_{2},x_{3})$. 
\item\label{mcart1:69} $(x_{1},x_{2},x_{3})\in X_{1}\times X_{2}\times X_{3}$
  if and only if $x_{1}\in X_{1}$ and $x_{2}\in X_{2}$ and $x_{3}\in X_{3}$.
\item\label{mcart1:70} Suppose every object $z$ satisfies $z\in Z$
  if and only if there exists objects $x_{1}$, $x_{2}$, $x_{3}$ such that
  $x_{1}\in X_{1}$ and $x_{2}\in X_{2}$ and $x_{3}\in X_{3}$ and
  $z=(x_{1},x_{2},x_{3})$. Then $Z=X_{1}\times X_{2}\times X_{3}$.
\item\label{mcart1:71} (Cancelled)
\item\label{mcart1:72} Let $X_{1}$, $X_{2}$, $X_{3}$ be nonempty sets.
  Let $A_{1}$ be a nonempty subset of $X_{1}$,
  let $A_{2}$ be a nonempty subset of $X_{2}$,
  let $A_{3}$ be a nonempty subset of $X_{3}$.
  Let $x$ be an element of $X_{1}\times X_{2}\times X_{3}$.
  If $x\in A_{1}\times A_{2}\times A_{3}$,
  then $x_{\mathbf{1},3}\in A_{1}$ and
  $x_{\mathbf{2},3}\in A_{2}$ and
  $x_{\mathbf{3},3}\in A_{3}$.
\item\label{mcart1:73} If $X_{1}\subset Y_{1}$ and
  $X_{2}\subset Y_{2}$ and
  $X_{3}\subset Y_{3}$,
  then $X_{1}\times X_{2}\times X_{3}\subset Y_{1}\times Y_{2}\times Y_{3}$.
\end{thm}

\section{Quadruples}

Let $X_{1}$, $X_{2}$, $X_{3}$, $X_{4}$ be nonempty sets.
Let $x$ be an element of $X_{1}\times X_{2}\times X_{3}\times X_{4}$.
\begin{thm}
\item\label{mcart1:74} Let $x_{1}$, $x_{2}$, $x_{3}$, $x_{4}$ be sets.
  If $x=(x_{1},x_{2},x_{3},x_{4})$, then
  $x_{\mathbf{1},4}=x_{1}$ and
  $x_{\mathbf{2},4}=x_{2}$ and
  $x_{\mathbf{3},4}=x_{3}$ and
  $x_{\mathbf{4},4}=x_{4}$.
\item\label{mcart1:75} Let $y_{1}$ be an object.
  Suppose for any elements $x_{1}$ of $X_{1}$, $x_{2}$ of $X_{2}$,
  $x_{3}$ of $X_{3}$, $x_{4}$ of $X_{4}$,
  we have $x=(x_{1},x_{2},x_{3},x_{4})$ implies $y_{1}=x_{1}$.
  Then $y_{1}=x_{\mathbf{1},4}$.
\item\label{mcart1:76} Let $y_{2}$ be an object.
  Suppose for any elements $x_{1}$ of $X_{1}$, $x_{2}$ of $X_{2}$,
  $x_{3}$ of $X_{3}$, $x_{4}$ of $X_{4}$,
  we have $x=(x_{1},x_{2},x_{3},x_{4})$ implies $y_{2}=x_{2}$.
  Then $y_{2}=x_{\mathbf{2},4}$.
\item\label{mcart1:77} Let $y_{3}$ be an object.
  Suppose for any elements $x_{1}$ of $X_{1}$, $x_{2}$ of $X_{2}$,
  $x_{3}$ of $X_{3}$, $x_{4}$ of $X_{4}$,
  we have $x=(x_{1},x_{2},x_{3},x_{4})$ implies $y_{3}=x_{3}$.
  Then $y_{3}=x_{\mathbf{3},4}$.
\item\label{mcart1:78} Let $y_{4}$ be an object.
  Suppose for any elements $x_{1}$ of $X_{1}$, $x_{2}$ of $X_{2}$,
  $x_{3}$ of $X_{3}$, $x_{4}$ of $X_{4}$,
  we have $x=(x_{1},x_{2},x_{3},x_{4})$ implies $y_{4}=x_{4}$.
  Then $y_{4}=x_{\mathbf{4},4}$.
\end{thm}

Let $X_{1}$, \dots, $X_{4}$ be arbitrary sets. Then we have the
following results:
\begin{thm}
\item\label{mcart1:79} If $z\in X_{1}\times X_{2}\times X_{3}\times X_{4}$,
  then there exists objects $x_{1}$, $x_{2}$, $x_{3}$, $x_{4}$ such that
  $x_{1}\in X_{1}$, $x_{2}\in X_{2}$, $x_{3}\in X_{3}$, $x_{4}\in X_{4}$,
  and $z=(x_{1},x_{2},x_{3},x_{4})$.
\item\label{mcart1:80} $(x_{1},x_{2},x_{3},x_{4})\in X_{1}\times X_{2}\times X_{3}\times X_{4}$
  if and only if
  $x_{1}\in X_{1}$, $x_{2}\in X_{2}$, $x_{3}\in X_{3}$ and $x_{4}\in X_{4}$.
\item\label{mcart1:81} Suppose every object $z$ satisfies $z\in Z$
  if and only if there exists objects $x_{1}$, $x_{2}$, $x_{3}$, $x_{4}$
  such that
  $x_{1}\in X_{1}$ and $x_{2}\in X_{2}$ and $x_{3}\in X_{3}$ and
  $x_{4}\in X_{4}$ and $z=(x_{1},x_{2},x_{3},x_{4})$.
  Then $Z=X_{1}\times X_{2}\times X_{3}\times X_{4}$.
\item\label{mcart1:82} (Cancelled)
\item\label{mcart1:83} Let $X_{1}$, \dots, $X_{4}$ be nonempty sets.
  Let $A_{1}$ be a nonempty subset of $X_{1}$,
  let $A_{2}$ be a nonempty subset of $X_{2}$,
  let $A_{3}$ be a nonempty subset of $X_{3}$,
  let $A_{4}$ be a nonempty subset of $X_{4}$,
  and let $x$ be an element of $X_{1}\times X_{2}\times X_{3}\times X_{4}$.
  If $x\in A_{1}\times A_{2}\times A_{3}\times A_{4}$,
  then $x_{\mathbf{1},4}\in A_{1}$
  and $x_{\mathbf{2},4}\in A_{2}$
  and $x_{\mathbf{3},4}\in A_{3}$
  and $x_{\mathbf{4},4}\in A_{4}$.
\item\label{mcart1:84} If $X_{1}\subset Y_{1}$,
  $X_{2}\subset Y_{2}$, $X_{3}\subset Y_{3}$, $X_{4}\subset Y_{4}$,
  then $X_{1}\times X_{2}\times X_{3}\times X_{4}\subset Y_{1}\times Y_{2}\times Y_{3}\times Y_{4}$.
\end{thm}

\begin{definition}
Let $X_{1}$, $X_{2}$ be sets, let $A_{1}$ be a subset of $X_{1}$, let
$A_{2}$ be a subset of $X_{2}$. We redefine the type of $A_{1}\times A_{2}$
to be a subset of $X_{1}\times X_{2}$.
\end{definition}

\begin{definition}
Let $X_{1}$, $X_{2}$, $X_{3}$ be sets, let $A_{1}$ be a subset of $X_{1}$, let
$A_{2}$ be a subset of $X_{2}$, let $A_{3}$ be a subset of $X_{3}$.
We redefine the type of $A_{1}\times A_{2}\times A_{3}$
to be a subset of $X_{1}\times X_{2}\times X_{3}$.
\end{definition}

\begin{definition}
Let $X_{1}$, $X_{2}$, $X_{3}$, $X_{4}$ be sets,
let $A_{1}$ be a subset of $X_{1}$, let
$A_{2}$ be a subset of $X_{2}$, let $A_{3}$ be a subset of $X_{3}$,
let $A_{4}$ be a subset of $X_{4}$.
We redefine the type of $A_{1}\times A_{2}\times A_{3}\times A_{4}$
to be a subset of $X_{1}\times X_{2}\times X_{3}\times X_{4}$.
\end{definition}

\begin{definition}
Let $f$ be a function. We define $\pr1(f)$ (Mizar: ``\verb#pr1 f#'')
to be a function defined by:
\begin{defn}
\item $\dom(\pr1(f))=\dom(f)$ and for $x$ be any object such that $x\in\dom(f)$
we have $\pr1(f)(x)=(f(x))_{1}$.
\end{defn}
We define the term $\pr2(f)$ to be the function satisfying
\begin{defn}
\item $\dom(\pr2(f))=\dom(f)$ and for $x$ be any object such that $x\in\dom(f)$
we have $\pr2(f)(x)=(f(x))_{2}$.
\end{defn}
\end{definition}

\begin{definition}
Let $x$ be an object.
We define the term $x_{11}$ (Mizar: ``\verb#x`11#'') to be the set equal
to
\begin{defn}
\item $(x_{1})_{1}$
\end{defn}
We define the term $x_{12}$ (Mizar: ``\verb#x`12#'') to be the set equal
to
\begin{defn}
\item $(x_{1})_{2}$
\end{defn}
We define the term $x_{21}$ (Mizar: ``\verb#x`21#'') to be the set equal
to
\begin{defn}
\item $(x_{2})_{1}$
\end{defn}
We define the term $x_{22}$ (Mizar: ``\verb#x`22#'') to be the set equal
to
\begin{defn}
\item $(x_{2})_{2}$
\end{defn}
\end{definition}

\begin{thm}
\item\label{mcart1:85} The following results are all provable:
  \begin{enumerate}[label=(\roman*)]
  \item $((x_{1},x_{2}),y)_{11}=x_{1}$
  \item $((x_{1},x_{2}),y)_{12}=x_{2}$
  \item $(x,(y_{1},y_{2}))_{21}=y_{1}$
  \item $(x,(y_{1},y_{2}))_{22}=y_{2}$
  \end{enumerate}
\item\label{mcart1:86} If $x\in R$, then $x_{1}\in\dom(R)$ and $x_{2}\in\rng(R)$. 
\item\label{mcart1:87} Let $R$ be a nonempty relation, let $x$ be an object.
  Then $\RelIm{R}{x}=\{I_{2}\mbox{ where $I$ is Element of }R\mid I_{1}=x\}$.
\item\label{mcart1:88} If $x\in R$,
  then $x_{2}\in\RelIm{R}{x_{1}}$.
\item\label{mcart1:89} If $x\in R$, $y\in R$, $x_{1}=y_{1}$, and
  $x_{2}=y_{2}$,
  then $x=y$.
\item\label{mcart1:90} Let $R$ be a nonempty relation, let $x$ and $y$
  be elements of $R$. If $x_{1}=y_{1}$ and $x_{2}=y_{2}$, then $x=y$.
\item\label{mcart1:91} $\proj1\bigl(\proj1\{(x_{1},x_{2},x_{3}),(y_{1},y_{2},y_{3})\}\bigr)=\{x_{1},y_{1}\}$.
\item\label{mcart1:92} $\proj1\bigl(\proj1\{(x_{1},x_{2},x_{3})\}\bigr)=\{x_{1},\}$.
\end{thm}

\begin{scheme}[BiFuncEx]
Let $\mathcal{A}$, $\mathcal{B}$, and $\mathcal{C}$ be sets, let $P[-,-,-]$
be a ternary predicate of objects.
There exists functions $f$ and $g$ such that
$\dom(f)=\mathcal{A}$ and $\dom(g)=\mathcal{A}$ and for all objects
$x\in\mathcal{A}$ we have $P[x,f(x),g(x)]$;
provided
\begin{enumerate}
\item If $x\in\mathcal{A}$,
  then there exists objects $y\in\mathcal{B}$ and $z\in\mathcal{C}$ such
  that $P[x,y,z]$.
\end{enumerate}
\end{scheme}

We have the final result:
\begin{thm}
\item\label{mcart1:93} If $((x_{1},x_{2}),(x_{3},x_{4}))=((y_{1},y_{2}),(y_{3},y_{4}))$,
then $x_{1}=y_{1}$ and $x_{2}=y_{2}$ and $x_{3}=y_{3}$ and $x_{4}=y_{4}$.
\end{thm}

\end{document}