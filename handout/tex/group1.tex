\section{Groups (GROUP-1)}

\begin{definition}
Let $X$ be a magma. We define the attribute $X$ is \define{unital} to
mean
\begin{defn}
\item There exists $x$ being an element of $X$ such that for any element
  $h$ of $X$ we have $he=h$ and $eh=h$.
\end{defn}
We call $X$ \define{Group-like} to mean
\begin{defn}
\item There exists an element $e$ of $X$ such that for every element $h$
  of $X$ we have $he=h$ and $eh=h$ and there exists an element $g$ of
  $X$ such that $hg=e$ and $gh=e$.
\end{defn}
We call $X$ \define{associative} to mean
\begin{defn}
\item for any elements $x$, $y$, $z$ of $X$, we have $(xy)z=x(yz)$.
\end{defn}
\end{definition}

Observe that every Group-like magma is unital. We observe there exists a
non-empty  strict Group-like associative magma.

\begin{definition}
We define a \define{Group} to be a non-empty Group-like associative magma.
\end{definition}