\documentclass{article}

\title{Three-Argument Operations and Four-Argument Operations (MULTOP-1)}
\author{Michal Muzalewski and Wojciech Skaba}
\date{October 2, 1990}
\begin{document}
\maketitle

\begin{definition}
Let $f$ be a function, let $a$, $b$, $c$ be objects.
We define the term $f(a,b,c)$ to be the set equal to
\begin{defn}
\item $f(a,b,c) = f((a,b,c))$.
\end{defn}
\end{definition}

\begin{definition}
Let $A$, $B$, $C$, $D$ be nonempty sets, let $f\colon A\times B\times C\to D$,
let $a$ be an element of $A$, $b$ be an element of $B$, $c$ be an
element of $C$.
We redefine the type of the term $f(a,b,c)$ to be an element of $D$.
\end{definition}

Let $A$, $B$, $C$, $D$ be nonempty sets, let $f\colon A\times B\times C\to D$,
let $a$ be an element of $A$, $b$ be an element of $B$, $c$ be an
element of $C$. Let $X$, $Y$, $Z$ be sets.
\begin{thm}
\item\label{multop1:1} Let $f_{1},f_{2}\colon X\times Y\times Z\to D$.
  If every $x\in X$, $y\in Y$, and $z\in Z$ satisfies $f_{1}(x,y,z)=f_{2}(x,y,z)$,
  then $f_{1}=f_{2}$.
\item\label{multop1:2} Let $f_{1},f_{2}\colon A\times B\times C\to D$.
  If every $a\in A$, $b\in B$, and $c\in C$ satisfies $f_{1}(a,b,c)=f_{2}(a,b,c)$,
  then $f_{1}=f_{2}$.
\item\label{multop1:3} Let $f_{1},f_{2}\colon A\times B\times C\to D$.
  If every elements $a$ of $A$, $b$ of $B$, and $c$ of $C$ satisfies $f_{1}(a,b,c)=f_{2}(a,b,c)$,
  then $f_{1}=f_{2}$.
\end{thm}

\begin{definition}
Let $A$ be a set. We define the mode, a \define{Ternary Operator of $A$}
(Mizar: ``\verb#TriOp of A#'') to be a Function from $A\times A\times A$
to $A$.
\end{definition}

\begin{scheme}[FuncEx3D]
Let $\mathcal{X}$, $\mathcal{Y}$, $\mathcal{Z}$, $\mathcal{T}$ be
nonempty sets, let $\mathcal{P}[-,-,-,-]$ be a tetradic predicate
of objects.
There exists $f\colon\mathcal{X}\times\mathcal{Y}\times\mathcal{Z}\to\mathcal{T}$
such that every element $x$ of $\mathcal{X}$, $y$ of $\mathcal{Y}$, $z$
of $\mathcal{Z}$ satisfies $\mathcal{P}[x,y,z,f(x,y,z)]$; provided:
\begin{enumerate}
\item For any elements $x$ of $\mathcal{X}$, $y$ of $\mathcal{Y}$, $z$
of $\mathcal{Z}$, there exists an element $t$ of $\mathcal{T}$ such that $\mathcal{P}[x,y,z,t]$.
\end{enumerate}
\end{scheme}

\begin{scheme}[TriOpEx]
Let $\mathcal{A}$ be a nonempty set, let $\mathcal{P}[-,-,-,-]$ be a
tetradic predicate of elements of $\mathcal{A}$.
There exists a ternary operator $p$ of $\mathcal{A}$ such that for all
elements $a$, $b$, $c$ of $\mathcal{A}$ we have $\mathcal{P}[a,b,c,p(a,b,c)]$;
provided:
\begin{enumerate}
\item for all elements $x$, $y$, $z$ of $\mathcal{A}$, there exists an
  element $t$ of $\mathcal{A}$ such that $\mathcal{P}[x,y,z,t]$.
\end{enumerate}
\end{scheme}

\begin{scheme}[Lambda3D]
Let $\mathcal{X}$, $\mathcal{Y}$, $\mathcal{Z}$, $\mathcal{T}$ be
nonempty sets, let $\mathcal{F}(-,-,-,)$ be an element of $\mathcal{T}$
parametrized by an element of $\mathcal{X}$, $\mathcal{Y}$ and $\mathcal{Z}$.
There exists a function $f\colon\mathcal{X}\times\mathcal{Y}\times\mathcal{Z}\to\mathcal{T}$
such that for each element $x$ of $\mathcal{X}$, $y$ of $\mathcal{Y}$,
$z$ of $\mathcal{Z}$, we have $f(x,y,z)=\mathcal{F}(x,y,z)$.
\end{scheme}

\begin{scheme}[TriOpLambda]
Let $\mathcal{A}$, $\mathcal{B}$, $\mathcal{C}$, $\mathcal{D}$ be
nonempty sets, let $\mathcal{F}(-,-,-)$ be an element of $\mathcal{D}$
parametrized by an element of $\mathcal{A}$, $\mathcal{B}$, and $\mathcal{C}$.
There existsa function $f\colon\mathcal{A}\times\mathcal{B}\times\mathcal{C}\to\mathcal{D}$
such that for each element $a$ of $\mathcal{A}$, $b$ of $\mathcal{B}$,
$c$ of $\mathcal{C}$, we have $f(a,b,c)=\mathcal{F}(a,b,c)$.
\end{scheme}

\begin{definition}
Let $f$ be a function.
Let $a$, $b$, $c$, $d$ be sets.
We define the term $f(a,b,c,d)$ (Mizar: ``\verb#f.(a,b,c,d)#'') to be
the set equal to
\begin{defn}
\item $f(a,b,c,d):=f((a,b,c,d))$.
\end{defn}
\end{definition}

\begin{definition}
Let $A$, $B$, $C$, $D$, $E$ be nonempty sets, let $f\colon A\times B\times C\times D\to E$,
let $a$ be an element of $A$, $b$ an element of $B$, $c$ of $C$, $d$ of $D$.
We redefine the type of $f(a,b,c,d)$ to be an element of $E$.
\end{definition}

Let $X$, $Y$, $Z$, $S$ be arbitrary sets, let $A$, $B$, $C$, $D$, $E$ be
nonempty sets.
We have the following three results:
\begin{thm}
\item\label{multop1:4} Let $f_{1},f_{2}\colon X\times Y\times Z\times S\to D$
  be functions. Suppose for all sets $x$, $y$, $z$, $s$ with $x\in X$,
  $y\in Y$, $z\in Z$, $s\in S$ we have $f_{1}(x,y,z,s)=f_{2}(x,y,z,s)$.
  Then $f_{1}=f_{2}$.
\item\label{multop1:5} Let $f_{1},f_{2}\colon A\times B\times C\times D\to E$.
  Suppose for all elements $a$ of $A$, $b$ of $B$, $c$ of $C$, $d$ of
  $D$, we have $f_{1}((a,b,c,d))=f_{2}((a,b,c,d))$.
  Then $f_{1}=f_{2}$.
\item\label{multop1:6} Let $f_{1},f_{2}\colon A\times B\times C\times D\to E$.
  Suppose for all elements $a$ of $A$, $b$ of $B$, $c$ of $C$, $d$ of
  $D$, we have $f_{1}(a,b,c,d)=f_{2}(a,b,c,d)$.
  Then $f_{1}=f_{2}$.
\end{thm}

\begin{definition}
Let $A$ be a nonempty set.
We define a new mode, a \define{Quaternary Operator of $A$} is a
function from $A\times A\times A\times A$ to $A$.
\end{definition}

\begin{scheme}[FuncEx4D]
Let $\mathcal{X}$, $\mathcal{Y}$, $\mathcal{Z}$, $\mathcal{S}$, $\mathcal{T}$ be
nonempty sets, let $\mathcal{P}[-,-,-,-,-]$ be a pentadic predicate of
objects.
There exists a function $f\colon\mathcal{X}\times\mathcal{Y}\times\mathcal{Z}\times\mathcal{S}\to\mathcal{T}$
such that for all elements $x$ of $\mathcal{X}$, $y$ of $\mathcal{Y}$,
$z$ of $\mathcal{Z}$, $s$ of $\mathcal{S}$, we have $\mathcal{P}[x,y,z,s,f(x,y,z,s)]$;
provided
\begin{enumerate}
\item for all elements $x$ of $\mathcal{X}$, $y$ of $\mathcal{Y}$,
$z$ of $\mathcal{Z}$, $s$ of $\mathcal{S}$,
there exists an element $t$ of $\mathcal{T}$ such that $\mathcal{P}[x,y,z,s,t]$.
\end{enumerate}
\end{scheme}

\begin{scheme}[QuaOpEx]
Let $\mathcal{A}$ be a nonempty set, let $\mathcal{P}[-,-,-,-,-]$ be a
pentadic predicate of elements of $\mathcal{A}$.
There exists a quaternary operator $f$ of $\mathcal{A}$ such that for
all elements $a_{1}$, $a_{2}$, $a_{3}$, $a_{4}$ of $\mathcal{A}$ we have
$\mathcal{P}[a_{1},a_{2},a_{3},a_{4},f(a_{1},a_{2},a_{3},a_{4})]$;
provided:
\begin{enumerate}
\item for all elements $x$, $y$, $z$, $s$ of $\mathcal{A}$, there exists
  an element $t$ of $\mathcal{A}$ such that $\mathcal{P}[x,y,z,s,t]$.
\end{enumerate}
\end{scheme}

\begin{scheme}[Lambda4D]
Let $\mathcal{X}$, $\mathcal{Y}$, $\mathcal{Z}$, $\mathcal{S}$, $\mathcal{T}$
be nonempty sets, let $\mathcal{F}(-,-,-,-)$ be an element of $\mathcal{T}$
parametrized by an element of $\mathcal{X}$, $\mathcal{Y}$,
$\mathcal{Z}$, and $\mathcal{T}$ (respectively).
There exists a function $f\colon\mathcal{X}\times\mathcal{Y}\times\mathcal{Z}\times\mathcal{S}\to\mathcal{T}$
such that for all elements $x$ of $\mathcal{X}$, $y$ of $\mathcal{Y}$,
$z$ of $\mathcal{Z}$, $s$ of $\mathcal{S}$, we have $f(x,y,z,s)=\mathcal{F}(x,y,z,s)$.
\end{scheme}

\begin{scheme}[QuaOpLambda]
Let $\mathcal{A}$ be a nonempty set, let $\mathcal{F}(-,-,-,-)$ be an
element of $\mathcal{A}$ parametrized by four elements of $\mathcal{A}$.
There exists a quaternary operator $f$ of $\mathcal{A}$ such that for
all elements $a$, $b$, $c$, $d$ of $\mathcal{A}$ we have $f(a,b,c,d)=\mathcal{F}(a,b,c,d)$.
\end{scheme}





\end{document}