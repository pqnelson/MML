\documentclass{article}
\title{Complex Numbers --- Basic Theorems (XCMPLX-1)}
\author{Library Committee}
\date{April 10, 2003}
\begin{document}
\maketitle

Let $a$, $b$, $c$, $d$ be Complex. Then we have the following results:
\begin{thm}
\item\label{xcmplx1:1} $a+(b+c)=(a+b)+c$.
\item\label{xcmplx1:2} If $a+c=b+c$, then $a=b$.
\item\label{xcmplx1:3} If $a=a+b$, then $b=0$.
\item\label{xcmplx1:4} $a\cdot(b\cdot c)=(a\cdot b)\cdot c$.
  \bigbreak
\item\label{xcmplx1:5} If $c\neq0$ and $a\cdot c=b\cdot c$, then $a=b$.
\item\label{xcmplx1:6} If $a\cdot b=0$, then either $a=0$ or $b=0$.
\item\label{xcmplx1:7} If $b\neq0$ and $a\cdot b=b$, then $a=1$.
  \bigbreak
\item\label{xcmplx1:8} $a\cdot(b+c)=a\cdot b+a\cdot c$.
\item\label{xcmplx1:9} $(a+b+c)\cdot d=a\cdot d+b\cdot d+c\cdot d$.
\item\label{xcmplx1:10} $(a+b)\cdot(c+d)=a\cdot c+a\cdot d+b\cdot c+b\cdot d$.
\item\label{xcmplx1:11} $2\cdot a=a+a$
\item\label{xcmplx1:12} $3\cdot a=a+a+a$
\item\label{xcmplx1:13} $4\cdot a=a+a+a+a$
  \bigbreak
\item\label{xcmplx1:14} $a-a=0$
\item\label{xcmplx1:15} If $a-b=0$, then $a=b$.
\item\label{xcmplx1:16} If $b-a=b$, then $a=0$.
  \bigbreak
\item\label{xcmplx1:17} $a=a-(b-b)$.
\item\label{xcmplx1:18} $a-(a-b)=b$.
\item\label{xcmplx1:19} If $a-c=b-c$, then $a=b$.
\item\label{xcmplx1:20} If $c-a=c-b$, then $a=b$.
\item\label{xcmplx1:21} $a-b-c=a-c-b$.
\item\label{xcmplx1:22} $a-c=(a-b)-(c-b)$
\item\label{xcmplx1:23} $(c-a)-(c-b)=b-a$.
\item\label{xcmplx1:24} If $a-b=c-d$, then $a-c=b-d$.
  \bigbreak
\item\label{xcmplx1:25} $a=a+(b-b)$
\item\label{xcmplx1:26} $a=(a+b)-b$
\item\label{xcmplx1:27} $a=(a-b)+b$
\item\label{xcmplx1:28} $a+c=(a+b)+(c-b)$
  \bigbreak
\item\label{xcmplx1:29} $a+b-c=a-c+b$
\item\label{xcmplx1:30} $a-b+c=c-b+a$
\item\label{xcmplx1:31} $a+c=a+b-(b-c)$
\item\label{xcmplx1:32} $a-c=a+b-(c+b)$
\item\label{xcmplx1:33} If $a+b=c+d$, then $a-c=d-b$.
\item\label{xcmplx1:34} If $a-c=d-b$, then $a+b=c+d$.
\item\label{xcmplx1:35} If $a+b=c-d$, then $a+d=c-b$.
  \bigbreak
\item\label{xcmplx1:36} $a-(b+c)=a - b - c$.
\item\label{xcmplx1:37} $a - (b - c)=a - b + c$
\item\label{xcmplx1:38} $a - (b - c) = a + (c - b)$
\item\label{xcmplx1:39} $a - c = (a - b) + (b - c)$
\item\label{xcmplx1:40} $a\cdot(b - c)=a\cdot b-a\cdot c$.
\item\label{xcmplx1:41} $(a-b)\cdot(c-d)=(b-a)\cdot(d-c)$
\item\label{xcmplx1:42} $(a-b-c)\cdot d=a\cdot d-b\cdot d-c\cdot d$.
  \bigbreak
\item\label{xcmplx1:43} $(a+b-c)\cdot d=a\cdot d+b\cdot d-c\cdot d$
\item\label{xcmplx1:44} $(a+b-c)\cdot d=a\cdot d+b\cdot d+c\cdot d$
\item\label{xcmplx1:45} $(a+b)\cdot(c-d)=a\cdot c-a\cdot d+b\cdot c-b\cdot d$
\item\label{xcmplx1:46} $(a-b)\cdot(c+d)=a\cdot c+a\cdot d-b\cdot c-b\cdot d$
\item\label{xcmplx1:47} $(a-b)\cdot(c-d)=a\cdot c-a\cdot d-b\cdot c+b\cdot d$
  \bigbreak
\item\label{xcmplx1:48} $(a/b)/c=(a/c)/b$.
  \bigbreak
\item\label{xcmplx1:49} $a/0=0$.
\item\label{xcmplx1:50} If $a\neq0$ and $b\neq0$, then $a/b\neq0$.
  \bigbreak
\item\label{xcmplx1:51} If $b\neq 0$, then $a=a/(b/b)$.
\item\label{xcmplx1:52} If $a\neq0$, then $a/(a/b)=b$.
\item\label{xcmplx1:53} If $c\neq0$ and $a/c=b/c$, then $a=b$.
\item\label{xcmplx1:54} If $a/b\neq0$, then $b=a/(a/b)$.
\item\label{xcmplx1:55} If $c\neq0$, then $a/b=(a/c)/(b/c)$.
  \bigbreak
\item\label{xcmplx1:56} $1/(1/a)=a$.
\item\label{xcmplx1:57} $1/(a/b)=b/a$.
\item\label{xcmplx1:58} If $a/b=1$, then $a=b$.
\item\label{xcmplx1:59} If $1/a=1/b$, then $a=b$.
  \bigbreak
\item\label{xcmplx1:60} If $a\neq0$, then $a/a=1$
\item\label{xcmplx1:61} If $b\neq0$ and $b/a=b$, then $a=1$.
  \bigbreak
\item\label{xcmplx1:62} $(a/c)+(b/c)=(a+b)/c$
\item\label{xcmplx1:63} $(a+b+c)/d=(a/d)+(b/d)+(c/d)$.
  \bigbreak
\item\label{xcmplx1:64} $(a+a)/2=a$.
\item\label{xcmplx1:65} $(a/2)+(a/2)=a$.
\item\label{xcmplx1:66} If $a=(a+b)/2$, then $a=b$.
  \bigbreak
\item\label{xcmplx1:67} $(a+a+a)/3=a$.
\item\label{xcmplx1:68} $(a/3)+(a/3)+(a/3)=a$.
  \bigbreak
\item\label{xcmplx1:69} $(a+a+a+a)/4=a$.
\item\label{xcmplx1:70} $(a/4)+(a/4)+(a/4)+(a/4)=a$.
\item\label{xcmplx1:71} $(a/4)+(a/4)=a/2$
\item\label{xcmplx1:72} $(a+a)/4=a/2$.
  \bigbreak
\item\label{xcmplx1:73} If $a\cdot b=1$, then $a=1/b$
\item\label{xcmplx1:74} $a\cdot(b/c)=(a\cdot b)/c$
\item\label{xcmplx1:75} $(a/b)\cdot c=(c/b)\cdot a$
  \bigbreak
\item\label{xcmplx1:76} $(a/b)\cdot(c/d)=(a\cdot c)/(b\cdot d)$
\item\label{xcmplx1:77} $a/(b/c)=(a\cdot c)/b$
\item\label{xcmplx1:78} $a/(b\cdot c)=(a/b)/c$
\item\label{xcmplx1:79} $a/(b/c)=a\cdot(c/b)$
\item\label{xcmplx1:80} $a/(b/c)=(c/b)\cdot a$
\item\label{xcmplx1:81} $a/(b/c)=c\cdot(a/b)$
\item\label{xcmplx1:82} $a/(b/c)=(a/b)\cdot c$
\item\label{xcmplx1:83} $(a\cdot b)/(c\cdot d)=((a/c)\cdot b)/d$
  \bigbreak
\item\label{xcmplx1:84} $(a/b)/(c/d)=(a\cdot d)/(b\cdot c)$
\item\label{xcmplx1:85} $(a/c)\cdot(b/d)=(a/d)\cdot(b/c)$.
\item\label{xcmplx1:86} $a/(b\cdot c\cdot (d/e))=(e/c)\cdot(a/(b\cdot d))$
  \bigbreak
\item\label{xcmplx1:87} If $b\neq0$, then $(a/b)\cdot b=a$
\item\label{xcmplx1:88} If $b\neq0$, then $a=a\cdot(b/b)$
\item\label{xcmplx1:89} If $b\neq0$, then $a=(a\cdot b)/b$
\item\label{xcmplx1:90} If $b\neq 0$, then $a\cdot c=a\cdot b\cdot(c/b)$.
  \bigbreak
\item\label{xcmplx1:91} If $c\neq0$, then $a/b=(a\cdot c)/(b\cdot c)$
\item\label{xcmplx1:92} If $c\neq0$, then $a/b=(a/(b\cdot c))\cdot c$.
\item\label{xcmplx1:93} If $b\neq0$, then $a\cdot c=(a\cdot b)/(b/c)$.
\item\label{xcmplx1:94} If $c\neq 0$, $d\neq 0$, and $a\cdot c=b\cdot d$,
  then $a/d=b/c$.
\item\label{xcmplx1:95} If $c\neq0$, $d\neq0$, and $a/d=b/c$, then
  $a\cdot c=b\cdot d$.
\item\label{xcmplx1:96} If $c\neq0$, $d\neq0$, and $a\cdot c=b/d$,
  then $a\cdot d=b/c$.
  \bigbreak
\item\label{xcmplx1:97} If $c\neq0$, then $a/b=c\cdot((a/c)/b)$.
\item\label{xcmplx1:98} If $c\neq0$, then $a/b=(a/c)\cdot(c/b)$
  \bigbreak
\item\label{xcmplx1:99} $a\cdot(1/b)=a/b$
\item\label{xcmplx1:100} $a/(1/b)=a\cdot b$
\item\label{xcmplx1:101} $(a/b)\cdot c=((1/b)\cdot c)\cdot a$
  \bigbreak
\item\label{xcmplx1:102} $(1/a)\cdot(1/b)=1/(a\cdot b)$
\item\label{xcmplx1:103} $(1/c)\cdot(a/b)=a/(b\cdot c)$
  \bigbreak
\item\label{xcmplx1:104} $(a/b)/c=(1/b)\cdot(a/c)$
\item\label{xcmplx1:105} $(a/b)/c=(1/c)\cdot(a/b)$
  \bigbreak
\item\label{xcmplx1:106} If $a\neq0$, then $a\cdot(1/a)=1$
\item\label{xcmplx1:107} If $b\neq0$, then $a=(a\cdot b)\cdot(1/b)$
\item\label{xcmplx1:108} If $b\neq0$, then $a=a\cdot((1/b)\cdot b)$
\item\label{xcmplx1:109} If $b\neq0$, then $a=(a\cdot(1/b))\cdot b$
\item\label{xcmplx1:110} If $b\neq0$, then $a=a/(b\cdot(1/b))$
\item\label{xcmplx1:111} If $a\neq0$ and $b\neq0$, then $1/(a\cdot b)\neq0$
\item\label{xcmplx1:112} If $a\neq0$ and $b\neq0$, then $(a/b)\cdot(b/a)=1$
  \bigbreak
\item\label{xcmplx1:113} If $b\neq0$, then $(a/b)+c=(a+(b\cdot c))/b$
\item\label{xcmplx1:114} If $c\neq0$, then $a+b=c\cdot((a/c)+(b/c))$
\item\label{xcmplx1:115} If $c\neq0$, then $a+b=((a\cdot c)+(b\cdot c))/c$
\item\label{xcmplx1:116} If $b\neq0$ and $d\neq0$,
  then $(a/b)+(c/d)=((a\cdot d)+(c\cdot b))/(b\cdot d)$
\item\label{xcmplx1:117} If $a\neq0$, then $a+b=a\cdot(1+(b/a))$.
  \bigbreak
\item\label{xcmplx1:118} $(a/(2\cdot b))+(a/(2\cdot b))=a/b$
  \bigbreak
\item\label{xcmplx1:119} $(a/(3\cdot b))+(a/(3\cdot b))+(a/(3\cdot b))=a/b$
\item\label{xcmplx1:120} $(a/c)-(b/c)=(a-b)/c$
\item\label{xcmplx1:121} $a-(a/2)=a/2$
\item\label{xcmplx1:122} $((a-b)-c)/d=(a/d)-(b/d)-(c/d)$
\item\label{xcmplx1:123} If $b\neq0$, $d\neq0$, $b\neq d$, and $a/b=c/d$,
  then $a/b=(a-c)/(b-d)$.
  \bigbreak
\item\label{xcmplx1:124} $(a+b-c)/d=(a/d)+(b/d)-(c/d)$
\item\label{xcmplx1:125} $(a-b+c)/d=(a/d)-(b/d)+(c/d)$
  \bigbreak
\item\label{xcmplx1:126} If $b\neq0$, then $(a/b)-c=(a-(c\cdot b))/b$
\item\label{xcmplx1:127} If $b\neq0$, then $c-(a/b)=((c\cdot b)-a)/b$
\item\label{xcmplx1:128} If $c\neq0$, then $a-b=c\cdot((a/c)-(b/c))$.
\item\label{xcmplx1:129} If $c\neq0$, then $a-b=((a\cdot c)-(b\cdot c))/c$
\item\label{xcmplx1:130} If $b\neq0$ and $d\neq0$, then
  $(a/b)-(c/d)=(a\cdot d-c\cdot b)/(b\cdot d)$
\item\label{xcmplx1:131} If $a\neq0$, then $a-b=a\cdot(1-(b/a))$.
  \bigbreak
\item\label{xcmplx1:132} If $a\neq0$, then
  $c=((a\cdot c) + b - b)/a$
  \bigbreak
\item\label{xcmplx1:133} If $-a=-b$, then $a=b$.
\item\label{xcmplx1:134} If $-a=0$, then $a=0$.
\item\label{xcmplx1:135} If $a+(-b)=0$, then $a=b$
\item\label{xcmplx1:136} $a=(a+b)+(-b)$
\item\label{xcmplx1:137} $a=a+(b + (-b))$
\item\label{xcmplx1:138} $a=((-b)+a)+b$
\item\label{xcmplx1:139} $-(a+b)=(-a)+(-b)$
\item\label{xcmplx1:140} $-((-a)+b)=a+(-b)$
\item\label{xcmplx1:141} $a+b=-((-a)+(-b))$
  \bigbreak
\item\label{xcmplx1:142} $-(a-b)=b-a$
\item\label{xcmplx1:143} $(-a)-b=(-b)-a$
\item\label{xcmplx1:144} $a=(-b)-((-a)-b)$
  \bigbreak
\item\label{xcmplx1:145} $((-a)-b)-c=((-a)-c)-b$
\item\label{xcmplx1:146} $((-a)-b)-c=((-b)-c)-a$
\item\label{xcmplx1:147} $((-a)-b)-c=((-c)-b)-a$
\item\label{xcmplx1:148} $(c-a)-(c-b)=-(a-b)$
  \bigbreak
\item\label{xcmplx1:149} $0-a=-a$
  \bigbreak
\item\label{xcmplx1:150} $a+b=a-(-b)$
\item\label{xcmplx1:151} $a=a-(b+(-b))$
\item\label{xcmplx1:152} If $a-c=b+(-c)$, then $a=b$
\item\label{xcmplx1:153} If $c-a=c+(-b)$, then $a=b$
  \bigbreak
\item\label{xcmplx1:154} $(a+b)-c=((-c)+a)+b$
\item\label{xcmplx1:155} $(a-b)+c=((-b)+c)+a$
\item\label{xcmplx1:156} $a-((-b)-c)=(a+b)+c$
  \bigbreak
\item\label{xcmplx1:157} $(a-b)-c=((-b)-c)+a$
\item\label{xcmplx1:158} $(a-b)-c=((-c)+a)-b$
\item\label{xcmplx1:159} $(a-b)-c=((-c)-b)+a$
  \bigbreak
\item\label{xcmplx1:160} $-(a+b)=(-b)-a$
\item\label{xcmplx1:161} $-(a-b)=(-a)+b$
\item\label{xcmplx1:162} $-((-a)+b)=a-b$
\item\label{xcmplx1:163} $a+b=-((-a)-b)$
\item\label{xcmplx1:164} $((-a)+b)-c=((-c)+b)-a$
  \bigbreak
\item\label{xcmplx1:165} $((-a)+b)-c = ((-c)-a)+b$
\item\label{xcmplx1:166} $-(a+b+c)=((-a)-b)-c$
\item\label{xcmplx1:167} $-((a+b)-c)=((-a)-b)+c$
\item\label{xcmplx1:168} $-((a-b)+c)=((-a)+b)-c$
\item\label{xcmplx1:169} $-((a-b)-c)=((-a)+b)+c$
\item\label{xcmplx1:170} $-(-a+b+c)=a-b-c$
\item\label{xcmplx1:171} $-(-a+b-c)=a-b+c$
\item\label{xcmplx1:172} $-(-a-b+c)=a+b-c$
\item\label{xcmplx1:173} $-(-a-b-c)=a+b+c$
  \bigbreak
\item\label{xcmplx1:174} $(-a)\cdot b=-(a\cdot b$
\item\label{xcmplx1:175} $(-a)\cdot b=a\cdot(-b)$
\item\label{xcmplx1:176} $(-a)\cdot(-b)=a\cdot b$
\item\label{xcmplx1:177} $-(a\cdot(-b))=a\cdot b$
\item\label{xcmplx1:178} $-((-a)\cdot b)=a\cdot b$
\item\label{xcmplx1:179} $(-1)\cdot a=-a$
\item\label{xcmplx1:180} $(-a)\cdot(-1)=a$
\item\label{xcmplx1:181} If $b\neq0$ and $a\cdot b=-b$, then $a=-1$.
  \bigbreak
\item\label{xcmplx1:182} Either $a\cdot a\neq1$ or $a=1$ or $a=-1$
\item\label{xcmplx1:183} $-a+2\cdot a=a$
\item\label{xcmplx1:184} $(a-b)\cdot c=(b-a)\cdot(-c)$
\item\label{xcmplx1:185} $(a-b)\cdot c=-((b-a)\cdot c)$
\item\label{xcmplx1:186} $a-(2\cdot a)=-a$
  \bigbreak
\item\label{xcmplx1:187} $-(a/b)=(-a)/b$
\item\label{xcmplx1:188} $a/(-b)=-(a/b)$
\item\label{xcmplx1:189} $-(a/(-b))=a/b$
\item\label{xcmplx1:190} $-((-a)/b)=a/b$
\item\label{xcmplx1:191} $(-a)/(-b)=a/b$
\item\label{xcmplx1:192} $(-a)/b=a/(-b)$
\item\label{xcmplx1:193} $-a=a/(-1)$
\item\label{xcmplx1:194} $a=(-a)/(-1)$
\item\label{xcmplx1:195} If $a/b=-1$, then $a=-b$ and $b=-a$.
\item\label{xcmplx1:196} If $b\neq0$ and $b/a=-b$, then $a=-1$
\item\label{xcmplx1:197} If $a\neq0$, then $(-a)/a=-1$
\item\label{xcmplx1:198} If $a\neq0$, then $a/(-a)=-1$
\item\label{xcmplx1:199} If $a\neq0$, $a=1/a$, and $a\neq1$, then $a=-1$
\item\label{xcmplx1:200} If $b\neq0$, $d\neq0$, $b\neq-d$, and
  $a/b=c/d$, then $a/b=(a+c)/(b+d)$.
  \bigbreak
\item\label{xcmplx1:201} If $a^{-1}=b^{-1}$, then $a=b$.
\item\label{xcmplx1:202} $0^{-1}=0$
  \bigbreak
\item\label{xcmplx1:203} If $b\neq0$, then $a=(a\cdot b)\cdot b^{-1}$
\item\label{xcmplx1:204} $a^{-1}\cdot b^{-1}=(a\cdot b)^{-1}$
\item\label{xcmplx1:205} $(a\cdot b^{-1})^{-1}=a^{-1}\cdot b$
\item\label{xcmplx1:206} $(a^{-1}\cdot b^{-1})^{-1}=a\cdot b$
\item\label{xcmplx1:207} If $a\neq0$ and $b\neq0$, then $a\cdot b^{-1}\neq0$
\item\label{xcmplx1:208} If $a\neq0$ and $b\neq0$, then $a^{-1}\cdot b^{-1}\neq0$
\item\label{xcmplx1:209} If $a\cdot b^{-1}=1$, then $a=b$
\item\label{xcmplx1:210} If $a\cdot b=1$, then $a=b^{-1}$
  \bigbreak
\item\label{xcmplx1:211} If $a\neq0$ and $b\neq0$, then
  $a^{-1}+b^{-1}=(a+b)\cdot(a\cdot b)^{-1}$.
  \bigbreak
\item\label{xcmplx1:212} If $a\neq0$ and $b\neq0$, then
  $a^{-1}-b^{-1}=(b-a)\cdot(a\cdot b)^{-1}$.
  \bigbreak
\item\label{xcmplx1:213} $(a/b)^{-1}=b/a$
\item\label{xcmplx1:214} $a^{-1}/b^{-1}=b/a$
\item\label{xcmplx1:215} $1/a=a^{-1}$
\item\label{xcmplx1:216} $1/a^{-1}=a$
\item\label{xcmplx1:217} $(1/a)^{-1}=a$
\item\label{xcmplx1:218} If $1/a=b^{-1}$, then $a=b$.
  \bigbreak
\item\label{xcmplx1:219} $a/b^{-1}=a\cdot b$
\item\label{xcmplx1:220} $a^{-1}\cdot(c/b)=c/(a\cdot b)$
\item\label{xcmplx1:221} $a^{-1}/b=(a\cdot b)^{-1}$
  \bigbreak
\item\label{xcmplx1:222} $(-a)^{-1}=-(a^{-1})$
\item\label{xcmplx1:223} If $a\neq0$, $a=a^{-1}$, and $a\neq1$, then $a=-1$.
\item\label{xcmplx1:224} $a+b+c-b=a+c$
\item\label{xcmplx1:225} $a-b+c+b=a+c$
\item\label{xcmplx1:226} $a+b-c-b=a-c$
\item\label{xcmplx1:227} $a-b-c+b=a-c$
\item\label{xcmplx1:228} $a-a-b=-b$
\item\label{xcmplx1:229} $-a+a-b=-b$
\item\label{xcmplx1:230} $a-b-a=-b$
\item\label{xcmplx1:231} $-a-b+a=-b$
  \bigbreak
\item\label{xcmplx1:232} If $b\neq0$, there exists a Complex $c$ such
  that $a=b/c$.
\item\label{xcmplx1:233} $a/(b\cdot c\cdot(d/e))=(e/c)\cdot(a/(b\cdot d))$
\item\label{xcmplx1:234} $((d-c)/b)\cdot a + c=(1-(a/b))\cdot c + (a/b)\cdot d$
\item\label{xcmplx1:235} If $a\neq0$, then $a\cdot b+c=a\cdot(b+(c/a))$.
\end{thm}

\end{document}