\documentclass{article}
\title{Preliminaries to Structures (STRUCT-0)}
\author{Library Committee}
\date{January 6, 1995}
\begin{document}
\maketitle

\begin{definition}\index{Carrier}
  We define the \define{1-sorted} structure as
  \[\langle\mbox{a carrier}\rangle\]
  where the carrier is a set.
\end{definition}

\begin{definition}
Let $S$ be a 1-sorted structure.
We define the attribute $S$ is \define{empty} to mean
\begin{defn}
\item the carrier of $S$ is empty.
\end{defn}
\end{definition}

Observe there exists a strict empty 1-sorted structure and a strict
nonempty 1-sorted structure. Observe the carrier of an empty 1-sorted
structure is empty. We also observe the carrier of a nonempty 1-sorted
structure is nonempty.

\begin{definition}
  Let $S$ be a 1-sorted structure.
  We define the modes \define{Element of $S$} is an element of the
  carrier of $S$, and a \define{Subset of $S$} is a subset of the
  carrier of $S$, and a \define{Subset-Family of $S$} is a subset-family
  of the carrier of $S$.
\end{definition}

\begin{definition}
Let $S$ be a 1-sorted structure, let $X$ be a set.
We define the modes \define{Function of $S$, $X$} is a function from the
carrier of $S$ to $X$, and a \define{Function of $X$, $S$}
is a function from $X$ to the carrier of $S$.
\end{definition}

\begin{definition}
Let $S$ and $T$ be 1-sorted structures.
We define the mode \define{Function of $S$, $T$} is a function from the
carrier of $S$ to the carrier of $T$.
\end{definition}

\begin{definition}
Let $T$ be a 1-sorted structure.
We define the term $\emptyset_{T}$ (Mizar: ``\verb#{} T#'') to be the subset of $T$ equal to
\begin{defn}
\item $\emptyset_{T}=\emptyset$.
\end{defn}
We define the term $\Omega_{T}$ (Mizar: ``\verb|[#] T|'') to be the
subset of $T$ equal to
\begin{defn}
\item $\Omega_{T}=\carrier{T}$.
\end{defn}
\end{definition}

\begin{definition}
Let $S$ be a 1-sorted structure.
We define the mode a \define{Finite Sequence of $S$}
(Mizar: ``\verb#FinSequence of S#'')
is a finite sequence of the carrier of $S$.
\end{definition}

\begin{definition}
Let $S$ be a 1-sorted structure.
We define the mode a \define{Many Sorted Set of $S$} (Mizar:
``\verb#ManySortedSet of S#'') is a many sorted set of the carrier of $S$.
\end{definition}

\begin{definition}
Let $S$ be a 1-sorted structure.
We define the term $\id_{S}$ to be a function of $S$, $S$ equal to
\begin{defn}
\item $\id_{S}=\id_{\carrier{S}}$.
\end{defn}
\end{definition}

\begin{definition}
Let $S$ be a 1-sorted structure.
We define the mode \define{sequence of $S$} is a sequence of the carrier
of $S$.
\end{definition}

\begin{definition}
Let $S$ be a 1-sorted structure, let $X$ be a set.
We define the mode \define{Partial Function of $S$, $X$} (Mizar:
``\verb#PartFunc of S,X#'') is a partial function from the carrier of
$S$ to $X$.
We define the mode \define{Partial Function of $X$, $S$} (Mizar:
``\verb#PartFunc of X,S#'') is a partial function from $X$ to the carrier of
$S$.
\end{definition}

\begin{definition}
Let $S$, $T$ be 1-sorted structures.
We define the mode \define{Partial Function from $S$ to $T$}
(Mizar: ``\verb#PartFunc of S,T#'') is a partial function from the carrier of
$S$ to the carrier of $T$.
\end{definition}

\begin{definition}
Let $S$ be a 1-sorted structure, let $x$ be an object.
We define the predicate $x\in S$ (Mizar: ``\verb#x in S#'') to mean
\begin{defn}
\item $x\in\carrier{S}$.
\end{defn}
\end{definition}

\section{Pointed Structures}

\begin{definition}\index{ZeroF}
We define the system of \define{ZeroStr} which extends the 1-sorted
structures as
\[\langle \mbox{a carrier}, \mbox{a ZeroF}\rangle\]
where ZeroF is an element of the carrier.
\end{definition}

\begin{definition}\index{OneF}
We define the system of \define{OneStr} which extends the 1-sorted
structures as
\[\langle \mbox{a carrier}, \mbox{a OneF}\rangle\]
where OneF is an element of the carrier.
\end{definition}

\begin{definition}
We define the system of \define{ZeroOneStr} which extends the ZeroStr
and OneStr structures as
\[\langle \mbox{a carrier}, \mbox{a ZeroF}, \mbox{a OneF}\rangle\]
where ZeroF is an element of the carrier,
and OneF is an element of the carrier.
\end{definition}

\begin{definition}
Let $S$ be a ZeroStr structure. We define the term $0_{S}$ (Mizar:
``\verb#0. S#'') to be an element of $S$ equal to
\begin{defn}
\item $0_{S} = $ the ZeroF of $S$.
\end{defn}
\end{definition}

\begin{definition}
Let $S$ be a OneStr structure. We define the term $0_{S}$ (Mizar:
``\verb#1. S#'') to be an element of $S$ equal to
\begin{defn}
\item $1_{S} = $ the OneF of $S$.
\end{defn}
\end{definition}

\begin{definition}
Let $S$ be a ZeroOneStr structure.
We define the attribute $S$ is \define{degenerated} to mean
\begin{defn}
\item $0_{S}=1_{S}$.
\end{defn}
\end{definition}

\begin{definition}
Let $S$ be a 1-sorted structure.
We define the attribute $S$ is trivial to mean
\begin{defn}
\item the carrier of $S$ is trivial.
\end{defn}
\end{definition}

\begin{definition}
Let $S$ be a 1-sorted structure.
We can alternatively redefine the attribute that $S$ is \define{trivial}
to mean:
\begin{defn}
\item For any elements $x$ and $y$ of $S$, we have $x=y$.
\end{defn}
\end{definition}

Observe, nondegenerated ZeroOneStr structures are nontrivial.

\section{Finite 1-sorted Structures}

\begin{definition}
Let $S$ be a 1-sorted structure.
We define the attribute $S$ is \define{finite} to mean
\begin{defn}
\item the carrier of $S$ is finite.
\end{defn}
\end{definition}

\begin{definition}
Let $S$ be a ZeroStr structure, let $x$ be an element of $S$.
We define the attribute that $x$ is \define{zero} to mean
\begin{defn}
\item $x=0_{S}$.
\end{defn}
\end{definition}

\begin{definition}
Let $S$ be a 1-sorted structure.
We define the mode \define{Cover of $S$} is a cover of the carrier of $S$.
\end{definition}

\section{2-Sorted Structures}

\begin{definition}
We define a system of \define{2-sorted} structures which extends
1-sorted structures as
\[\langle\mbox{a carrier}, \mbox{a carrier'}\rangle\]
where carrier' is a set.
\end{definition}

\begin{definition}
Let $S$ be a 2-sorted structure.
We define the attribute $S$ is \define{void} to mean
\begin{defn}
\item the carrier' of $S$ is empty.
\end{defn}
\end{definition}

\begin{definition}
Let $X$ be a 1-sorted structure, $Y$ be a nonempty 1-sorted structure,
let $y$ be an element of $Y$.
We define the term $X\longmapsto y$ (Mizar: ``\verb#X --> y#'') to be a function from $X$ to $Y$ meaning
\begin{defn}
\item $X\longmapsto y=\carrier{X}\longmapsto y$
\end{defn}
\end{definition}

\begin{definition}
Let $X$ be a set, $S$ be a ZeroStr structure, $R$ a relation of $X$ and
the carrier of $S$.
We define the attribute $R$ is \define{non-zero} to mean
\begin{defn}
\item $0_{S}\notin\rng(R)$.
\end{defn}
\end{definition}

\begin{definition}
Let $S$ be a 1-sorted structure.
We define the term $\card{S}$ (Mizar: ``\verb#card S#'') to be a
Cardinal meaning
\begin{defn}
\item $\card{S}=\card{\carrier{S}}$.
\end{defn}
\end{definition}

\begin{definition}
Let $S$ be a 1-sorted structure.
We define the modes \define{Unary Operator of $S$} (Mizar:
``\verb#UnOp of S#'') is a Unary Operator of the carrier of $S$, and
\define{Binary Operator of $S$} (Mizar: ``\verb#BinOp of S#'') is a
binary operator of the carrier of $S$.
\end{definition}

\begin{definition}
Let $S$ be a ZeroStr structure.
We define the term \define{NonZero $S$} (Mizar: ``\verb#NonZero S#'')
is a subset of $S$ meaning
\begin{defn}
\item NonZero $S = \Omega_{S}\setminus\{0_{S}\}$.
\end{defn}
\end{definition}

We can prove the following result:
\begin{thm}
\item\label{struct0:1} Let $S$ be a nonempty ZeroStr structure, let $u$
  be an element of $S$. Then $u\in\mbox{NonZero\ }S$ iff $u$ is not zero.
\end{thm}

\begin{definition}
Let $V$ be a nonempty ZeroStr structure.
We redefine the attribute $V$ is \define{trivial} to mean
\begin{defn}
\item every element $u$ of $V$ satisfies $u=0_{V}$.
\end{defn}
\end{definition}

We can prove the following results:
\begin{thm}
\item\label{struct0:2} Let $F$ be a nondegenerated ZeroOneStr structure.
  Then $1_{F}\in\mbox{NonZero\ }F$.
\item\label{struct0:3} Let $S$ be a ZeroStr structure. Then $0_{S}\notin\NonZero(S)$.
\item\label{struct0:4} Let $S$ be a nonempty ZeroStr structure.
  Then $\carrier{S}=\{0_{S}\}\cup\NonZero(S)$.
\end{thm}

\begin{definition}
Let $C$ be a set and let $X$ be a 1-sorted structure.
We define the attribute $X$ is \define{C-element} to mean
\begin{defn}
\item The carrier of $X$ is $C$-element.
\end{defn}
\end{definition}

\begin{definition}
Let $S$ be a 2-sorted structure.
We define the attribute $S$ is \define{feasible} to mean
\begin{defn}
\item If the carrier of $S$ is empty, then the carrier' of $S$ is empty.
\end{defn}
\end{definition}

\begin{definition}
Let $S$ be a 2-sorted structure. We define the attribute $S$ is \define{trivial'}
to mean
\begin{defn}
\item the carrier' of $S$ is trivial.
\end{defn}
\end{definition}

\begin{definition}
Let $x$ be an object, let $S$ be a 1-sorted structure.
We define the term $\In{x}{S}$ to be an element of $S$ such that
\begin{defn}
\item $\In{x}{S}=\In{x}{\carrier{S}}$.
\end{defn}
\end{definition}

\begin{remark}
If we have a structure $S$ with fields $\langle F_{1},\dots,F_{n}\rangle$,
we could extend it by merging with other structures. For example, a
topological group has the fields of a group structure, and the fields of
a topological structure. But if we wanted to insist on looking at a
structure strictly with the fields from $S$, then we use the attribute
\define{strict} to do so.
\end{remark}

\end{document}