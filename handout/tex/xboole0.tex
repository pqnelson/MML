\documentclass{article}
\title{Boolean Properties of Sets: Definitions (XBOOLE-0)}
\author{Library Committee}
%% \makeatletter
%% \@ifclassloaded{combine}
%%   {\let\@begindocumenthook\@empty}
%%   {}
%% \makeatother
\begin{document}
\maketitle
%\section[Boolean Properties of Sets: Definitions]{Boolean Properties of Sets: Definitions (XBOOLE-0)}

\begin{scheme}[Separation]
  Let $P[-]$ be a unary predicate, $\mathcal{F}_{1}$ be a set.
  Then there exists a set $X$ such that for any object $x$, we have
  $x\in X$ if and only if $x\in\mathcal{F}_{1}$ and $P[x]$.
\end{scheme}

\begin{definition}
  Let $X$ be a set. We define the attribute $X$ is \define{Empty} to mean:
\begin{defn}
\item for any object $x$, we have $x\notin X$.
\end{defn}
\end{definition}

Observe there exists an empty set.

\begin{definition}
  We define the new term $\{\}$ or $\emptyset$ (Mizar: ``\verb#{}#'') to
  be the set equal to
  \begin{defn}
  \item the empty set.
  \end{defn}
  Let $X$, $Y$ be sets. We define the term $X\cup Y$ to be the set
  satisfying
  \begin{defn}
  \item for any object $x$, we have $x\in X\cup Y$ if and only if either
    $x\in X$ or $x\in Y$.
  \end{defn}
  We define the new term $X\cap Y$ to be the set satisfying
  \begin{defn}
  \item for any object $x$, we have $x\in X\cap Y$ if and only if $x\in X$ and $x\in Y$.
  \end{defn}
  We define the new term $X\setminus Y$ to be the set satisfying
  \begin{defn}
  \item for any object $x$, we have $x\in X\setminus Y$
    if and only if $x\in X$ and $x\notin Y$.
  \end{defn}
\end{definition}

\begin{definition}
Let $X$, $Y$ be sets. We define the \define{Symmetric Difference} of $X$
and $Y$ to be the new term $X\symdiff Y$ (Mizar: ``\verb#X \+\ Y#'') equal to
\begin{defn}
\item $(X\setminus Y)\cup(Y\setminus X)$
\end{defn}
Observe it is commutative $X\symdiff Y=Y\symdiff X$.

We define the predicate $X$ \define{Misses} $Y$ (or $X$ \emph{is disjoint with} $Y$, Mizar: ``\verb#X misses Y#'')
which means
\begin{defn}
\item $X\cap Y=\emptyset$
\end{defn}
Observe $X$ misses $Y$ iff $Y$ misses $X$, i.e., it is a symmetric predicate.

We define the predicate $X\properSubset Y$ (Mizar: ``\verb#X c< Y#'')
to mean
\begin{defn}
\item $X\subset Y$ and $X\neq Y$.
\end{defn}
Observe it is an irreflexive relation (there is no set $X$ which is a
proper subset of itself) and asymmetric (if $X\properSubset Y$, then
$\neg(Y\properSubset X)$).

We define the predicate $X$ and $Y$ are \define{$\subset$-Comparable}
(Mizar: ``\verb#X,Y are_c=-comparable#'') to mean:
\begin{defn}
\item $X\subset Y$ or $Y\subset X$.
\end{defn}
Observe it is reflexive ($X$ is always $\subset$-comparable to itself)
and symmetric ($X$ and $Y$ are $\subset-comparable$ implies $Y$ and $X$
are $\subset$-comparable).

We redefine the predicate $X=Y$ to mean
\begin{defn}
\item $X\subset Y$ and $Y\subset X$.
\end{defn}
\end{definition}

\begin{definition}
Let $X$, $Y$ be sets. We define $X$ \define{meets} $Y$ (Mizar:
``\verb#X meets Y#'') as the antonym of $X$ misses $Y$.
\end{definition}

We now have the following theorems. Let $X$, $Y$ be sets.

\begin{thm}
\item For any object $x$, we have $x\in X\symdiff Y$ if and only if
  ($x\in X$ and $x\notin Y$) or ($x\notin X$ and $x\in Y$).
\item Let $Z$ be a set. If every object $x$ is such that $x\notin X$ iff
  ($x\in Y$ iff $x\in Z$), then $X=Y\symdiff Z$.
\end{thm}

Observe $\emptyset$ is empty, $\{x\}$ is non-empty, and $\{x,y\}$ is
non-empty. We observe there exists a non-empty set.
When $D$ is a non-empty set, and $X$ is any set, we observe $D\cup X$ is
non-empty and $X\cup D$ is non-empty.

Let $X$, $Y$ be sets. We have the following theorems:
\begin{thm}[resume]
\item $X$ meets $Y$ if and only if there exists an object $x$ such that
  $x\in X$ and $x\in Y$.
\item $X$ meets $Y$ if and only if there exists an object $x$ such that
  $x\in X\cap Y$.
\item Let $x$ be any object. If $X$ misses $Y$, $x\in X\cup Y$, but we
  don't have both $x\in X$ and $x\notin Y$, then $x\in Y$ and $x\notin X$.
\end{thm}

\begin{scheme}[Extensionality]
Let $\mathcal{F}_{1}$, $\mathcal{F}_{2}$ be sets, let $P[-]$ be a unary predicate.
We have  $\mathcal{F}_{1}=\mathcal{F}_{2}$ provided:
\begin{enumerate}
\item Every object $x$ satisfies $x\in\mathcal{F}_{1}$ iff $P[x]$, and
\item Every object $x$ satisfies $x\in\mathcal{F}_{2}$ iff $P[x]$.
\end{enumerate}
\end{scheme}

\begin{scheme}[SetEq]
Let $P[-]$ be a unary predicate.
For any sets $X_{1}$ and $X_{2}$ such that every object $x$ satisfies
$x\in X_{1}$ iff $P[x]$, and every object $x$ satisfies $x\in X_{2}$ iff $P[x]$,
then we conclude $X_{1}=X_{2}$.
\end{scheme}

Let $X$, $Y$ be sets.
\begin{thm}
\item If $X\properSubset Y$, then there exists an object $x$ such that
  $x\in Y$ and $x\notin X$.
\item If $X\neq\emptyset$, then there exists an object $x$ such that
  $x\in X$.
\item If $X\properSubset Y$, then there exists an object $x$ such that
  $x\in Y$ and $X\subset Y\setminus\{x\}$.
\end{thm}

\begin{notation}
Let $X$, $Y$ be sets. We write $X\nsubset Y$ (Mizar: ``\verb#X C/= Y#'') as the antonym for
$X\subset Y$.
\end{notation}

\begin{notation}\hypertarget{notation:xboole0:nin}{}%
Let $x$ be an object, let $X$ be a set. We write $x\notin X$ (Mizar:
``\verb#x nin X#'') for the
antonym of $x\in X$.
\end{notation}

\end{document}
