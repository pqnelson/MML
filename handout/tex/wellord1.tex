\documentclass{article}
\title{The Well Ordering Relations}
\author{Grzegorz Bancerek}
\begin{document}
\maketitle

\begin{definition}
Let $R$ be a Relation, let $a$ be an object.
We define the term $R-\Seg(a)$ to be the set equal to
\begin{defn}
\item $\Coim{R}{a}\setminus\{a\}$.
\end{defn}
\end{definition}

Let $R$ be a Relation, let $a$ and $x$ be objects. We have the following
two theorems:
\begin{thm}
\item\label{wellord1:1} $x\in R-\Seg(a)$ iff $x\neq a$ and $(x,a)\in R$.
\item\label{wellord1:2} $x\in\field(R)$ or $R-\Seg(x)=\emptyset$.
\end{thm}

\begin{definition}
Let $R$ be a Relation. We define the attribute $R$ is \define{well founded}
to mean
\begin{defn}
\item for every set $Y$, if $Y\subset\field(R)$ and $Y\neq\emptyset$,
  then there exists an object $a$ such that $a\in Y$ and $R-\Seg(a)$
  misses $Y$.
\end{defn}
Let $X$ be a set. We define the predicate $R$ \define{is well founded in $X$}
to mean
\begin{defn}
\item For all sets $Y$, if $Y\subset X$ and $Y\neq\emptyset$,
  then there exists an object $a$ such that $a\in Y$ and $R-\Seg(a)$
  misses $Y$.
\end{defn}
\end{definition}

We can prove the following proposition:
\begin{thm}
\item\label{wellord1:3} $R$ is well founded iff $R$ is well founded in
  the field of $R$.
\end{thm}

\begin{definition}
Let $R$ be a Relation. We define the attribute $R$ is \define{well-ordering}
to mean
\begin{defn}
\item $R$ is reflexive, transitive, antisymmetric, connected, and well founded.
\end{defn}
\end{definition}

\begin{definition}
Let $R$ be a relation, let $X$ be a set. We define the predicate $R$
\define{well orders} $X$ to mean
\begin{defn}
\item $R$ is reflexive in $X$,
  $R$ is transitive in $X$,
  $R$ is antisymmetric in $X$,
  $R$ is connected in $X$, and
  $R$ is well founded in $X$.
\end{defn}
\end{definition}

We have the following results:
\begin{thm}
\item\label{wellord1:4} $R$ well orders the field of $R$ iff $R$ is well-ordering.
\item\label{wellord1:5} If $R$ well orders $X$,
  then for any set $Y$ such that $Y\subset X$ and $Y\neq\emptyset$
  there exists an object $a$ such that $a\in Y$ and $(a,b)\in R$ for every object $b\in Y$.
\item\label{wellord1:6} If $R$ is well-ordering,
  then for any set $Y$ such that $Y\subset\field(R)$ and $Y\neq\emptyset$,
  there exists an object $a$ such that $a\in Y$ and $(a,b)\in R$ for every object $b\in Y$.
\item\label{wellord1:7} If $R$ is well-ordering and $\field(R)\neq\emptyset$,
  then there exists an object $a$ such that $a\in\field(R)$ and
  $(a,b)\in R$ for every object $b\in\field(R)$.
\item\label{wellord1:8} If $R$ is a well-ordering and $a\in\field(R)$,
  then \begin{enumerate*}[label=(\roman*)]
  \item $(b,a)\in R$ for every $b\in\field(R)$ or
  \item there exists an object $b\in\field(R)$ such that $(a,b)\in R$
    and for every object $c\in\field(R)$ if $(a,c)\in R$ then $c=a$ or
    $(b,c)\in R$.
  \end{enumerate*}
\item\label{wellord1:9} $R-\Seg(a)\subset\field(R)$.
\end{thm}

\begin{definition}
Let $R$ be a relation, let $Y$ be a set.
We define the term $R|^{2}Y$ to be the relation equal to
\begin{defn}
\item $R\cap(Y\times Y)$.
\end{defn}
\end{definition}

\begin{thm}
\item\label{wellord1:10} $R|^{2}X=(R|^{X})|_{X}$.
\item\label{wellord1:11} $R|^{2}X=(R|_{X})|^{X}$.
\item\label{wellord1:12} If $x\in\field(R|^{2}X)$, then $x\in\field(R)$
  and $x\in X$.
\item\label{wellord1:13} $\field(R|^{2}X)\subset\field(R)$ and
  $\field(R|^{2}X)\subset X$.
\item\label{wellord1:14} $(R|^{2}X)-\Seg(a)\subset R-\Seg(a)$.
\item\label{wellord1:15} If $R$ is reflexive, then $R|^{2}X$ is reflexive.
\item\label{wellord1:16} If $R$ is connected, then $R|^{2}X$ is connected.
\item\label{wellord1:17} If $R$ is transitive, then $R|^{2}X$ is transitive.
\item\label{wellord1:18} If $R$ is antisymmetric, then $R|^{2}X$ is antisymmetric.
\item\label{wellord1:19} $(R|^{2}X)|^{2}Y=R|^{2}(X\cap Y)$.
\item\label{wellord1:20} $(R|^{2}X)|^{2}Y=(R|^{2}Y)|^{2}X$.
\item\label{wellord1:21} $(R|^{2}Y)|^{2}Y=R|^{2}Y$.
\item\label{wellord1:22} If $Z\subset Y$, then $(R|^{2}Y)|^{2}Z=R|^{2}Z$.
\item\label{wellord1:23} $R|^{2}\field(R)=R$.
\item\label{wellord1:24} If $R$ is well found, then $R|^{2}X$ is well founded.
\item\label{wellord1:25} If $R$ is well-ordering, then $R|^{2}X$ is well-ordering.
\item\label{wellord1:26} If $R$ is well-ordering, then $R-\Seg(a)$ is
  $\subset$-comparable to $R-\Seg(b)$.
\item\label{wellord1:27} If $R$ is well-ordering and $b\in R-\Seg(a)$,
  then $(R|^{2}(R-\Seg(a)))-\Seg(b)=R-\Seg(b)$.
\item\label{wellord1:28}%%  If $R$ is well-ordering and $Y\subset\field(R)$,
  %% then the following are logically equivalent:
  %% \begin{enumerate}[label=(\roman*)]
  %% \item $Y=\field(R)$
  %% \end{enumerate}
\item\label{wellord1:29}
\item\label{wellord1:3}
\end{thm}

\end{document}