\documentclass{article}
\title{The Well Ordering Relations (WELLORD1)}
\author{Grzegorz Bancerek}
\begin{document}
\maketitle

\begin{definition}
Let $R$ be a Relation, let $a$ be an object.
We define the term $R-\Seg(a)$ to be the set equal to
\begin{defn}
\item $\Coim{R}{a}\setminus\{a\}$.
\end{defn}
\end{definition}

Let $R$ be a Relation, let $a$ and $x$ be objects. We have the following
two theorems:
\begin{thm}
\item\label{wellord1:1} $x\in R-\Seg(a)$ iff $x\neq a$ and $(x,a)\in R$.
\item\label{wellord1:2} $x\in\field(R)$ or $R-\Seg(x)=\emptyset$.
\end{thm}

\begin{definition}
Let $R$ be a Relation. We define the attribute $R$ is \define{well founded}
to mean
\begin{defn}
\item for every set $Y$, if $Y\subset\field(R)$ and $Y\neq\emptyset$,
  then there exists an object $a$ such that $a\in Y$ and $R-\Seg(a)$
  misses $Y$.
\end{defn}
Let $X$ be a set. We define the predicate $R$ \define{is well founded in $X$}
to mean
\begin{defn}
\item For all sets $Y$, if $Y\subset X$ and $Y\neq\emptyset$,
  then there exists an object $a$ such that $a\in Y$ and $R-\Seg(a)$
  misses $Y$.
\end{defn}
\end{definition}

We can prove the following proposition:
\begin{thm}
\item\label{wellord1:3} $R$ is well founded iff $R$ is well founded in
  the field of $R$.
\end{thm}

\begin{definition}
Let $R$ be a Relation. We define the attribute $R$ is \define{well-ordering}
to mean
\begin{defn}
\item $R$ is reflexive, transitive, antisymmetric, connected, and well founded.
\end{defn}
\end{definition}

\begin{definition}
Let $R$ be a relation, let $X$ be a set. We define the predicate $R$
\define{well orders} $X$ to mean
\begin{defn}
\item $R$ is reflexive in $X$,
  $R$ is transitive in $X$,
  $R$ is antisymmetric in $X$,
  $R$ is connected in $X$, and
  $R$ is well founded in $X$.
\end{defn}
\end{definition}

We have the following results:
\begin{thm}
\item\label{wellord1:4} $R$ well orders the field of $R$ iff $R$ is well-ordering.
\item\label{wellord1:5} If $R$ well orders $X$,
  then for any set $Y$ such that $Y\subset X$ and $Y\neq\emptyset$
  there exists an object $a$ such that $a\in Y$ and $(a,b)\in R$ for every object $b\in Y$.
\item\label{wellord1:6} If $R$ is well-ordering,
  then for any set $Y$ such that $Y\subset\field(R)$ and $Y\neq\emptyset$,
  there exists an object $a$ such that $a\in Y$ and $(a,b)\in R$ for every object $b\in Y$.
\item\label{wellord1:7} If $R$ is well-ordering and $\field(R)\neq\emptyset$,
  then there exists an object $a$ such that $a\in\field(R)$ and
  $(a,b)\in R$ for every object $b\in\field(R)$.
\item\label{wellord1:8} If $R$ is a well-ordering and $a\in\field(R)$,
  then \begin{enumerate*}[label=(\roman*)]
  \item $(b,a)\in R$ for every $b\in\field(R)$ or
  \item there exists an object $b\in\field(R)$ such that $(a,b)\in R$
    and for every object $c\in\field(R)$ if $(a,c)\in R$ then $c=a$ or
    $(b,c)\in R$.
  \end{enumerate*}
\item\label{wellord1:9} $R-\Seg(a)\subset\field(R)$.
\end{thm}

\begin{definition}
Let $R$ be a relation, let $Y$ be a set.
We define the term $R|^{2}Y$ to be the relation equal to
\begin{defn}
\item $R\cap(Y\times Y)$.
\end{defn}
\end{definition}

\begin{thm}
\item\label{wellord1:10} $R|^{2}X=(R|^{X})|_{X}$.
\item\label{wellord1:11} $R|^{2}X=(R|_{X})|^{X}$.
\item\label{wellord1:12} If $x\in\field(R|^{2}X)$, then $x\in\field(R)$
  and $x\in X$.
\item\label{wellord1:13} $\field(R|^{2}X)\subset\field(R)$ and
  $\field(R|^{2}X)\subset X$.
\item\label{wellord1:14} $(R|^{2}X)-\Seg(a)\subset R-\Seg(a)$.
\item\label{wellord1:15} If $R$ is reflexive, then $R|^{2}X$ is reflexive.
\item\label{wellord1:16} If $R$ is connected, then $R|^{2}X$ is connected.
\item\label{wellord1:17} If $R$ is transitive, then $R|^{2}X$ is transitive.
\item\label{wellord1:18} If $R$ is antisymmetric, then $R|^{2}X$ is antisymmetric.
\item\label{wellord1:19} $(R|^{2}X)|^{2}Y=R|^{2}(X\cap Y)$.
\item\label{wellord1:20} $(R|^{2}X)|^{2}Y=(R|^{2}Y)|^{2}X$.
\item\label{wellord1:21} $(R|^{2}Y)|^{2}Y=R|^{2}Y$.
\item\label{wellord1:22} If $Z\subset Y$, then $(R|^{2}Y)|^{2}Z=R|^{2}Z$.
\item\label{wellord1:23} $R|^{2}\field(R)=R$.
\item\label{wellord1:24} If $R$ is well found, then $R|^{2}X$ is well founded.
\item\label{wellord1:25} If $R$ is well-ordering, then $R|^{2}X$ is well-ordering.
\item\label{wellord1:26} If $R$ is well-ordering, then $R-\Seg(a)$ is
  $\subset$-comparable to $R-\Seg(b)$.
\item\label{wellord1:27} If $R$ is well-ordering and $b\in R-\Seg(a)$,
  then $(R|^{2}(R-\Seg(a)))-\Seg(b)=R-\Seg(b)$.
\item\label{wellord1:28} Suppose $R$ is well-ordering and $Y\subset\field(R)$.
  then the following are logically equivalent:
  \begin{enumerate}[label=(\roman*)]
  \item $Y=\field(R)$ or there exists an object $a\in\field(R)$ such
    that $Y=R-\Seg(a)$.
  \item For all objects $a\in Y$ and for all objects $b$, if $(b,a)\in R$,
    then $b\in Y$
  \end{enumerate}
\item\label{wellord1:29} Suppose $R$ is well-ordering, $a\in\field(R)$, and
  $b\in\field(R)$. Then $(a,b)\in R$ if and only if $R-\Seg(a)\subset R-\Seg(b)$.
\item\label{wellord1:30} Suppose $R$ is well-ordering, $a\in\field(R)$, and
  $b\in\field(R)$. Then $R-\Seg(a)\subset R-\Seg(b)$ if and only if
  either $a=b$ or $a\in R-\Seg(b)$.
\item\label{wellord1:31} If $R$ is well-ordering and
  $X\subset\field(R)$, then $\field(R|^{2}X)=X$.
\item\label{wellord1:32} If $R$ is well-ordering, then $\field(R|^{2}R-\Seg(a))=R-\Seg(a)$.
\item\label{wellord1:33} Suppose $R$ is well-ordering. If every object
  $a\in\field(R)$ has $R-\Seg(a)\subset Z$ imply $a\in Z$, then
  $\field(R)\subset Z$.
\item\label{wellord1:34} Suppose $R$ is well-ordering, $a\in\field(R)$,
  and $b\in\field(R)$. If every object $c\in R-\Seg(a)$ has $(c,b)\in R$
  and $c\neq b$, then $(a,b)\in R$.
\item\label{wellord1:35} Suppose $R$ is well-ordering,
  $\dom(F)=\field(R)$, and $\rng(F)\subset\field(R)$.
  If all distinct objects $a\neq b$ has $(a,b)\in R$ imply
  $(F(a),F(b))\in R$ with $F(a)\neq F(b)$,
  then every object $a\in\field(R)$ has $(a,F(a))\in R$.
\end{thm}

\begin{definition}
Let $R$, $S$ be relations, let $F$ be a function.
We define the predicate $f$ \define{is isomorphism of} $R$, $S$ mean
\begin{defn}
\item $\dom(F)=\field(R)$, $\rng(F)=\field(S)$, $F$ is one-to-one,
  and for any objects $a$, $b$ we have $(a,b)\in R$ iff $a\in\field(R)$
  and $b\in\field(R)$ and $(F(a),F(b))\in S$.
\end{defn}
\end{definition}

We have the following result:
\begin{thm}
\item\label{wellord1:36} Suppose $F$ is an isomorphism of $R$, $S$.
  For all objects $a$ and $b$, if $(a,b)\in R$ and $a\neq b$,
  then $(F(a),F(b))\in S$ and $F(a)\neq F(b)$.
\end{thm}

\begin{definition}
Let $R$, $S$ be relations. We define the predicate $R$ and $S$
\define{are isomorphic} to mean
\begin{defn}
\item There exists a function $F$ such that $F$ is an isomorphism of
  $R$, $S$.
\end{defn}
\end{definition}

\begin{thm}
\item\label{wellord1:37} $\id_{\field(R)}$ is an isomorphism of $R$, $R$.
\item\label{wellord1:38} $R$, $R$ are isomorphic.
\item\label{wellord1:39} If $F$ is an isomorphism of $R$, $S$, then
  $F^{-1}$ is an isomorphism of $S$, $R$.
\item\label{wellord1:40} If $R$ and $S$ are isomorphic, then $S$ and $R$
  are isomorphic.
\item\label{wellord1:41} If $F$ is an isomorphism of $R$ with $S$, and
  $G$ is an isomorphism of $S$ with $T$, then $G\circ F$ is an
  isomorphism of $R$ with $T$.
\item\label{wellord1:42} If $R$ and $S$ are isomorphic, and $S$ and $T$
  are isomorphic, then $R$ and $T$ are isomorphic.
\item\label{wellord1:43} Suppose $F$ is an isomorphism of $R$ with
  $S$. Then
  \begin{enumerate}[label=(\roman*)]
  \item If $R$ is reflexive, then $S$ is reflexive; and
  \item If $R$ is transitive, then $S$ is transitive; and
  \item If $R$ is connected, then $S$ is connected; and
  \item If $R$ is antisymmetric, then $S$ is antisymmetric; and
  \item If $R$ is well founded, then $S$ is well founded.
  \end{enumerate}
\item\label{wellord1:44} If $R$ is well-ordering and $F$ is an
  isomorphism of $R$ with $S$, then $S$ is a well-ordering.
\item\label{wellord1:45} Suppose $R$ is a well-ordering. Then every
  functions $F$ and $G$ such that $F$ is an isomorphism of $R$ with $S$,
  and $G$ is an isomorphism of $R$ with $S$, implies $F=G$.
\end{thm}

\begin{definition}
Let $R$ and $S$ be isomorphic relations. Assume $R$ is well-ordering.
We define the term \define{canonical isomorphism of $R$, $S$}
(Mizar: ``\verb#canonical_isomorphism_of(R,S)#'') to be the function
such that
\begin{defn}
\item it is an isomorphism of $R$, $S$.
\end{defn}
\end{definition}

\begin{thm}
\item\label{wellord1:46} Suppose $R$ is well-ordering.
  Then for any object $a\in\field(R)$, we have $R$ is not isomorphic to $R|^{2}(R-\Seg(a))$.
\item\label{wellord1:47} If $R$ is well-ordering, $a\in\field(R)$,
  $b\in\field(R)$, and $a\neq b$, then $R|^{2}(R-\Seg(a))$ is not
  isomorphic with $R|^{2}(R-\Seg(b))$.
\item\label{wellord1:48} Suppose $R$ is well-ordering,
  $Z\subset\field(R)$, and $F$ is an isomorphism of $R$ with $S$.
  Then $F|_{Z}$ is an isomorphism of $R|^{2}Z$ with $S|^{2}F(Z)$, and in
  particular $R|^{2}Z$ is isomorphic to $S|^{2}F(Z)$.
\item\label{wellord1:49} Suppose $F$ is an isomorphism of $R$ with $S$.
  Then for each object $a\in\field(R)$, there exists an object
  $b\in\field(S)$ such that $F(R-\Seg(a))=S-\Seg(b)$.
\item\label{wellord1:50} Suppose $R$ is well-ordering and $F$ is
  isomorphism of $R$ with $S$.
  Then for each object $a\in\field(R)$ there exists an object $b\in\field(S)$
  such that $R |^{2} (R-\Seg(a))$ is isomorphic with $S |^2 (S-\Seg(b))$.
\item\label{wellord1:51} Suppose $R$ is well-ordering,
  $S$ is well-ordering, $a\in\field(R)$, $b\in\field(S)$, and
  $c\in\field(S)$.
  If $R$ is isomorphic with $S |^{2} (S-\Seg(b))$,
  $R |^{2} (R-\Seg(a))$ is isomorphic with $S|^{2} (S-\Seg(c))$
  then $S-\Seg(c) \subset S-\Seg(b)$ and $(c,b)\in S$.
\item\label{wellord1:52} If $R$ is well-ordering and $S$ is
  well-ordering, then one of the following holds:
  \begin{enumerate}[label=(\roman*)]
  \item $R$ is isomorphic with $S$, or
  \item There exists an object $a\in\field(R)$ such that $R |^{2} (R-\Seg(a))$
    is isomorphic to $S$, or
  \item There exists an object $a\in\field(S)$ such that $R$ is
    isomorphic with $S |^{2} (S-\Seg(a))$.
  \end{enumerate}
\item\label{wellord1:53} Suppose $Y\subset\field(R)$ and $R$ is
  well-ordering.
  Then either $R$ is isomorphic with $R |^{2} Y$, or
  there is an object $a\in\field(R)$ such that
  $R |^{2} (R-\Seg(a))$ is isomorphic with $R |^{2} Y$.
\item\label{wellord1:54} If $R$ and $S$ are isomorphic and $R$ is well-ordering, then
  $S$ is well-ordering.
\end{thm}


\end{document}