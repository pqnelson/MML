\documentclass{article}

\title[Zermelo Theorem and Axiom of Choice (WELLORD2)]{Zermelo Theorem and Axiom of Choice. The correspondence of well ordering relations and ordinal numbers (WELLORD2)}
\author{Grzegorz Bancerek}
\date{June 26, 1989}
\begin{document}
\maketitle

\begin{definition}
Let $X$ be a set. We define the term $\RelIncl{X}$ (Mizar:
``\verb#RelIncl X#'') to be the relation satisfying
\begin{defn}
\item $\field(\RelIncl{X})=X$, and for all sets $Y$ and $Z$ if $Y\in X$
and $Z\in X$, then $(Y,Z)\in\RelIncl{X}$ iff $Y\subset Z$.
\end{defn}
\end{definition}
When $X$ is a set, observe $\RelIncl{X}$ is reflexive, transitive, antisymmetric.
When $A$ is an Ordinal, $\RelIncl{A}$ is connected and well founded.

Let $X$, $Y$, $Z$ be sets. Let $R$ be a relation. Let $A$, $B$, $C$ be Ordinals.
We have the following results:
\begin{thm}
\item\label{wellord2:1} (Cancelled)
\item\label{wellord2:2} (Cancelled)
\item\label{wellord2:3} (Cancelled)
\item\label{wellord2:4} (Cancelled)
\item\label{wellord2:5} (Cancelled)
\item\label{wellord2:6} (Cancelled)
\item\label{wellord2:7} If $Y\subset X$, then $\RelIncl{X}|^{Y}_{Y}=\RelIncl{Y}$.
\item\label{wellord2:8} For all ordinals $A$ and sets $X$,
  if $X\subset A$, then $\RelIncl{X}$ is well-ordering.
\item\label{wellord2:9} If $A\in B$, then $A=(\RelIncl{B})-\Seg(A)$.
\item\label{wellord2:10} If $\RelIncl{A}$ is isomorphic to $\RelIncl{B}$,
  then $A=B$.
\item\label{wellord2:11} If $R$ is isomorphic to $\RelIncl{A}$,
  and if $R$ and $\RelIncl{B}$ are isomorphic,
  then $A=B$.
\item\label{wellord2:12} If $R$ is well-ordering, then there exists an
  Ordinal $A$ such that $R$ is isomorphic to $\RelIncl{A}$.
\item\label{wellord2:13} 
\end{thm}

\begin{definition}
Let $R$ be a relation. Assume $R$ is well-ordering.
We define the term \define{order type of $R$} to be an Ordinal
satisfying
\begin{defn}
\item $R$ and $\RelIncl{\mbox{order type of $R$}}$ are isomorphic.
\end{defn}
\end{definition}

\begin{definition}
Let $A$ be an Ordinal, let $R$ be a relation.
We define the predicate \define{$A$ is order type of $R$} to mean
\begin{defn}
\item $A = $ order type of $R$.
\end{defn}
\end{definition}

\begin{thm}
\item\label{wellord2:14} If $X\subset A$, then the order type of
  $\RelIncl{X}\subset A$.
\end{thm}

\begin{definition}
Let $X$, $Y$ be sets.
We redefine the predicate $X$ and $Y$ \define{are equipotent}
means
\begin{defn}
\item There exists a function $f$ such that $f$ is one-to-one and
  $\dom(f)=X$ and $\rng(f)=Y$.
\end{defn}
\end{definition}

\begin{remark}
The $\rng(f)=Y$ condition is the same as $f$ being surjective. Hence
this definition matches our intuition that equipotent sets are the same
thing as bijective sets.
\end{remark}

\begin{thm}
\item\label{wellord2:15} If $X\equipotent Y$ and $Y\equipotent Z$ are
  equipotent,
  then $X\equipotent Z$ is equipotent.
\item\label{wellord2:16} If $R$ well orders $X$,
  then $\field(R|^{X}_{X})=X$ and $R|^{X}_{X}$ is well-ordering.
\item\label{wellord2:17} (\textsc{Zermelo's Theorem}\index{Zermelo's Theorem})
  For any set $X$, there exists a relation $R$ such that $R$ well orders $X$.
\item\label{wellord2:18} (\textsc{Axiom of Choice}\index{Axiom of Choice}\index{Choice!Axiom})
  Let $M$ be a nonempty set.
  If every $X\in M$ is nonempty,
  and if every $X\in M$ and $Y\in M$ with $X\neq Y$ has $X$ misses $Y$,
  then there exists a set $C$ such that for all sets $X\in M$ there
  exists an object $x$ such that $C\cap X=\{x\}$.
\item\label{wellord2:19} $\RelIncl{X}$ is reflexive in $X$.
\item\label{wellord2:20} $\RelIncl{X}$ is transitive in $X$.
\item\label{wellord2:21} $\RelIncl{X}$ is antisymmetric in $X$.
\end{thm}
Observe $\RelIncl\emptyset$ is empty.
Let $X$ be a nonempty set, then we observe $\RelIncl{X}$ is nonempty.

We can prove the following two propositions:
\begin{thm}
\item\label{wellord2:22} $\RelIncl{\{x\}}=\{(x,x)\}$.
\item\label{wellord2:23} $\RelIncl{X}\subset X\times X$.
\end{thm}

\end{document}